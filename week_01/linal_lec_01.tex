\documentclass[14pt,xcolor=dvipsnames]{beamer}

\usetheme{metropolis}
\metroset{
  %progressbar=none,
  numbering=none,
  subsectionpage=progressbar,
  block=fill
}

\usepackage{amsmath, amsthm, amsfonts}

\usepackage{fontspec}
\usepackage{polyglossia}
\setmainlanguage{russian}

% download "Linux Libertine" fonts:
% http://www.linuxlibertine.org/index.php?id=91&L=1
\setmainfont{Myriad Pro} % or Helvetica, Arial, Cambria
\setsansfont{Myriad Pro}



\usepackage{unicode-math}
\usepackage{centernot}

\usepackage{graphicx}
\graphicspath{{img/}}

\usepackage{physics} % \abs \norm, переопределяет \cos, \sin, и т.д.


\usepackage{wrapfig}
\usepackage{animate}
\usepackage{tikz}
\usetikzlibrary{shapes.geometric,patterns,positioning,matrix,calc,arrows,shapes,fit,decorations,decorations.pathmorphing}

\usepackage{pifont}
\usepackage{comment}
\usepackage[font=small,labelfont=bf]{caption}
\captionsetup[figure]{labelformat=empty}
\includecomment{techno}

%\usefonttheme[onlymath]{serif}

\definecolor{hseblue}{RGB}{0,90,171} % это вышкинский голубой
\setbeamercolor{alerted text}{fg=Orange}  % or fg=hseblue



\colorlet{mc}{mLightBrown}
\definecolor{cb}{RGB}{160, 82, 45}
\definecolor{cw}{RGB}{254, 218, 186}
%\colorlet{myred}{RedOrange}
\colorlet{myred}{BrickRed}
\colorlet{mygreen}{ForestGreen}
\colorlet{myblue}{RoyalBlue}
\colorlet{mypurple}{Purple}
\colorlet{myorange}{Orange}
\colorlet{myyellow}{Goldenrod}
\colorlet{mybrown}{Brown}
\colorlet{eggplant}{Plum}

% Набор команд для удобства верстки

\newcommand{\RR}{\mathbb{R}}

% Набор команд для структуризации

\newcommand{\subheader}[1]{\vspace*{-15 pt}\hfill \textcolor{gray}{#1} \hfill}
\newcommand{\tsld}{\textcolor{Orange}{Технический слайд с комментариями}}
\newcommand{\quest}{\textcolor{RoyalBlue}{\boxed{\bf ?}} }
\newcommand{\acc}[1]{\bf #1}

\newcommand{\treepic}[2]{
\subheader{#1}
\vspace{-5mm}
\begin{center}
\includegraphics[scale=0.9]{#2}
\end{center}
}

\usepackage{listings}
\lstset{
  basicstyle=\small, 
  language=Python,
  %tabsize = 2,
  backgroundcolor=\color{mc!20!white}
}

\tikzstyle{miss}=[draw=none,fill=gray]

\newenvironment{mypic}
{\begin{center}\begin{tikzpicture}[line width=1.5pt]}
{\end{tikzpicture}\end{center}}

\newcommand{\mypart}[1]{\begin{frame}[standout]{\huge #1}\end{frame}}

\setbeamercolor{background canvas}{bg=}
% frame title setup
\setbeamercolor{frametitle}{bg=,fg=gray}
\setbeamertemplate{frametitle}[default][left]

\addtobeamertemplate{frametitle}{}{\vspace*{0.25cm}}

\AtBeginSection[]{\frame{\frametitle{Содержание лекции}\tableofcontents[current]}}

%\date{}

\newcommand{\drawchessboard}{\path[fill=cw] (0,0) rectangle (8,8);
\foreach \x in {0,...,7} \foreach \y in {0,...,7} {
    \pgfmathparse{Mod(\x+\y,2)==0?1:0}
    \ifnum\pgfmathresult>0
      \draw[draw=none,fill=cb] (\x,\y) rectangle (\x+1,\y+1);
    \fi  
}}

\newcommand{\myintropic}[4]{
\begin{minipage}{\textwidth-#1-5mm}
#4
\end{minipage}
\null\hfill
\begin{minipage}{#1+1mm}
\includegraphics[width=#1]{#2}\\
\tiny \textcolor{gray}{#3}
\end{minipage}
} 

\newcommand{\myyes}{\textcolor{mc}{\ding{51}}}
\newcommand{\myno}{{\ding{55}}}
\DeclareMathOperator{\Lin}{\mathrm{Lin}}

\tikzstyle{rb}=[right,text width=55mm,inner sep=2mm]
\tikzstyle{M}=[matrix of nodes,ampersand replacement=\&, line width=.5mm, row sep=-\pgflinewidth, column sep=-\pgflinewidth, nodes={Mnodes}] 

\newcommand\notdiv{\mathrel/\joinrel\mkern-4mu\mid}
\newcommand\R{\mathbb{R}}




\title{Линейная алгебра:\\ от идеи к формуле}

\date{}
\author{Борис Демешев}
\institute{НИУ ВШЭ}

\begin{document}

% \maketitle


\begin{frame} % название лекции

Векторы и действия с ними

\end{frame}


\begin{frame} % название фрагмента

Вектор: длина и скалярное произведение 
  
\end{frame}
  

% Видео 1. Вектор, скалярное произведение, длина вектора, угол между векторами, условие ортогональности. Косинусная близость (cosine similarity). Обозначение $R^n$. 
% Видео 2. Альтернативные метрики (манхэттен, минковский). Ядерные функции. Уравнение гиперплоскости как равенство скалярного произведения константе. 
% Видео 3. Линейные операторы. Определение. Примеры: отражение, растяжение вдоль направления, домножение на константу. Ортогональный оператор. 
% Видео 4. Ещё примеры: повороты, перестановка компонент вектора. Примеры операторов из $R^n$ в $R^k$.
% Видео 5. Линейные операторы, продолжение. Последовательное применение операторов. Проекция. 
% Видео 6. Линейные операторы. Определения. Обращение. Транспонирование. Собственные числа и собственные вектора. Спектр линейного оператора. 
% Видео 7. (доска) Обращение: поворот, отражение, проекция.
% Видео 8. (доска) Транспонирование. Примеры: поворот, отражение. 
% Видео 9.: (доска) Собственные числа и векторы. Поворот, проекция, растяжение вдоль направления.  
% Видео 10. (доска) Проекция: транспонирование, собственные векторы и собственные числа. 



\begin{frame}{Краткое напутствие}

%\uncover<1->{
\begin{block}{Зачем нужна линейная алгебра?}
\begin{itemize}
  \item Линейная алгебра прекрасна сама по себе!
  \item Работает «под капотом» практически всех методов машинного обучения.
\end{itemize}
\end{block}
%}
\end{frame}



\begin{frame}{Краткий план:}
  \begin{itemize}
    \item Вектор — это столбец чисел.
    \item Сложение двух векторов и умножение на число.
    \item Расстояние и косинус угла между векторами.
  \end{itemize}
  
\end{frame}


\begin{frame}{Вектор}


\begin{itemize}
\item Рабочее определение. \alert{Вектор} — столбец из нескольких чисел.   

\[
v = \begin{pmatrix}
  \sqrt{5} \\
  3 \\
  -3.45
\end{pmatrix}
\]
\item Мы не пишем стрелочку над вектором.

\item Идея вектора. Вектор — всё, что можно описать столбцом из нескольких чисел. 
\end{itemize}

\end{frame}
  



 


\begin{frame}{Длина вектора}

    \alert{Евклидова длина} вектора:

    Определение. \alert{Длина} или \alert{норма} вектора $\norm{x} = \sqrt{x_1^2 + x_2^2 + \ldots + x_n^2}$.

    \begin{block}{TODO: картинка с теоремой Пифагора}

    \end{block}
    
\end{frame}



\begin{frame}{Простая поэлементная арифметика}

\begin{itemize}
  \item \alert{Сложение и вычитание} двух векторов:
    \[
    \begin{pmatrix}
      2 \\
      3.5 \\
      -1 
    \end{pmatrix} + \begin{pmatrix}
      3 \\
      -3 \\
      1 
    \end{pmatrix}  = \begin{pmatrix}
      5 \\
      0.5 \\
      0 
    \end{pmatrix}
    \]  
  
  \item \alert{Умножение} вектора на число:
    \[
    4 \cdot \begin{pmatrix}
      2 \\
      3.5 \\
      -1 
    \end{pmatrix} = \begin{pmatrix}
      8 \\
      14 \\
      -4 
    \end{pmatrix}  
    \]
  \end{itemize}
  
  
  \end{frame}

  

\begin{frame}{Простая геометрия}

\begin{block}{TODO: картинка}
геометрия суммы векторов и произведение число на вектор
\end{block}


\end{frame}




\begin{frame}{Расстояние между векторами}

Определение. \alert{Евклидово расстояние} между векторами
  \[
  d(a, b) = \norm{a - b} = \sqrt{(a_1 - b_1)^2 + \ldots + (a_n - b_n)^2}
  \]

\begin{block}{TODO: картинка с расстоянием между векторами}

\end{block}

\begin{itemize}
  \item по определению, $d(a, b) \geq 0$. 
  \item также говорят \alert{Евклидова метрика}
\end{itemize}


\end{frame}





\begin{frame}{Пространство $\R^n$}

 \uncover<1->{
   \begin{block}{Пространство $\R^n$:}
     Множество всех возможных векторов из $n$ чисел. 
   \end{block}
 \[
 \R^n = \left\{ \begin{pmatrix}
 x_1 \\
 x_2 \\
 \vdots \\
 x_n \\
 \end{pmatrix} \middle| x_1 \in \R, \ldots, x_n \in \R
   \right\}  
 \]
}  

\uncover<2->{
\begin{block}{Размерность пространства $\R^n$:}
    Количество чисел в каждом векторе, $n$.
\end{block}
}
  
\end{frame}
  



\begin{frame}{Скалярное произведение и угол}

\begin{itemize}
\item Скалярное произведение векторов $a$ и $b$:
  $\langle a, b \rangle = a_1 b_1 + a_2 b_2 + \ldots + a_n b_n$.

\item Косинус угла между векторами $a$ и $b$:
Косинусная близость, cosine similarity:  
  \[
  \cos \angle (a, b) =  \frac{\langle a, b \rangle}{ \norm{a} \norm{b}} 
  \]
Косинус определён, если $\norm{a} > 0$ и $\norm{b} > 0$.

\item Первая теоремка!
  $\langle a, a \rangle = \norm{a}^2$
\end{itemize}

\end{frame}




\begin{frame}{Вектор как направленный отрезок}
  
  \begin{block}{TODO: Картинка, где изображён угол между векторами}
    
    
  \end{block}
\end{frame}
   

% \begin{frame}{Почти любой объект — вектор!}
  
%   \begin{block}{С помощью вектора можно закодировать:}
%   \begin{itemize}
%     \item многочлен 
%     \[
%     3x^2 + 6 x - 7 \to (3, 6, -7)  
%     \]
%     \item характеристики индивида
%     \[
%     \text{Блондин с ростом 182 см и весов 81 килограмм} \to (0, 182, 81)
%     \]
%     \item TODO
%   \end{itemize}

%   \end{block}
% \end{frame}




\begin{frame}{Ортогональность векторов}
  
\begin{block}{Векторы $a$ и $b$ ортогональны, если $a\perp b$,}
\[
  \langle a, b \rangle =0
\]
Также говорят «перпендикулярны».
\end{block}
  
\begin{block}{TODO: картинка}
Векторы $a$ и $b$ ортогональны, векторы $a$ и $c$ нет.
\end{block}

\end{frame}
  




\begin{frame}
  Гиперплоскость 
\end{frame}

\begin{frame}{Краткий план:}

\begin{block}{Вокруг метрик и скалярного произведения:}
\begin{itemize}[<+->]
  \item Да будет больше разных расстояний!
  \item Делаем из вектора прямую и гиперплоскость.
  \item Ядерные функции из скалярного произведения.
\end{itemize}
\end{block}

\end{frame}


\begin{frame}{Больше метрик в студию!}
 \begin{block}{Манхэттэнская метрика}
  Расстояние по Майкопски:
  \[
  d(a, b) = \abs{a_1 - b_1}  + \abs{a_2 - b_2} + \ldots + \abs{a_n - b_n}
  \]
 \end{block}

\begin{block}{TODO:}
  Два вектора с евклидовым и манхэттенским расстоянием.
\end{block}
 
\end{frame}




\begin{frame}{У нас и у них}

 \begin{block}{TODO:}
 Рядом картинки Манхэттэна и Майкопа
 \end{block}

\end{frame}


\begin{frame}{Ещё больше метрик!}
\begin{block}{Метрика Чебышёва}
  \[
      d(a, b) = \max\left\{\abs{a_1 - b_1}, \abs{a_2 - b_2}, \ldots, \abs{a_n - b_n}\right\}
  \]
\end{block}
  
\begin{block}{Метрика Минковского}
  \[
      d_p(a, b) = \left(\sum_{i=1}^n\abs{a_i - b_i}^p\right)^{1/p}
  \]
\end{block}
\end{frame}

\begin{frame}{Частные случаи метрики Минковского}

\begin{block}{Евклидова метрика, $p=2$}
$
  \sqrt{(a_1 - b_1)^2 + \ldots + (a_n - b_n)^2} = d_2(a, b)
$
\end{block}

\begin{block}{Манхэттэнская метрика, $p=1$}
$
    \abs{a_1 - b_1}  + \abs{a_2 - b_2} + \ldots + \abs{a_n - b_n} = d_1(a, b)
$
\end{block}
  
\begin{block}{Метрика Чебышёва, $p\to \infty$}
$
    \max\left\{\abs{a_1 - b_1}, \ldots, \abs{a_n - b_n}\right\} = \lim_{p\to\infty} d_p(a, b)
$
\end{block}
\end{frame}

\begin{frame}{Вектор порождает прямую}

\begin{block}{Прямая порождённая вектором $a$, $\Lin a$}
множество векторов, получаемых при домножении вектора $a$ на произвольное число,
\end{block}
\[
\Lin a = \left\{t\cdot a \middle| t \in \R \right\}  
\]

\begin{block}{TODO: картинка прямой порожденной вектором}
\end{block}

\end{frame}


\begin{frame}{Вектор задаёт гиперплоскость}

Вектор $a$ фиксирован, например, $a=(1, 2, 3)$.

\begin{block}{TODO: две картинки рядом}
$\langle a, v \rangle = 0$ и $\langle a, v \rangle = 1$   
\end{block}


\end{frame}


\begin{frame}{Ядерные функции}

Векторная функция $f$ фиксирована, например, 
\[
  f : \begin{pmatrix}
    v_1 \\
    v_2 \\
  \end{pmatrix} \to 
  \begin{pmatrix}
    -1 \\
    v_1^2 + v_2^2 \\
  \end{pmatrix}
\]

\begin{block}{Ядерная функция, ядро $K$}
Скалярное произведение в спрямляющем пространстве:
$K(a, b) = \langle f(a), f(b) \rangle$.
\end{block}
\end{frame}

\begin{frame}
  \frametitle{Спрямляющее пространство:}

\begin{block}{TODO: картинка с исходным и спрямляющим пространством} 
  
\end{block}
  

\end{frame}


\begin{frame}
  
  Линейный оператор: первые шаги

\end{frame}


\begin{frame}{Линейный оператор}


\begin{block}{Идея линейности}
  Результат не изменится, если поменять местами действие $L$ и
  \begin{itemize}
    \item растягивание вектора, например, $L(42a)=42L(a)$;
    \item усреднение двух векторов, $L(0.5a+0.5b)=0.5L(a) + 0.5L(b)$.
  \end{itemize}
\end{block}

\end{frame}


\begin{frame}{Стандартное определение линейности}

\begin{block}{Линейная функция $L$ из $\R^n$ в $\R^k$}
\begin{itemize}
  \item Для любого числа $t$ и вектора $a \in \R^n$: $L(t a) = tL(a)$.
  \item Для любых двух векторов $a$ и $b$ из $\R^n$: $L(a + b) = L(a) + L(b)$.
\end{itemize}
\end{block}

\begin{block}{$L(a) \equiv La$}
\end{block}

\end{frame}


\begin{frame}{Растягивание координат}

\begin{block}{Обобщаем умножение вектора на число!}
  $L : \begin{pmatrix}
    a_1 \\
    a_2 \\
  \end{pmatrix} \to 
  \begin{pmatrix}
    2a_1 \\
    -3a_2 \\
  \end{pmatrix}$
\end{block}

\begin{block}{TODO: картинка}

\end{block}
\end{frame}



\begin{frame}{Перестановка координат вектора}

\begin{block}{На пути к произвольному повороту}
  $L : \begin{pmatrix}
    a_1 \\
    a_2 \\
    a_3 \\
  \end{pmatrix} \to 
  \begin{pmatrix}
    a_2 \\
    a_3 \\
    a_1 \\
  \end{pmatrix}$
\end{block}

\begin{block}{TODO: картинка}

\end{block}
    
\end{frame}




\begin{frame}{Обрезка компонент вектора}

\begin{block}{На пути к произвольной проекции}
  $L : \begin{pmatrix}
    a_1 \\
    a_2 \\
    a_3 \\
  \end{pmatrix} \to 
  \begin{pmatrix}
    a_1 \\
    a_2 \\
  \end{pmatrix}$
\end{block}

\begin{block}{TODO: картинка}

\end{block}
    
\end{frame}


\begin{frame}{Дописывание нулей}

\begin{block}{Увеличиваем размерность пространства}
  $L : \begin{pmatrix}
    a_1 \\
    a_2 \\
  \end{pmatrix} \to 
  \begin{pmatrix}
    a_1 \\
    0  \\
    a_2 \\
  \end{pmatrix}$
\end{block}

\begin{block}{TODO: картинка}

\end{block}
    
\end{frame}
  



\begin{frame}{Первая проекция}

\begin{block}{Проекция на прямую $x_1 + 2x_2 = 0$}
Оператор $H: \R^2 \to \R^2$ 

TODO: картинка для аргументации линейности
\end{block}

\end{frame}
  

\begin{frame}{Первый поворот}

\begin{block}{Поворот на $30^{\circ}$ против часовой стрелки}
Оператор $R: \R^2 \to \R^2$ 

TODO: картинка для аргументации линейности
\end{block}

\end{frame}
  
\begin{frame}{Ортогональный линейный оператор}

\begin{block}{Идея ортогональности}
  Действие $L$ не изменяет углов и расстояний. 
\end{block}


\begin{block}{Ортогональный оператор $L : \R^n \to \R^n$}
 Для любых векторов $a$ и $b$: $\langle La, Lb \rangle = \langle a, b\rangle$
\end{block}



\end{frame}
  

\begin{frame} % видеофрагмент
  Проекция и поворот на плоскости: формулы
\end{frame}

  
\begin{frame}{видео с ДОСКОЙ}

\begin{block}{вывод формулы поворота на плоскости}
\end{block}

\begin{block}{вывод формулы проекции на плоскости}
\end{block}


\end{frame}
  
\begin{frame} % видеофрагмент
  Ещё больше линейных операторов
\end{frame}

\begin{frame}{Композиция линейных операторов}

\begin{block}{Делай раз, делай два!}
Если последовательно применить два линейных действия, то получится линейное действие, $L_2 (L_1 (a)) = L(a)$.
\end{block}

\begin{block}{доказательство}
  \begin{itemize}
    \item $L_2 (L_1 (t a)) = L_2 (t L_1 (a)) = a L_2(L_1(a))$
    \item $L_2 (L_1 (a + b)) = L_2 (L_1 (a) + L_1(b)) = L_2(L_1(a)) + L_2(L_1(a))$
  \end{itemize}
\end{block}

\end{frame}


\begin{frame}{Транспонирование}

\begin{block}{У любого оператора $L$ есть брат $L^T$}
\begin{itemize}
  \item $d(La, b)=d(a, L^Tb)$
  \item $\angle(La, b) = \angle(a, L^Tb)$
\end{itemize}
\end{block}

\begin{block}{Транспонирование оператора $L$}
  $\langle La, b\rangle = \langle a, L^T b\rangle$
\end{block}

\end{frame}

\begin{frame}{Некоторые действия можно отменить!}

\begin{block}{Тождественный оператор $I$}
  Для любого вектора $v$: $I(v) = v$. 
\end{block}

\begin{block}{Обратный оператор $L^{-1}$}
  $L^{-1}L(a) = a$ 

Не у всех действий $L$ есть обратное $L^{-1}$!
\end{block}

\end{frame}
  

\begin{frame}{Обратимы ли поворот и проекция?}

\begin{block}{TODO: картинка обратимость поворота и необратимость проекции}
\end{block}

\end{frame}
  

\begin{frame}{Собственные векторы и собственные числа}

%\begin{block}{Бывает, что }
%  в целом сложное линейное действие всего лишь растягивает особые векторы.
%\end{block}
  
\begin{block}{Определение}
  Если для действия $L$ найдётся такой вектор $v$, что $Lv=\lambda \cdot v$, где $\lambda \in \R$, то:
  \begin{itemize}
    \item вектор $v$ называется собственным;
    \item число $\lambda$ называется собственным.
  \end{itemize} 
\end{block}
  
\end{frame}


\begin{frame}{Растягивание вдоль осей}

Рассмотрим $L : \begin{pmatrix}
  a_1 \\
  a_2 \\
\end{pmatrix} \to
\begin{pmatrix}
  2a_1 \\
  -3a_2 \\
\end{pmatrix}
$

\begin{block}{Собственные векторы с $\lambda = 2$}
  $v=\begin{pmatrix}
    x \\
    0 \\
  \end{pmatrix}$
\end{block}


\begin{block}{Собственные векторы с $\lambda = -3$}
  $v=\begin{pmatrix}
    0 \\
    x \\
  \end{pmatrix}$
\end{block}


\end{frame}


\begin{frame}{Обращение растягивания}

Рассмотрим $L : \begin{pmatrix}
  a_1 \\
  a_2 \\
\end{pmatrix} \to
\begin{pmatrix}
  2a_1 \\
  -3a_2 \\
\end{pmatrix}
$
  
\begin{block}{Обратное действие}
  $L^{-1} : \begin{pmatrix}
    a_1 \\
    a_2 \\
  \end{pmatrix} \to
  \begin{pmatrix}
    \frac{1}{2}a_1 \\
    \frac{1}{-3}a_2 \\
  \end{pmatrix}
  $
\end{block}

\begin{block}{$L^{-1}L = I$}  
\end{block}

\end{frame}



\begin{frame}{Транспонирование растягивания}

Рассмотрим $L : \begin{pmatrix}
  a_1 \\
  a_2 \\
\end{pmatrix} \to
\begin{pmatrix}
  2a_1 \\
  -3a_2 \\
\end{pmatrix}
$
  
\begin{block}{Транспонирование}
$\langle La, b\rangle = (2a_1) b_1 + (-3a_2)b_2 = a_1 (2b_1) + a_2(-3b_2) = \langle a, Lb\rangle$
\end{block}

\begin{block}{$L^T = L$}  
\end{block}

\end{frame}





\begin{frame} % видеофрагмент
  
  Проекция: собственные векторы и собственные числа, транспонирование

\end{frame}
  
\begin{frame}{видео с ДОСКОЙ}

\begin{block}{геометрический смысл собственных векторов}
\end{block}

\begin{block}{геометрический смысл транспонирования}
\end{block}

\begin{block}{отсутствие обратного действия}
\end{block}



\end{frame}

\begin{frame} % видеофрагмент
  Поворот: обращение, транспонирование, собственные числа и векторы
\end{frame}
  
\begin{frame}{видео с ДОСКОЙ}

\begin{block}{геометрический смысл собственных векторов}
\end{block}

\begin{block}{геометрический смысл транспонирования}
\end{block}

\begin{block}{обратный поворот на плоскости}
\end{block}



\end{frame}



\begin{frame} % видеофрагмент
  Линейная алгебра и игра Ним
\end{frame}


  
\begin{frame}{видео с ДОСКОЙ}

\begin{block}{Важная мысль}
числа могут быть не обязательно действительные, например, $\{0, 1\}$
сложение может быть необычным
\end{block}


\begin{block}{доказываем}
что позиция в Ним проигрышна, если и только если сумма векторов кучек равна нулю
\end{block}

\end{frame}

\begin{frame} % видеофрагмент
  Задача о переворачивании монетки на шахматной доске
\end{frame}



\begin{frame}{видео с ДОСКОЙ в шахматном смысле}

\begin{block}{Важная мысль}
что вектором может быть всё!
\end{block}
    
\begin{block}{вектор это}
\begin{itemize}
   \item Клетка на доске как вектор
   \item Чётность расстановки монеток на доске как вектор
\end{itemize}
\end{block}


\end{frame}





\end{document}


\begin{frame}{}

\begin{block}{}
\end{block}

\end{frame}


\begin{frame}{}

\begin{block}{}
\end{block}
  
\end{frame}
  



\begin{frame}{}

\begin{block}{}
\end{block}

\end{frame}


\begin{frame}{}

\begin{block}{}
\end{block}
  
\end{frame}



\begin{frame}{}

\begin{block}{}
\end{block}

\end{frame}


\begin{frame}{}

\begin{block}{}
\end{block}
  
\end{frame}




\begin{frame}{}

\begin{block}{}
\end{block}

\end{frame}


\begin{frame}{}

\begin{block}{}
\end{block}
  
\end{frame}
  



\end{document}