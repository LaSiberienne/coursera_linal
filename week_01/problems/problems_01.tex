%!TEX TS-program = xelatex
\documentclass[12pt]{article}
\usepackage[utf8]{inputenc}


%%%%%%%%%% Шрифты %%%%%%%%
\usepackage[english, russian]{babel} % выбор языка для документа
\usepackage[utf8]{inputenc} % задание utf8 кодировки исходного tex файла
\usepackage[X2,T2A]{fontenc}        % кодировка

\usepackage{fontspec}         % пакет для подгрузки шрифтов
\setmainfont{Times New Roman}       % задаёт основной шрифт документа

\usepackage{amsmath,amsfonts,amssymb,amsthm,mathtools}
\usepackage{hyphenat}


\usepackage{bbold}
\usepackage{verbatim}
\RequirePackage{lastpage}
\RequirePackage{epsfig}
\RequirePackage{graphicx}
\RequirePackage{float}
\RequirePackage{subfigure}
%\RequirePackage{overcite}
\RequirePackage{psfrag}
\RequirePackage{ifthen}
\RequirePackage{xcolor}
\RequirePackage{hyperref}
%\RequirePackage{cleveref}

\usepackage{indentfirst}

\textwidth=6.5in
\textheight=9in
\topmargin=-0.25in
\headheight=0in
\headsep=0in
\topskip=0in
\oddsidemargin=0in
\evensidemargin=0in

\newcommand{\PHNote}[1]{\ColorNote{red}{PH}{#1}}

\renewcommand{\maketitle}{%
	\begin{center}
		\noindent{\bf \Large  Линейная алгебра. Задачи}
		\\\vspace{1pt}
	\end{center}
}

 \usepackage{multirow}

\usepackage[backend=biber, style=numeric-comp,
	sorting=ynt,  
	defernumbers=true,
	 babel=other]{biblatex}

\addbibresource{references.bib}

\usepackage{unicode-math}      % пакет для установки математического шрифта

\newcommand{\R}{\mathbb{R}}
\newcommand{\E}{\mathbb{E}}
\newcommand{\var}{\mathbb{V}}
\newcommand{\heta}{\hat{\eta}}
\newcommand{\hgamma}{\hat{\gamma}}
\newcommand{\Bor}{\mathcal{B}}
\newcommand{\N}{\mathbb{N}}
\newcommand{\Xtn}{\tilde{X}^n}
\newcommand{\clawn}{\underset{n\rightarrow
		+\infty}{\Longrightarrow}} 
\newcommand{\convn}{\underset{n\rightarrow
		+\infty}{\longrightarrow}} 
\newcommand{\p}{\mathbb{P}}
\newcommand{\pperp}{\perp \! \! \! \perp}
\newcommand{\Var}{\text{Var}}
\newcommand{\lam}{\lambda}
\newcommand{\convps}{\underset{n\rightarrow +\infty}{\overset{\text{p.s.}}{\longrightarrow}}}
\newcommand{\convpsm}{\underset{m\rightarrow +\infty}{\overset{\text{p.s.}}{\longrightarrow}}}
\newcommand{\convproba}{\underset{n\rightarrow +\infty}{\overset{\text{proba.}}{\longrightarrow}}}
\newcommand{\convloi}{\underset{n\rightarrow +\infty}{\overset{\text{d}}{\longrightarrow}}}
\newcommand{\convlp}{\underset{n\rightarrow +\infty}{\overset{L^p}{\longrightarrow}}}
\newcommand{\convldeux}{\underset{n\rightarrow +\infty}{\overset{L^2}{\longrightarrow}}}
\newcommand{\Z}{\mathbb{Z}}
\newcommand{\simU}{\sim \mathcal{U}([0;1])}
\newcommand{\Yh}{\hat{Y}}
\newcommand{\uh}{\hat{u}}
\newcommand{\Yb}{\bar{Y}}
\newcommand{\Xb}{\bar{X}}
\newcommand{\uhb}{\bar{\hat{u}}}
\renewcommand{\b}{\beta}
\newcommand{\1}{\mathbb{1}}


\usepackage{enumitem}
\setenumerate[1]{label={(\alph*)}} 

\newcounter{rtaskno}
\DeclareRobustCommand{\rtask}[1]{%
	\refstepcounter{rtaskno}%
	\thertaskno\label{#1}}


\usepackage{comment}
%Uncomment below to hide/show solution
\includecomment{teacher}
%\excludecomment{teacher}

\begin{document}
\maketitle

\section{Неделя 1. Открытые вопросы}


\textbf{Задача \rtask{1}}  

%  Stanford p.57\(

\begin{enumerate}
	\item Введем обозначения \(\alpha=\|a\|, \beta=\|b\|\). 
	Выразите  \(\|\beta a-\alpha b\|^{2}\) 
	через  \( \|a\|, \|b\|\)   и  
	\(\langle a,b \rangle\).
	\item Докажите неравенство Коши — Буняковского,  \(\left|a^{T} b\right| \leq\|a\|\|b\|\), пользуясь свойствами нормы.
	\item Докажите неравенство треугольника,  \(\|a+b\| \leq\|a\|+\|b\|\),  пользуясь неравенством Коши — Буняковского.	
\end{enumerate}





\textbf{Задача \rtask{1}}  

Пусть \(a\) и \(b\) — векторы одинаковой размерности. 

При каких условиях верно, что

\begin{enumerate}
	\item    \((a+b)^{T}(a-b)=\|a\|^{2}-\|b\|^{2}\)?
	\item \(\|a+b\|^{2}+\|a-b\|^{2}=2\left(\|a\|^{2}+\|b\|^{2}\right)\)?
	\item   \(\|a+b\|=\sqrt{\|a\|^{2}+\|b\|^{2}}\)?
	
\end{enumerate}

\textit{ Ответ: }

\begin{enumerate}
	\item Всегда.
	\item  Всегда  (тождество параллелограмма). 
	\item  Верно при \(\langle a, b \rangle =  0 \).
\end{enumerate} 





\textbf{Задача \rtask{1}}  


Расстояние между булевыми векторами. Предположим, что \(x\) и 
\(y\) -- булевы  
\(n\)-мерные векторы 
(т.е. \(x_i, y_i \in \{0,1\} \)). 
Чему равно расстояние между векторами  \(\|x - y\|\)?


Ответ: \(\|x - y\| = \sqrt{\sum_{i=1}^{n} x_i\oplus y_i} = \sqrt{\sum_{i=1}^{n}   (x_i + y_i) \bmod 2 } \)



\textbf{Задача \rtask{1}}  

Пусть \(x\) и 
\(y\) -- два ненулевых 
\(n\)-мерных вектора, таких, 
что  \(x_i \geq 0, y_i\geq0, \forall i=1,\dots n\).  


\begin{enumerate}
	\item В каких пределах лежит \(\cos(x,y)\)? 
	\item Если \(n = 2\),  какие векторы являются ортогональными? 
\end{enumerate}

\textit{Ответ:}


\begin{enumerate}
	\item \(\cos(x,y) \in [0,1]\). 
	\item \(\begin{pmatrix} a \\ 	0  	\end{pmatrix}\) 
	и  \(\begin{pmatrix} 	0 \\ 	b   	\end{pmatrix}\), 
	где \(a,b\geq0\).
\end{enumerate}



\textbf{Задача \rtask{1}}  

Пусть \( с = \begin{pmatrix} 1 \\ 0  \end{pmatrix}\).

Приведите  пример такие два  вектора  \(a,b\), для которых: 

\(\|a-c\| < \|b-c\| \) 

\(\angle (a,c)  > \angle (b,c) \).


\textbf{Задача \rtask{1}}  

Пусть  \(f(x)=\|x-c\|^{2}-\|x-d\|^{2}\). 

Верно ли (или при каком условии верно), что  \(f\) -- линейный оператор? 

\textit{ Ответ:}

\(\langle x, x \rangle  - 2\langle x, c \rangle + \langle c, c \rangle - \langle x, x \rangle + 2\langle x, d \rangle - \langle d, d \rangle  \Rightarrow \)  верно, 
при \(\|c\| = \|d\|.\)   





\textbf{Задача \rtask{1}}  

Вася кодирует многочлен вида \(ax^2 + bx + c \)  с помощью вектора   
\(\begin{pmatrix}a \\b \\c\end{pmatrix}\).

Пусть оператор \(D_1\) берет первую производную,  
а \(D_2\)  -- вторую производную. 

Найдите \(D_1\begin{pmatrix}a \\b \\c\end{pmatrix}\) 
и  \(D_2\begin{pmatrix}a \\b \\c  \end{pmatrix}\).

Является ли \(D_1\)   линейным оператором? 
Является ли \(D_2\)   линейным оператором? 


\textbf{Задача \rtask{1}}

Пусть \(A: \begin{pmatrix}	v_{1} \\	v_{2} \\ 	\end{pmatrix} \to \begin{pmatrix}	5  v_{1}   \\	3  v_{2} 	\end{pmatrix}   \) -- операция растягивания, \(B: \begin{pmatrix}	v_{1} \\	v_{2}	\end{pmatrix} \to \begin{pmatrix}	v_{2}   \\	v_{1} 	\end{pmatrix}  \) -- операция перестановки координат. 


Покажите, что 	\(A+B\) -- линейный оператор.


\textbf{Задача \rtask{1}} 

% Stanford p.60

Пусть \(h: v \to \bar{v} \cdot u  \), 
где  \(u = \begin{pmatrix}1 \\\vdots \\ 1\end{pmatrix}\), 
\(\bar{v} = \dfrac{v_1 + \dots + v_n}{n} \)



\begin{enumerate}
	\item Найдите \(h\begin{pmatrix}	5 \\	6 \\ 	3	\end{pmatrix}\) .
	\item  Является ли \(h\) линейным оператором?  
	\item  Найдите \(h^2\).
\end{enumerate}

\textbf{Задача \rtask{1}}  


\textit{Результаты  анкетирования.} В анкете журнала ``Коммерсантъ''  30 вопросов, разбитых на
два блока из 15 вопросов.  
На каждый из вопросов читатель журнала может ответить   
``Редко'', ``Иногда'' или ``Часто''. Ответы записываются в виде 30-мерного вектора \(a\), 
где \(a_i\) принимает значение
\( 1, 2\)  или  
\(3\), 
если ответ на вопрос \(i\) был     ``Редко'', ``Иногда'' или ``Часто'', соответственно. 

При подсчете результатов опроса
\begin{itemize}
	\item за каждый ответ ``Иногда'' в вопросах 1-15 добавляется 1 балл, в вопросах 16-30 добавляется 2 балла; 
	\item  за каждый ответ ``Часто'' в вопросах 1-15 добавляется 2  балла, в вопросах 16-30 добавляется 4 балла;
	\item  за  ответ ``Редко'' баллы не добавляются.	   
\end{itemize}

\begin{enumerate}
	\item Найдите функцию \(f\), подсчитывающую сумму баллов.  
	\item Запишите её в виде \(s = \langle w, a  \rangle + c  \), где
	\(w\) --   30-мерный вектор, 
	\(c\) -- скаляр.  
	\item Линейна ли эта функция? 
\end{enumerate}



\textbf{Задача \rtask{1}}  



\textit{Энергия  сигнала по Дирихле.} 
Пусть временной ряд (или сигнал) задается  \(T\)-мерным 
вектором \(x\).

Тогда энергия сигнала по Дирихле
\[\mathcal{D}(x)=\left(x_{1}-x_{2}\right)^{2}+\cdots+\left(x_{T-1}-x_{T}\right)^{2},\]
рассчитывается как сумма квадратов разностей соседних значений сигнала. Энергия Дирихле -- это мера того, насколько сильно колеблется сигнал.

Введем следующие обозначения: \(x_{-1}\)  -- вектор
\(x\) без 1-ого наблюдения, 
\(x_{-T}\)  -- вектор
\(x\) без последнего наблюдения. 

\begin{enumerate}
	\item Представьте \(\mathcal{D}(x)\)  в векторной записи.
	\item Каково минимальное значение \(\mathcal{D}(x)\)?  Какие сигналы имеют минимальную энергию?
	\item Пусть вектор \(x'\):  
	\(x_i \in [0,1], \forall i=1,\dots,T\).  Какие сигналы, в таком случае,  будут иметь  максимальную энергию?  Каково максимальное значение 
	\(\mathcal{D}'(x)\)?  
\end{enumerate}

\textit{Ответ:}

\begin{enumerate}
	\item \(\mathcal{D}(x)=\| x_{-T} - x_{-1} \|^2\).
	\item \(\min \mathcal{D}(x) = 0\).
	\item \(\max \mathcal{D}'(x) = T-1\). 
\end{enumerate}










\end{document}

