% !TEX root = ../linal_lecture_01.tex



\begin{frame} % название фрагмента

\videotitle{Игра Ним}
\todo{это видео является бонусным, поэтому ничего нет страшного, что с ним выходит много видео, 
или оно будет долгое}

\todo{В начале фрагмента идёт слайд с правилами}
\todo{затем видеофрагмент с прозрачной доской}


\end{frame}


\begin{frame}{Игра Ним} % название фрагмента

\begin{itemize}
    \item Есть три кучки с камнями: 3, 5 и 8 камней;
    \item Два игрока ходят по очереди;
    \item За ход: \\
    \quad игрок выбирает одну кучку; \\
    \quad берёт из неё положительное число камней;
    \item Выигрывает берущий последний камень.
\end{itemize}

Какой ход сделать первому игроку?

\end{frame}
    
        
\begin{frame}{видео с доской}

\todo{Краткое содержание:}

\todo{закодируем каждую кучку двоичным вектором}

\todo{стоимость позиции — сумма этих двух векторов}

\todo{финальная позиция имеет стоимость ноль}

\todo{любой ход из нулевой позиции ведёт в положительную}

\todo{Из положительной позиции можно попасть в нулевую}

\todo{С помощью нижней единички убиваем остальные}

\todo{Выигрышный ход: взять 2 камня из кучки в 8 камней}


\end{frame}
