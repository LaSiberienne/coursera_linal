% !TEX root = ../linal_lecture_01.tex


\begin{frame} % название фрагмента

\videotitle{Линейный оператор: определение и примеры}

% примеры: R^n -> R^n: перестановка координат, растягивание, проекция, поворот, единичный оператор
% R^n -> R^k: обрезка, дописывание нулей,

\end{frame}


\begin{frame}{Краткий план:}

\begin{itemize}[<+->]
  \item Определение линейного оператора.
  \item Примеры линейных операторов.
  \item Как из двух операторов сделать новый оператор?
  % Ядерные функции из скалярного произведения. - в упражнения!
\end{itemize}

\end{frame}
    


\begin{frame}{Линейный оператор}


\alert{Идея линейности}:
  
Результат действия не изменится, 

если поменять местами действие $\LL$ и

\begin{itemize}[<+->]
    \item растягивание вектора, например, $\LL(42a)=42\LL(a)$;
    \item усреднение двух векторов, $\LL(0.5a+0.5b)=0.5\LL(a) + 0.5\LL(b)$.
  \end{itemize}


\end{frame}


\begin{frame}{Стандартное определение линейности}

\begin{block}{Определение}

\alert{Линейный оператор} $\LL$ из $\R^n$ в $\R^k$.

\begin{itemize}[<+->]
  \item Для любого числа $t$ и вектора $a \in \R^n$: $\LL(t a) = t\LL(a)$.
  \item Для любых двух векторов $a$ и $b$ из $\R^n$: $L(a + b) = \LL(a) + \LL(b)$. 
\end{itemize}
\end{block}

\pause
\vspace{10pt}
Вместо скобок часто пишут знак умножения, 

$\LL(a) \equiv \LL \cdot a \equiv \LL a$.



\end{frame}


\begin{frame}{Линейный оператор}


\begin{tikzpicture}[
scale=2,
MyPoints/.style={draw=blue,fill=white,thick},
Segments/.style={draw=blue!50!red!70,thick},
MyCircles/.style={green!50!blue!50,thin}, 
every node/.style={scale=1}
]
\draw[color=gray,step=1.0,dotted] (-1.9,-0.9) grid (7.5,7.0); 
% \clip (-1.5,-0.5) rectangle (6.1,6.5);

%{\verb!->!new, arrowhead = 2mm, line width=4pt}
%, arrowhead = 3mm
%, arrowhead = 0.2

% Feel free to change here coordinates of points A and B
\pgfmathparse{0}		\let\Xa\pgfmathresult
\pgfmathparse{4}		\let\Ya\pgfmathresult
\coordinate (A) at (\Xa,\Ya);

\pgfmathparse{0}		\let\Xb\pgfmathresult
\pgfmathparse{5}		\let\Yb\pgfmathresult
\coordinate (B) at (\Xb,\Yb);

\pgfmathparse{2}		\let\Xc\pgfmathresult
\pgfmathparse{5}		\let\Yc\pgfmathresult
\coordinate (C) at (\Xc,\Yc);

\pgfmathparse{2}		\let\Xd\pgfmathresult
\pgfmathparse{4}		\let\Yd\pgfmathresult
\coordinate (D) at (\Xd,\Yd);

\pgfmathparse{4}		\let\Xe\pgfmathresult
\pgfmathparse{4}		\let\Ye\pgfmathresult
\coordinate (E) at (\Xe,\Ye);

\pgfmathparse{0}		\let\Xf\pgfmathresult
\pgfmathparse{1}		\let\Yf\pgfmathresult
\coordinate (F) at (\Xf,\Yf);

\pgfmathparse{4}		\let\Xg\pgfmathresult
\pgfmathparse{1}		\let\Yg\pgfmathresult
\coordinate (G) at (\Xg,\Yg);


% Let I be the midpoint of [AB]
\pgfmathparse{(\Xb+\Xa)/2} \let\XI\pgfmathresult
\pgfmathparse{(\Yb+\Ya)/2} \let\YI\pgfmathresult
\coordinate (I) at (\XI,\YI);	


\draw[-{Latex[length=4.5mm, width=2.5mm]}, >=stealth, veca,thick] (A)--(B) node[left]{$\left(\begin{array}{l}0 \\ 1\end{array}\right)$} ;

\draw[black, dashed, thick] (B)--(C);
\draw[black, dashed, thick] (D)--(C);
\draw[black, dashed, thick] (F)--(G);
\draw[black, dashed, thick] (E)--(G);


\draw[-{Latex[length=4.5mm, width=2.5mm]}, >=stealth,  veca] (A)--(C) node[above right]{$\ba = \begin{pmatrix} 2 \\ 1 \end{pmatrix}$};

\draw[-{Latex[length=4.5mm, width=2.5mm]}, >=stealth,  vecb] (A)--(E) node[below]{$\LL  \cdot \left(\begin{array}{l}2 \\ 0\end{array}\right)$};

\draw[-{Latex[length=4.5mm, width=2.5mm]}, >=stealth,  veca] (A)--(D) node[below]{$\left(\begin{array}{l}2 \\ 0\end{array}\right)$};

\draw[-{Latex[length=4.5mm, width=2.5mm]}, >=stealth,  vecb] (A)--(G) node[below left,pos=1.1]{$\LL  \cdot \ba $};

\draw[-{Latex[length=4.5mm, width=2.5mm]}, >=stealth,  vecb] (A)--(F) node[below]{$\LL  \cdot \left(\begin{array}{l}0 \\ 1\end{array}\right)$ };


\end{tikzpicture}




\end{frame}


\begin{frame}{Растягивание координат}

Обобщаем умножение вектора на число!

  $L : \begin{pmatrix}
    a_1 \\
    a_2 \\
  \end{pmatrix} \to 
  \begin{pmatrix}
    2a_1 \\
    -3a_2 \\
  \end{pmatrix}$


\end{frame}



\begin{frame}{Перестановка координат вектора}


$L : \begin{pmatrix}
    a_1 \\
    a_2 \\
    a_3 \\
  \end{pmatrix} \to 
  \begin{pmatrix}
    a_2 \\
    a_3 \\
    a_1 \\
  \end{pmatrix}$

\vspace{10pt}

Пример. \alert{Отражение относительно $x_1=x_2$}:



\begin{tikzpicture}[
scale=1.8,
MyPoints/.style={draw=black,fill=black,thick},
Segments/.style={draw=blue!50!red!70,thick},
MyCircles/.style={green!50!blue!50,thin}, 
every node/.style={scale=1.2}
]
\draw[color=gray,step=1.0,dotted] (-1.5,-1.5) grid (3.5,3.5); 
% \clip (-1.5,-3.5) rectangle (3.5,3.5);

\pgfmathparse{0}		\let\Xa\pgfmathresult
\pgfmathparse{0}		\let\Ya\pgfmathresult
\coordinate (A) at (\Xa,\Ya);

\pgfmathparse{2}		\let\Xb\pgfmathresult
\pgfmathparse{1}		\let\Yb\pgfmathresult
\coordinate (B) at (\Xb,\Yb);

\pgfmathparse{3}		\let\Xc\pgfmathresult
\pgfmathparse{3}		\let\Yc\pgfmathresult
\coordinate (C) at (\Xc,\Yc);

\pgfmathparse{-1}		\let\Xd\pgfmathresult
\pgfmathparse{-1}		\let\Yd\pgfmathresult
\coordinate (D) at (\Xd,\Yd);

\pgfmathparse{1}		\let\Xe\pgfmathresult
\pgfmathparse{2}		\let\Ye\pgfmathresult
\coordinate (E) at (\Xe,\Ye);



\draw[ black,dashed] (D)--(C);

\draw[-{Latex[length=4.5mm, width=2.5mm]}, >=stealth, veca,thick] (A)--(B) node[below right]{$\left(\begin{array}{l}2 \\ 1\end{array}\right)$};

\draw[-{Latex[length=4.5mm, width=2.5mm]}, >=stealth, vecb,thick] (A)--(E) node[above left]{$\LL  \cdot \left(\begin{array}{l}2 \\ 1\end{array}\right) = \begin{pmatrix} 1 \\ 2 \end{pmatrix}$ };

% \node [right,darkgray] at (-1,-2.5) {$L:\left(\begin{array}{l}a_{1} \\ a_{2}\end{array}\right) \rightarrow\left(\begin{array}{l}a_{2} \\ a_{1}\end{array}\right)$ }; 


\fill[MyPoints]  (0,0) circle (0.8mm) node [below] {$0$};


\end{tikzpicture}


\end{frame}



\begin{frame}{Первый поворот}

\alert{Поворот} на $30^{\circ}$ против часовой стрелки

Оператор $R: \R^2 \to \R^2$ 

\begin{tikzpicture}[
scale=1.4,
MyPoints/.style={draw=blue,fill=white,thick},
Segments/.style={draw=blue!50!red!70,thick},
MyCircles/.style={blue!50,dashed}, 
every node/.style={scale=1.2}
]
%\grid;
%\draw[color=gray,step=1.0,dotted] (-7.1,-2.1) grid (7.6,6.1);
%\clip (-5.5,-.5) rectangle (5.5,5.5);


%%\draw[->, >=stealth] (-1,0)--(6.5,0) node[right]{$x_1$};
%\draw[-{Latex[length=4.5mm, width=2.5mm]}, >=stealth] (0,-1)--(0,5) node[above left]{$x_2$};
%
%\draw[-{Latex[length=4.5mm, width=2.5mm]}, >=stealth] (-1,0)--(6.5,0) 
%node[right]{$x_1$};

% Feel free to change here coordinates of points A and B
\pgfmathparse{0}		\let\Xa\pgfmathresult
\pgfmathparse{0}		\let\Ya\pgfmathresult
\coordinate (A) at (\Xa,\Ya);

\pgfmathparse{1}		\let\Xb\pgfmathresult
\pgfmathparse{3}		\let\Yb\pgfmathresult
\coordinate (B) at (\Xb,\Yb);

\pgfmathparse{3}		\let\Xc\pgfmathresult
\pgfmathparse{1}		\let\Yc\pgfmathresult
\coordinate (C) at (\Xc,\Yc);

\pgfmathparse{3}		\let\Xd\pgfmathresult
\pgfmathparse{4}		\let\Yd\pgfmathresult
\coordinate (D) at (\Xd,\Yd);

\pgfmathparse{75}		\let\angle\pgfmathresult;
\pgfmathparse{sqrt(10)}		\let\rad\pgfmathresult;


\pgfmathparse{\Xb*cos(\angle)  - \Yb*sin(\angle)}		\let\Xe\pgfmathresult
\pgfmathparse{\Xb*sin(\angle)  + \Yb*cos(\angle)}		\let\Ye\pgfmathresult
\coordinate (E) at (\Xe,\Ye);

\pgfmathparse{\Xd*cos(\angle)  - \Yd*sin(\angle)}		\let\Xf\pgfmathresult
\pgfmathparse{\Xd*sin(\angle)  + \Yd*cos(\angle)}		\let\Yf\pgfmathresult
\coordinate (F) at (\Xf,\Yf);


% Let I be the midpoint of [AB]
\pgfmathparse{(\Xb+\Xa)/2} \let\XI\pgfmathresult
\pgfmathparse{(\Yb+\Ya)/2} \let\YI\pgfmathresult
\coordinate (I) at (\XI,\YI);	


\draw[-{Latex[length=4.5mm, width=1.5mm]}, >=stealth, veca,thick] (A)--(B) node[midway,left]{$\ba$};

\draw[-{Latex[length=4.5mm, width=1.5mm]}, >=stealth, veca,thick] (B)--(D) node[midway,above]{$\bb$};

\draw[-{Latex[length=4.5mm, width=1.5mm]}, >=stealth, veca,thick] (A)--(D) node[midway,below right]{$\ba+\bb$};

\draw[-{Latex[length=4.5mm, width=1.5mm]}, >=stealth, vecb,thick] (A)--(E) node[midway,below left]{$\operatorname{R} \ba$};

\draw[-{Latex[length=4.5mm, width=1.5mm]}, >=stealth, vecb,thick] (E)--(F) node[midway,below left]{$\operatorname{R} \bb$};

\draw[-{Latex[length=4.5mm, width=1.5mm]}, >=stealth, vecb,thick] (A)--(F) node[midway,above right]{$\operatorname{R} (\ba+\bb)$};

\draw[MyCircles] (A) circle ({\rad});

\draw[MyCircles] (A) circle ({5});

\tkzMarkAngle[size=1, mark = none, arrows=->,line width=1.5pt, mkcolor=red ](B,A,E);


\end{tikzpicture}

    


\end{frame}
    


\begin{frame}{Первая проекция}

\alert{Проекция} на прямую $x_1 + 2x_2 = 0$

Оператор $H: \R^2 \to \R^2$ 



\begin{tikzpicture}[
scale=1.6,
MyPoints/.style={draw=blue,fill=blue,thick},
MyPoints2/.style={draw=red,fill=red,thick},
Segments/.style={draw=blue!50!red!70,thick},
MyCircles/.style={green!50!blue!50,thin}, 
every node/.style={scale=1.2}
]
%\grid;
% \clip (-1.5,-2.5) rectangle (6.5,5.5);


%%\draw[->, >=stealth] (-1,0)--(6.5,0) node[right]{$x_1$};
%\draw[-{Latex[length=4.5mm, width=2.5mm]}, >=stealth] (0,-1)--(0,5) node[above left]{$x_2$};
%
%\draw[-{Latex[length=4.5mm, width=2.5mm]}, >=stealth] (-1,0)--(6.5,0) 
%node[right]{$x_1$};

% Feel free to change here coordinates of points A and B
\pgfmathparse{0}		\let\Xa\pgfmathresult
\pgfmathparse{0}		\let\Ya\pgfmathresult
\coordinate (A) at (\Xa,\Ya);

\pgfmathparse{2}		\let\Xb\pgfmathresult
\pgfmathparse{3}		\let\Yb\pgfmathresult
\coordinate (B) at (\Xb,\Yb);

\pgfmathparse{3}		\let\Xc\pgfmathresult
\pgfmathparse{1}		\let\Yc\pgfmathresult
\coordinate (C) at (\Xc,\Yc);

\pgfmathparse{5}		\let\Xd\pgfmathresult
\pgfmathparse{4}		\let\Yd\pgfmathresult
\coordinate (D) at (\Xd,\Yd);

\pgfmathparse{2}		\let\Xe\pgfmathresult
\pgfmathparse{0}		\let\Ye\pgfmathresult
\coordinate (E) at (\Xe,\Ye);

\pgfmathparse{5}		\let\Xf\pgfmathresult
\pgfmathparse{0}		\let\Yf\pgfmathresult
\coordinate (F) at (\Xf,\Yf);

\pgfmathparse{5}		\let\Xg\pgfmathresult
\pgfmathparse{-1.5}		\let\Yg\pgfmathresult
\coordinate (G) at (\Xg,\Yg);

\pgfmathparse{0}		\let\Xh\pgfmathresult
\pgfmathparse{-1.5}		\let\Yh\pgfmathresult
\coordinate (H) at (\Xh,\Yh);

\pgfmathparse{-1.5}		\let\Xj\pgfmathresult
\pgfmathparse{0}		\let\Yj\pgfmathresult
\coordinate (J) at (\Xj,\Yj);

\pgfmathparse{6.5}		\let\Xk\pgfmathresult
\pgfmathparse{0}		\let\Yk\pgfmathresult
\coordinate (K) at (\Xk,\Yk);

\pgfmathparse{0}		\let\Xl\pgfmathresult
\pgfmathparse{-1}		\let\Yl\pgfmathresult
\coordinate (L) at (\Xl,\Yl);

\pgfmathparse{5}		\let\Xm\pgfmathresult
\pgfmathparse{-1}		\let\Ym\pgfmathresult
\coordinate (M) at (\Xm,\Ym);



% Let I be the midpoint of [AB]
\pgfmathparse{(\Xb+\Xa)/2} \let\XI\pgfmathresult
\pgfmathparse{(\Yb+\Ya)/2} \let\YI\pgfmathresult
\coordinate (I) at (\XI,\YI);	


\draw[-{Latex[length=4.5mm, width=2.5mm]}, >=stealth, veca,thick] (A)--(B) node[midway,left]{$\ba$};

\draw[-{Latex[length=4.5mm, width=1.5mm]}, >=stealth, veca,thick] (B)--(D) node[midway,above]{$\bb$};

\draw[-{Latex[length=4.5mm, width=2.5mm]}, >=stealth, vecc,thick] (A)--(D) node[below right]{$\ba+\bb$};

\draw[dashed] (B)--(E);
\draw[dashed] (D)--(F);
\draw[dashed] (J)--(K);

\draw[-{Latex[length=4.5mm, width=1.5mm]}, >=stealth, vecb,thick] (A)--(E) node[midway,below]{$\operatorname{H} \ba$};

\draw[-{Latex[length=4.5mm, width=1.5mm]}, >=stealth, vecb,thick] (E)--(F) node[midway,below]{$\operatorname{H} \bb$};

\draw[-{Latex[length=4.5mm, width=1.5mm]}, >=stealth, vecb,thick] (L)--(M) node[midway,below]{$\operatorname{H} (\ba+\bb)$};



\draw[thick] (F)--(G);
\draw[thick] (A)--(H);


\fill[MyPoints]  (L) circle (0.8mm);
\fill[MyPoints2]  (A) circle (0.8mm);


\end{tikzpicture}




\end{frame}
    




\begin{frame}{Обрезка компонент вектора}

\alert{Уменьшаем размерность}, $\LL : \begin{pmatrix}
  a_1 \\
  a_2 \\
  a_3 \\
\end{pmatrix} \to 
\begin{pmatrix}
  a_1 \\
  a_2 \\
\end{pmatrix}$
    


    \begin{minipage}{0.60\linewidth}
		
			
		\begin{tikzpicture}[
		scale=1.2,
		MyPoints/.style={draw=black,fill=black,thick},
		Segments/.style={draw=blue!50!red!70,thick},
		MyCircles/.style={green!50!blue!50,thin}, 
		every node/.style={scale=1}
		]

		%\clip (-4.5,-4.5) rectangle (4.5,6.5);

		\begin{scope}[cm={1,0.5,1,0,(0,0)}]
		\draw[draw=blue!30, dashed] (-2.2,-1.2) grid[step=1] (4,5);
		\end{scope}
			
		\draw[-{Latex[length=4.5mm, width=2.5mm]}, >=stealth] (0,-1)--(0,4) node[above left]{$x_3$};
	
		
		\draw[-{Latex[length=4.5mm, width=2.5mm]}, >=stealth] (-1,0)--(4,0) 
		node[below ]{$x_1$};
		
		\draw[-{Latex[length=4.5mm, width=2.5mm]}, >=stealth] (-2,-1)--(4,2) node[above left]{$x_2$};
		
		
		%{\verb!->!new, arrowhead = 2mm, line width=4pt}
		%, arrowhead = 3mm
		%, arrowhead = 0.2
		
		% Feel free to change here coordinates of points A and B
		\pgfmathparse{0}		\let\Xa\pgfmathresult
		\pgfmathparse{0}		\let\Ya\pgfmathresult
		\coordinate (A) at (\Xa,\Ya);
		
		\pgfmathparse{3}		\let\Xb\pgfmathresult
		\pgfmathparse{0.5}		\let\Yb\pgfmathresult
		\coordinate (B) at (\Xb,\Yb);

		\pgfmathparse{3}		\let\Xd\pgfmathresult
		\pgfmathparse{3}		\let\Yd\pgfmathresult
		\coordinate (D) at (\Xd,\Yd);

		
		\pgfmathparse{4}		\let\Xc\pgfmathresult
		\pgfmathparse{0}		\let\Yc\pgfmathresult
		\coordinate (C) at (\Xc,\Yc);
		
		
		\draw[-{Latex[length=4.5mm, width=2.5mm]}, >=stealth, veca, thick] (A)--(B) node[right]{$\widetilde{\ba}$};

		\draw[-{Latex[length=4.5mm, width=2.5mm]}, >=stealth, veca, thick] (A)--(D) node[above left]{$\ba$};

		
		\draw[black, dashed] (B)--(D);
				
		\fill[MyPoints]  (0,0) circle (0.8mm);
		
		\node [right,darkgray] at (-2.5,-3.0) {$\ba=\left(\begin{array}{c}a_{1} \\ a_{2} \\ a_{3}\end{array}\right) \quad \widetilde{\ba}=\left(\begin{array}{c}a_{1} \\ a_{2} \\ 0\end{array}\right)$}; 
		
		
		\end{tikzpicture}
		
		
    \end{minipage}\hfill
	\begin{minipage}{0.35\linewidth}
		
		
		\begin{tikzpicture}[
		scale=1.2,
		MyPoints/.style={draw=blue,fill=blue,thick},
		Segments/.style={draw=blue!50!red!70,thick},
		MyCircles/.style={green!50!blue!50,thin}, 
		every node/.style={scale=1}
		]
		%\clip (-4.5,-4.5) rectangle (4.5,6.5);
		
		\draw[color=blue!30,step=1.0,dashed] (-1.5,-1.5) grid (4.5,4.5);
		
		%\draw[->, >=stealth] (-1,0)--(6.5,0) node[right]{$x_1$};
		
		
		\draw[-{Latex[length=4.5mm, width=2.5mm]}, >=stealth] (0,-1)--(0,4) node[above left]{$x_2$};
		
		\draw[-{Latex[length=4.5mm, width=2.5mm]}, >=stealth] (-1,0)--(4,0) 
		node[below]{$x_1$};
		
		
		%{\verb!->!new, arrowhead = 2mm, line width=4pt}
		%, arrowhead = 3mm
		%, arrowhead = 0.2
		
			\pgfmathparse{0}		\let\Xa\pgfmathresult
		\pgfmathparse{0}		\let\Ya\pgfmathresult
		\coordinate (A) at (\Xa,\Ya);
		
		\pgfmathparse{2}		\let\Xb\pgfmathresult
		\pgfmathparse{1}		\let\Yb\pgfmathresult
		\coordinate (B) at (\Xb,\Yb);
		
		\pgfmathparse{3}		\let\Xd\pgfmathresult
		\pgfmathparse{3}		\let\Yd\pgfmathresult
		\coordinate (D) at (\Xd,\Yd);
		
		
		\pgfmathparse{4}		\let\Xc\pgfmathresult
		\pgfmathparse{0}		\let\Yc\pgfmathresult
		\coordinate (C) at (\Xc,\Yc);
		
		
		\draw[-{Latex[length=4.5mm, width=2.5mm]}, >=stealth, veca, thick] (A)--(B) node[above right]{$\LL  \cdot \ba$};
		
		
		%\draw[black] (A)--(C) node[midway,below]{$4$};
		
	
		\node [right,darkgray] at (0,-2.5) {$\LL  \ba =\left(\begin{array}{l}a_{1} \\ a_{2} \end{array}\right) $}; 
		
		
		
		\end{tikzpicture}
		
	\end{minipage}


\end{frame}




\begin{frame}{Дописывание нулей}

\alert{Увеличиваем размерность} пространства, $\LL : \begin{pmatrix}
    a_1 \\
    a_2 \\
  \end{pmatrix} \to 
  \begin{pmatrix}
    a_1 \\
    a_2 \\
    0  \\
  \end{pmatrix}$
    


\begin{minipage}{0.35\linewidth}


\begin{tikzpicture}[
scale=1,
MyPoints/.style={draw=blue,fill=blue,thick},
Segments/.style={draw=blue!50!red!70,thick},
MyCircles/.style={green!50!blue!50,thin}, 
every node/.style={scale=1}
]
%\clip (-1.5,-4.5) rectangle (4.5,6.5);

\draw[color=blue!30,step=1.0,dashed] (-1.5,-1.5) grid (4.5,4.5);

%\draw[->, >=stealth] (-1,0)--(6.5,0) node[right]{$x_1$};


\draw[-{Latex[length=4.5mm, width=2.5mm]}, >=stealth] (0,-1)--(0,4) node[above left]{$x_2$};

\draw[-{Latex[length=4.5mm, width=2.5mm]}, >=stealth] (-1,0)--(4,0) 
node[below]{$x_1$};


%{\verb!->!new, arrowhead = 2mm, line width=4pt}
%, arrowhead = 3mm
%, arrowhead = 0.2

\pgfmathparse{0}		\let\Xa\pgfmathresult
\pgfmathparse{0}		\let\Ya\pgfmathresult
\coordinate (A) at (\Xa,\Ya);

\pgfmathparse{2}		\let\Xb\pgfmathresult
\pgfmathparse{1}		\let\Yb\pgfmathresult
\coordinate (B) at (\Xb,\Yb);

\pgfmathparse{3}		\let\Xd\pgfmathresult
\pgfmathparse{3}		\let\Yd\pgfmathresult
\coordinate (D) at (\Xd,\Yd);


\pgfmathparse{4}		\let\Xc\pgfmathresult
\pgfmathparse{0}		\let\Yc\pgfmathresult
\coordinate (C) at (\Xc,\Yc);


\draw[-{Latex[length=4.5mm, width=2.5mm]}, >=stealth, veca, thick] (A)--(B) node[above right]{$ \ba$};


%\draw[black] (A)--(C) node[midway,below]{$4$};


\node [right,darkgray] at (0,-3.0) {$ \ba =\left(\begin{array}{l}a_{1} \\ a_{2} \end{array}\right) $}; 



\end{tikzpicture}
\end{minipage}
\begin{minipage}{0.60\linewidth}




\begin{tikzpicture}[
scale=1,
MyPoints/.style={draw=black,fill=black,thick},
Segments/.style={draw=blue!50!red!70,thick},
MyCircles/.style={green!50!blue!50,thin}, 
every node/.style={scale=1}
]

%\clip (-3.5,-4.5) rectangle (4.5,6.5);

\begin{scope}[cm={1,0.5,1,0,(0,0)}]
\draw[draw=blue!30, dashed] (-2.2,-1.2) grid[step=1] (4,5);
\end{scope}

\draw[-{Latex[length=4.5mm, width=2.5mm]}, >=stealth] (0,-1)--(0,4) node[above left]{$x_3$};


\draw[-{Latex[length=4.5mm, width=2.5mm]}, >=stealth] (-1,0)--(4,0) 
node[below ]{$x_1$};

\draw[-{Latex[length=4.5mm, width=2.5mm]}, >=stealth] (-2,-1)--(4,2) node[above left]{$x_2$};


%{\verb!->!new, arrowhead = 2mm, line width=4pt}
%, arrowhead = 3mm
%, arrowhead = 0.2

% Feel free to change here coordinates of points A and B
\pgfmathparse{0}		\let\Xa\pgfmathresult
\pgfmathparse{0}		\let\Ya\pgfmathresult
\coordinate (A) at (\Xa,\Ya);

\pgfmathparse{3}		\let\Xb\pgfmathresult
\pgfmathparse{0.5}		\let\Yb\pgfmathresult
\coordinate (B) at (\Xb,\Yb);

\pgfmathparse{3}		\let\Xd\pgfmathresult
\pgfmathparse{3}		\let\Yd\pgfmathresult
\coordinate (D) at (\Xd,\Yd);


\pgfmathparse{4}		\let\Xc\pgfmathresult
\pgfmathparse{0}		\let\Yc\pgfmathresult
\coordinate (C) at (\Xc,\Yc);


\draw[-{Latex[length=4.5mm, width=2.5mm]}, >=stealth, veca, thick] (A)--(B) node[right]{$\LL  \ba$};

%		\draw[-{Latex[length=4.5mm, width=2.5mm]}, >=stealth, veca, thick] (A)--(D) node[above left]{$\ba$};
    
%		\draw[black, dashed] (B)--(D);

\fill[MyPoints]  (0,0) circle (0.8mm);

\node [right,darkgray] at (0,-3.0) {$\LL  \ba=\left(\begin{array}{c}a_{1} \\ a_{2} \\ 0\end{array}\right)$}; 


\end{tikzpicture}

\end{minipage}


\end{frame}


\begin{frame}{Ничегонеделание}

	\begin{block}{Определение}
	\alert{Единичный} оператор, $I$, не меняет ни один вектор:

    \[
		I(\ba) = \ba.
	\]
	\end{block}


\end{frame}






\begin{frame}{Композиция линейных операторов}

\alert{Делай раз, делай два!}

\begin{itemize}[<+->]
    \item Если последовательно применить два линейных оператора, 
    то получится линейный оператор, $\LL_2 (\LL_1 (\ba)) = \LL(\ba)$.
\item Важно: $\LL_1 : \R^n \to \R^k$, $\LL_2 : \R^k \to \R^p$.
    \item \alert{Доказательство}
\begin{itemize}
  \item $\LL_2 (\LL_1 (t \ba)) = \LL_2 (t \LL_1 (\ba)) = t \LL_2(\LL_1(\ba))$
  \item $\LL_2 (\LL_1 (\ba + \bb)) = \LL_2 (\LL_1 (\ba) + \LL_1(\bb)) = \LL_2(\LL_1(\ba)) + \LL_2(\LL_1(\bb))$
\end{itemize}
\item Композицию также называют умножением.

\end{itemize}

\end{frame}


\begin{frame}{Сумма линейных операторов}




\begin{itemize}[<+->]
    \item Если сложить результаты двух линейных операторов, 
    то получится линейный линейный оператор, $\LL_1(\ba) + \LL_2(\ba) = \LL(\ba)$. 
\item Важно: $\LL_1 : \R^n \to \R^k$, $\LL_2 : \R^n \to \R^k$.
    \item \alert{Доказательство}
\begin{itemize}
  \item $\LL_1(t\ba) + \LL_2(t\ba) = t\LL_1(\ba) + t\LL_2(\ba)=t(\LL_1(\ba) + \LL_2(\ba))$
  \item $\LL(\ba + \bb) = \LL_1(\ba + \bb) + \LL_2(\ba + \bb) = \LL_1(\ba ) + \LL_2(\ba ) + \LL_1( \bb) + \LL_2(\bb) = \LL(\ba) + \LL(\bb)$
\end{itemize}

\end{itemize}

    





\end{frame}

    