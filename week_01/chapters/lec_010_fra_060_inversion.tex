% !TEX root = ../linal_lecture_01.tex


\begin{frame} % название фрагмента

\videotitle{Обращение оператора}

\end{frame}
  

\begin{frame}{Обращение}


Исходный оператор $\LL : \R^n \to \R^n$.


Определение. 

\alert{Обратным оператором} к $\LL$ называют такой оператор $\LL^{-1}$, 
что $\LL^{-1} \LL = I$. 


\end{frame}


\begin{frame}{Обращение растягивания}

\begin{itemize}[<+->]
    \item 
Исходный оператор $\LL : \begin{pmatrix}
  a_1 \\
  a_2 \\
\end{pmatrix} \to
\begin{pmatrix}
  2a_1 \\
  -3a_2 \\
\end{pmatrix}
$

\item \alert{Обратный оператор}:  
$\LL^{-1} : \begin{pmatrix}
    a_1 \\
    a_2 \\
  \end{pmatrix} \to
  \begin{pmatrix}
    \frac{1}{2}a_1 \\
    \frac{1}{-3}a_2 \\
  \end{pmatrix}
  $

\item $\LL^{-1} \LL = I$
\end{itemize}


\end{frame}
    

\begin{frame}{Обращение перестановки двух компонент}

\begin{itemize}[<+->]
    \item 
Исходный оператор $\LL : \begin{pmatrix}
  a_1 \\
  a_2 \\
\end{pmatrix} \to
\begin{pmatrix}
a_2 \\
a_1 \\
\end{pmatrix}
$

\item \alert{Обратный оператор}:  
$\LL^{-1} : \begin{pmatrix}
a_1 \\
a_2 \\
    \end{pmatrix} \to
  \begin{pmatrix}
a_2 \\
a_1 \\
    \end{pmatrix}
  $

\item $\LL^{-1} \LL = I$
\item $\LL^{-1} = \LL$
\end{itemize}


\end{frame}
    



\begin{frame}{Обращение перестановки компонент}

\begin{itemize}[<+->]
    \item 
Исходный оператор $\LL : \begin{pmatrix}
  a_1 \\
  a_2 \\
  a_3 \\
\end{pmatrix} \to
\begin{pmatrix}
a_3 \\
a_1 \\
a_2 \\
\end{pmatrix}
$

\item \alert{Обратный оператор}:  
$\LL^{-1} : \begin{pmatrix}
a_1 \\
a_2 \\
a_3 \\
    \end{pmatrix} \to
  \begin{pmatrix}
a_2 \\
a_3 \\
a_1 \\
    \end{pmatrix}
  $

\item $\LL^{-1} \LL = I$
\end{itemize}


\end{frame}


    
    


\begin{frame}{Обращение единичного оператора}

\begin{itemize}[<+->]
    \item 
Исходный оператор $I : \begin{pmatrix}
  a_1 \\
  a_2 \\
  a_3 \\
\end{pmatrix} \to
\begin{pmatrix}
a_1 \\
a_2 \\
a_3 \\
\end{pmatrix}
$

\item \alert{Обратный оператор}:  
$I : \begin{pmatrix}
a_1 \\
a_2 \\
a_3 \\
    \end{pmatrix} \to
  \begin{pmatrix}
a_1 \\
a_2 \\
a_3 \\
    \end{pmatrix}
  $

\item $I^{-1} I = I$
\end{itemize}


\end{frame}
    





\begin{frame}{Обращение поворота}

\begin{itemize}[<+->]
    \item 
Исходный оператор $R: \R^2 \to \R^2$, поворот на $30^{\circ}$ против часовой стрелки.

\item \alert{Обратный оператор}:  $R^{-1}$, поворот на $30^{\circ}$ по часовой стрелке.
\item $R^{-1} R = I$
\end{itemize}


\begin{tikzpicture}[
scale=1.4,
MyPoints/.style={draw=blue,fill=white,thick},
Segments/.style={draw=blue!50!red!70,thick},
MyCircles/.style={blue!50,dashed}, 
every node/.style={scale=1.2}
]
%\grid;
%\draw[color=gray,step=1.0,dotted] (-7.1,-2.1) grid (7.6,6.1);


%%\draw[->, >=stealth] (-1,0)--(6.5,0) node[right]{$x_1$};
%\draw[-{Latex[length=4.5mm, width=2.5mm]}, >=stealth] (0,-1)--(0,5) node[above left]{$x_2$};
%
%\draw[-{Latex[length=4.5mm, width=2.5mm]}, >=stealth] (-1,0)--(6.5,0) 
%node[right]{$x_1$};

% Feel free to change here coordinates of points A and B
\pgfmathparse{0}		\let\Xa\pgfmathresult
\pgfmathparse{0}		\let\Ya\pgfmathresult
\coordinate (A) at (\Xa,\Ya);

\pgfmathparse{1.5}		\let\Xb\pgfmathresult
\pgfmathparse{3}		\let\Yb\pgfmathresult
\coordinate (B) at (\Xb,\Yb);

\pgfmathparse{3}		\let\Xc\pgfmathresult
\pgfmathparse{1}		\let\Yc\pgfmathresult
\coordinate (C) at (\Xc,\Yc);

\pgfmathparse{3}		\let\Xd\pgfmathresult
\pgfmathparse{4}		\let\Yd\pgfmathresult
\coordinate (D) at (\Xd,\Yd);

\pgfmathparse{75}		\let\angle\pgfmathresult;
\pgfmathparse{sqrt(10)}		\let\rad\pgfmathresult;


\pgfmathparse{\Xb*cos(\angle)  - \Yb*sin(\angle)}		\let\Xe\pgfmathresult
\pgfmathparse{\Xb*sin(\angle)  + \Yb*cos(\angle)}		\let\Ye\pgfmathresult
\coordinate (E) at (\Xe,\Ye);

\pgfmathparse{\Xd*cos(\angle)  - \Yd*sin(\angle)}		\let\Xf\pgfmathresult
\pgfmathparse{\Xd*sin(\angle)  + \Yd*cos(\angle)}		\let\Yf\pgfmathresult
\coordinate (F) at (\Xf,\Yf);


% Let I be the midpoint of [AB]
\pgfmathparse{(\Xb+\Xa)/2} \let\XI\pgfmathresult
\pgfmathparse{(\Yb+\Ya)/2} \let\YI\pgfmathresult
\coordinate (I) at (\XI,\YI);	


\draw[-{Latex[length=4.5mm, width=1.5mm]}, >=stealth, veca,thick] (A)--(B) node[midway,right]{$\bv$};


\draw[-{Latex[length=4.5mm, width=1.5mm]}, >=stealth, vecb,thick] (A)--(E) node[midway,below left]{$\bb$};

\tkzMarkAngle[size=1, mark = none, arrows=->,line width=1.5pt, mkcolor=red ](B,A,E);

\draw[dashed] (B) to[bend right] (E);

\node [above right] at (-0.5, 1) {$\operatorname{R}$}; 

\node [above right] at (-2, 4) {$\operatorname{R} \bv = \bb$ или $\bv = \operatorname{R}^{-1} \bb$}; 


\end{tikzpicture}

\end{frame}




\begin{frame}{Не все действия обратимы!}
    
    

\begin{itemize}[<+->]
    \item По определению, исходный оператор $\LL : \R^n \to \R^n$.

    \item 
Исходный оператор $H: \R^2 \to \R^2$, проекция на прямую $x_1 + 2x_2 = 0$.

\item \alert{Обратный оператор}  $H^{-1}$ не существует!
\end{itemize}



\begin{tikzpicture}[
scale=1.6,
MyPoints/.style={draw=blue,fill=white,thick},
Segments/.style={draw=blue!50!red!70,thick},
MyCircles/.style={green!50!blue!50,thin}, 
every node/.style={scale=1.2}
]
%\grid;
%\clip (-1.5,-1.5) rectangle (5.5,5.5);

\pgfmathparse{2}		\let\Xa\pgfmathresult
\pgfmathparse{2}		\let\Ya\pgfmathresult
\coordinate (A) at (\Xa,\Ya);

\pgfmathparse{5}		\let\Xb\pgfmathresult
\pgfmathparse{5}		\let\Yb\pgfmathresult
\coordinate (B) at (\Xb,\Yb);

\pgfmathparse{-1}		\let\Xc\pgfmathresult
\pgfmathparse{5}		\let\Yc\pgfmathresult
\coordinate (C) at (\Xc,\Yc);

\pgfmathparse{4}		\let\Xd\pgfmathresult
\pgfmathparse{0}		\let\Yd\pgfmathresult
\coordinate (D) at (\Xd,\Yd);

\pgfmathparse{0}		\let\Xe\pgfmathresult
\pgfmathparse{4}		\let\Ye\pgfmathresult
\coordinate (E) at (\Xe,\Ye);

\pgfmathparse{3.5}		\let\Xf\pgfmathresult
\pgfmathparse{3.5}		\let\Yf\pgfmathresult
\coordinate (F) at (\Xf,\Yf);

\pgfmathparse{4}		\let\Xg\pgfmathresult
\pgfmathparse{4}		\let\Yg\pgfmathresult
\coordinate (G) at (\Xg,\Yg);

\pgfmathparse{4.5}		\let\Xh\pgfmathresult
\pgfmathparse{4.5}		\let\Yh\pgfmathresult
\coordinate (H) at (\Xh,\Yh);


\draw[ black,dashed] (D)--(C);

\draw[dashed] (A)--(B);

\draw[-{Latex[length=4.5mm, width=2.5mm]}, >=stealth, vecb,thick] (E)--(A) node[midway,below left]{$\bb$};

\draw[-{Latex[length=4.5mm, width=2.5mm]}, >=stealth, veca,thick] (E)--(F) node[right]{$\bv ?$};

\draw[-{Latex[length=4.5mm, width=2.5mm]}, >=stealth, veca,thick] (E)--(G) node[right]{$\bv ?$};

\draw[-{Latex[length=4.5mm, width=2.5mm]}, >=stealth, veca,thick] (E)--(H) node[right]{$\bv ?$};

\node [above right] at (0, 5) {$\operatorname{H} \bv = \bb$}; 


\end{tikzpicture}






\end{frame}
