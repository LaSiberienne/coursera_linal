\documentclass[14 pt,xcolor=dvipsnames]{beamer}

\usepackage{epsdice}

\usepackage[absolute,overlay]{textpos}

\usepackage[orientation=portrait,size=custom,width=25.4,height=19.05]{beamerposter}

%25,4 см 19,05 см размеры слайда в powerpoint

\usetheme{metropolis}
\metroset{
  %progressbar=none,
  numbering=none,
  subsectionpage=progressbar,
  block=fill
}

%\usecolortheme{seahorse}

\usepackage{fontspec}
\usepackage{polyglossia}
\setmainlanguage{russian}


\usepackage{fontawesome5} % removed [fixed]
\setmainfont[Ligatures=TeX]{Myriad Pro}
\setsansfont{Myriad Pro}


\usepackage{amssymb,amsmath,amsxtra,amsthm}


\usepackage{unicode-math}
\usepackage{centernot}

\usepackage{graphicx}
\graphicspath{{img/}}

\usepackage{wrapfig}
\usepackage{animate}
\usepackage{tikz}
%\usetikzlibrary{shapes.geometric,patterns,positioning,matrix,calc,arrows,shapes,fit,decorations,decorations.pathmorphing}
\usepackage{pifont}
\usepackage{comment}
\usepackage[font=small,labelfont=bf]{caption}
\captionsetup[figure]{labelformat=empty}
\includecomment{techno}

\usefonttheme[onlymath]{serif}


%Расположение

\setbeamersize{text margin left=15 mm,text margin right=5mm} 
\setlength{\leftmargini}{38 pt}

%\usepackage{showframe}
%\usepackage{enumitem}
%\setlist{leftmargin=5.5mm}


%Цвета от дирекции

\definecolor{dirblack}{RGB}{58, 58, 58}
\definecolor{dirwhite}{RGB}{245, 245, 245}
\definecolor{dirred}{RGB}{149, 55, 53}
\definecolor{dirblue}{RGB}{0, 90, 171}
\definecolor{dirorange}{RGB}{235, 143, 76}
\definecolor{dirlightblue}{RGB}{75, 172, 198}
\definecolor{dirgreen}{RGB}{155, 187, 89}
\definecolor{dircomment}{RGB}{128, 100, 162}

\setbeamercolor{title separator}{bg=dirlightblue!50, fg=dirblue}

%Цвета блоков

\setbeamercolor{block title}{bg=dirblue!30,fg=dirblack}

\setbeamercolor{block title example}{bg=dirlightblue!50,fg=dirblack}

\setbeamercolor{block body example}{bg=dirlightblue!20,fg=dirblack}

\AtBeginEnvironment{exampleblock}{\setbeamercolor{itemize item}{fg=dirblack}}
%\setbeamertemplate{blocks}[rounded][shadow]

% Набор команд для удобства верстки

\newcommand{\RR}{\mathbb{R}}
\newcommand{\ZZ}{\mathbb{Z}}
\newcommand{\la}{\lambda}

% Набор команд для структуризации

%\newcommand{\quest}{\faQuestionCircleO}
%\faPencilSquareO \faPuzzlePiece \faQuestionCircleO  \faIcon*[regular]{file} {\textcolor{dirblue}
%\newcommand{\quest}{\textcolor{dirblue}{\boxed{\textbf{?}}}
\newcommand{\task}{\faIcon{tasks}}
\newcommand{\exmpl}{\faPuzzlePiece}
\newcommand{\dfn}{\faIcon{pen-square}}
\newcommand{\quest}{\textcolor{dirblue}{\faQuestionCircle[regular]}}
\newcommand{\acc}[1]{\textcolor{dirred}{#1}}
\newcommand{\accm}[1]{\textcolor{dirred}{#1}}
\newcommand{\acct}[1]{\textcolor{dirblue}{#1}}
\newcommand{\acctm}[1]{\textcolor{dirblue}{#1}}
\newcommand{\accex}[1]{\textcolor{dirblack}{\bf #1}}
\newcommand{\accexm}[1]{\textcolor{dirblack}{ \mathbf{#1}}}
\newcommand{\acclp}[1]{\textcolor{dirorange}{\it #1}}


\newcommand{\videotitle}[1]{\begin{center}
    \textcolor{dirblue}{#1}

    \todo{название видеофрагмента}
\end{center}}

\newcommand{\lecturetitle}[1]{\begin{center}
    \textcolor{dirblue}{#1}

    \todo{название лекции}
\end{center}}




\newcommand{\todo}[1]{\textcolor{dircomment}{\bf #1}}

\newcommand{\spcbig}{\vspace{-10 pt}}
\newcommand{\spcsmall}{\vspace{-5 pt}}

%\usepackage{listings}
%\lstset{
%xleftmargin=0 pt,
%  basicstyle=\small, 
%  language=Python,
  %tabsize = 2,
%  backgroundcolor=\color{mc!20!white}
%}



%\newcommand{\mypart}[1]{\begin{frame}[standout]{\huge #1}\end{frame}}

\setbeamercolor{background canvas}{bg=}

% frame title setup
\setbeamercolor{frametitle}{bg=,fg=dirblue}
\setbeamertemplate{frametitle}[default][left]

\addtobeamertemplate{frametitle}{\hspace*{-0.5 cm}}{\vspace*{0.25cm}}


%Шрифты
\setbeamerfont{frametitle}{family=\rmfamily,series=\bfseries,size={\fontsize{33}{30}}}
\setbeamerfont{framesubtitle}{family=\rmfamily,series=\bfseries,size={\fontsize{26}{20}}}





\usepackage{physics}
\newcommand{\R}{\mathbb{R}}
\newcommand{\CC}{\mathbb{C}}

\newcommand{\Rot}{\mathrm{R}}
\newcommand{\HH}{\mathrm{H}}
\newcommand{\Id}{\mathrm{I}}

% 36 
% 37 черточки +
% 38  + без скачка


\usepackage[outline]{contour}


\usepackage{pgfplots}
\pgfplotsset{compat=newest}

\usepackage{tikz}
\usetikzlibrary{calc}
\usetikzlibrary{quotes,angles}
\usetikzlibrary{arrows}
\usetikzlibrary{arrows.meta}
\usetikzlibrary{positioning,intersections,decorations.markings}
\usetikzlibrary{patterns}

\usepackage{tkz-euclide} 

\newcommand{\grid}{\draw[color=gray,step=1.0,dotted] (-2.1,-2.1) grid (9.6,6.1)}

\newcommand{\ba}{\symbf{a}}
\newcommand{\be}{\symbf{e}}
\newcommand{\bb}{\symbf{b}}
\newcommand{\bc}{\symbf{c}}
\newcommand{\bd}{\symbf{d}}
\newcommand{\bx}{\symbf{x}}
\newcommand{\by}{\symbf{y}}
\newcommand{\bg}{\symbf{g}}
\newcommand{\bq}{\symbf{q}}
\newcommand{\bp}{\symbf{p}}
\newcommand{\bhy}{\symbf{\hat{y}}}
\newcommand{\bff}{\symbf{f}} % \bf is already def
\newcommand{\bv}{\symbf{v}}
\newcommand{\bu}{\symbf{u}}
\newcommand{\bzero}{\symbf{0}}
\newcommand{\red}[1]{\textcolor{red}{#1}}
\newcommand{\green}[1]{\textcolor{green}{#1}}
\newcommand{\blue}[1]{\textcolor{blue}{#1}}


\DeclareMathOperator{\eig}{Eig}

\DeclareMathOperator{\Lin}{Span}
\DeclareMathOperator{\col}{col}
\DeclareMathOperator{\row}{row}

\DeclareMathOperator{\adj}{adj}

\DeclareMathOperator{\sign}{sign}

\DeclareMathOperator{\charp}{char}

\DeclareMathOperator{\Span}{Span}
\DeclareMathOperator{\Image}{Image}


\DeclareMathOperator{\LL}{L}

%\tikzset{>=latex}

\colorlet{veca}{red}
\colorlet{vecb}{blue}
\colorlet{vecc}{olive}


\tikzset{cross/.style={cross out, draw=black, minimum size=2*(#1-\pgflinewidth), inner sep=0pt, outer sep=0pt},
%default radius will be 1pt. 
cross/.default={5pt}}





\begin{document}

% \maketitle


\begin{frame} % название лекции


\lecturetitle{Сингулярное разложение и главные компоненты}

\end{frame}


% !TEX root = ../linal_lecture_06.tex

\begin{frame} % название фрагмента

\videotitle{Сингулярное разложение}

\end{frame}



\begin{frame}{Краткий план:}
  \begin{itemize}[<+->]
    \item Жорданова нормальная форма.
    \item Сингулярное разложение.
    \item Доказательство существования.
  \end{itemize}

\end{frame}

 

\begin{frame}
\lecturetitle{Поиск SVD разложения}
\todo{Это видеофрагмент с доской, слайдов здесь нет :)}
\end{frame}
    
\begin{frame}
\lecturetitle{Нахождение проекции при известном SVD}
\todo{Это видеофрагмент с доской, слайдов здесь нет :)}
\end{frame}

% !TEX root = ../linal_lecture_06.tex

\begin{frame} % название фрагмента

\videotitle{Немного статистики}

\end{frame}



\begin{frame}{Краткий план:}
  \begin{itemize}[<+->]
    \item Выборочная дисперсия.
    \item Стандартизация.
    \item Выборочная корреляция.
  \end{itemize}

\end{frame}

\begin{frame}
  \frametitle{Центрирование}


  \begin{block}{Определение}
    Для вектора чисел $\bx = (x_1, x_2, \ldots, x_n)$ \alert{выборочным средним} называют величину
    \[
    \bar x = \frac{x_1 + x_2 + \ldots + x_n}{n}. \pause
    \]
  \end{block}

  \begin{block}{Определение}
    \alert{Центрирование переменной} — переход от набора чисел
    $(x_1, x_2, \ldots, x_n)$ к набору чисел $(x_1 - \bar x, x_2 - \bar x, \ldots, x_n - \bar x)$. \pause
  \end{block}

  Пример. $(4, 2, 3, 3) \to (1, -1, 0, 0)$.

\end{frame}


\begin{frame}
  \frametitle{Свойства центрирования}

  \begin{block}{Утверждение}
Если $\bx'$ — это центрированный вектор $\bx$, $x'_i = x_i - \bar x$, то 
$\bar x' = 0$. \pause
\[
\sum_{i=1}^n x'_i = \sum_{i=1}^n (x_i - \bar x) = \sum_{i=1}^n x_i - n\bar x =0  \pause
\]
  \end{block}

  С геометрической точки зрения:
  
  $(\bar x, \bar x, \ldots, \bar x)$ — проекция вектора $\bx$ на $\Span \bv$, где
  $\bv = (1, 1, \ldots, 1)$. \pause


$(x_1 - \bar x, x_2 - \bar x, \ldots, x_n - \bar x)$ — проекция вектора $\bx$ на $\Span^{\perp} \bv$.


\end{frame}



\begin{frame}
  \frametitle{Выборочная дисперсия}

  \begin{block}{Определение}
    \alert{Выборочной дисперсией} набора чисел $(x_1, x_2, \ldots, x_n)$ 
    называют величину $\norm{\bx'}^2 / (n-1)$, где вектор $\bx'$ — это 
    центрированный вектор $\bx$, 
    \[
    \frac{\sum_{i=1}^n(x_i - \bar x)^2}{n-1}.  
    \]
    \pause
  \end{block}

  Выборочная дисперсия вектора измеряет «разброс» $x_i$, 
  насколько далеки $x_i$ от своего среднего $\bar x$.
  

\end{frame}


\begin{frame}
  \frametitle{Стандартное отклонение}

  Если $x_i$ измеряется в сантиментрах, 
  то выборочная дисперсия измеряется в квадратных сантиметрах. \pause

  \begin{block}{Определение}
    \alert{Выборочным стандартным отклонением} набора $\bx$ называется 
    корень из выборочной дисперсии, 
    \[
      sd(\bx) = \sqrt{\frac{\sum_{i=1}^n(x_i - \bar x)^2}{n-1}}. \pause
    \]
  \end{block}

  \pause 
  Если $x_i$ измеряется в сантиментрах, 
  то выборочное стандартное отклонение измеряется в  сантиметрах. 
  

\end{frame}

\begin{frame}
  \frametitle{Стандартизация}

  Популярным вариантом масштабирования переменной является переход к безразмерной 
  величине по правилу:
  \[
  x_i  \to x_i' =  \frac{x_i - \bar x}{sd(\bx)}.  \pause
  \]

  После стандартизации величина имеет нулевое среднее $\bar x' =0$ и 
  единичное стандартное отлонение $sd(\bx') = 1$.


\end{frame}


\begin{frame}
  \frametitle{Выборочная корреляция}

  \begin{block}{Определение}
    \alert{Выборочной корреляцией} двух наборов чисел $\bx$ и $\by$
    называют косинус угла между их центрированными версиями:
    \[
    \rho(\bx, \by) = \cos \angle (\bx', \by') = \frac{\langle \bx', \by' \rangle}{\norm{\bx'} \norm{\by'}},   
    \]
    где $x'_i = x_i - \bar x$ и $y_i' = y_i - \bar y$. \pause
  \end{block}

Не определена, если $\bx$ или $\by$ состоит из одинаковых чисел. \pause
 
Лежит в диапазоне от $-1$ до $1$. 

\end{frame}


\begin{frame}
  \frametitle{Выборочная корреляция}

  \begin{block}{Утверждение}
    Выборочная корреляция равна 
    \[
      \rho(\bx, \by) = \frac{\sum (x_i - \bar x)(y_i - \bar y)}{\sqrt{\sum (x_i - \bar x)^2 \sum (y_i - \bar y)^2}}.
    \]
  \end{block}

 
  Если после центрирования векторы ортогональны, то выборочная корреляция равна $0$. \pause

  Не чувствительна к масштабированию.

\end{frame}


\begin{frame}
  \frametitle{Выборочная корреляционная матрица}

  \begin{block}{Определение}
  \alert{Выборочной корреляционной матрицей} переменных $\bx_1$, $\bx_2$, \ldots, $\bx_k$ называется
  матрица $C$, элементами которой являются выборочные корреляции,
  \[
  c_{ij} = \rho(\bx_i, \bx_j) \pause
  \] 
  \end{block}

  Диагональные элементы равны единице, $c_{ii}=1$. \pause

  Если $\bx_1$, $\bx_2$, \ldots, $\bx_k$ стандартизированы, то $C=X^TX/(n-1)$. \pause

  Матрица $C$ симметрична и положительно полуопределена. 

\end{frame}


 


% !TEX root = ../linal_lecture_06.tex

\begin{frame} % название фрагмента

\videotitle{PCA: максимизация разброса}

\end{frame}



\begin{frame}{Краткий план:}
  \begin{itemize}[<+->]
    \item Максимизация выборочной дисперсии.
  \item Свойства главных компонент.
    \end{itemize}

\end{frame}




\begin{frame}
  \frametitle{Метод главных компонент}


  Есть матрица $X$ исходных наблюдений:

  наблюдения отложены по строкам, 
  
  а переменные — по столбцам. \pause


  Переменных очень много. \pause


%  А мы хотим визуализировать данные. \pause

  Хотим иметь небольшое количество переменных, 
  которые бы почти без потерь 
  содержали всю информацию об исходных переменных. 


\end{frame}


\begin{frame}
  \frametitle{Метод главных компонент}

  Все исходные переменные предварительно стандартизируем! \pause

  Для каждого столбца $\bx$ выполнены условия $\bar x =0$, $sd(\bx)=1$. \pause


  На базе столбцов $\bx_1$, $\bx_2$, \ldots, $\bx_k$ матрицы $X$ мы создадим 
  $d\leq k$ новых переменных $\bp_1$, $\bp_2$, \ldots, $\bp_d$. \pause

  Новые переменные будем создавать по-очереди. \pause

  Новые переменные будем называть \alert{главными компонентами}.

  PCA — principal component analysis.

\end{frame}


\begin{frame}
  \frametitle{Максимизация разброса}

  Главные компоненты $\bp_1$, $\bp_2$, \ldots, $\bp_d$ будут линейными 
  комбинациями столбцов $X$. \pause

  \begin{block}{Алгоритм}
    \begin{enumerate}
      \item Компоненту $\bp_1 = X \bv_1$ подберём так, чтобы 
      выборочная дисперсия $\bp_1$ была максимальной при условии, что $\norm{\bv_1}=1$. \pause
    \item Компоненту $\bp_2 = X \bv_2$ подберём так, чтобы 
    выборочная дисперсия $\bp_2$ была максимальной при условии, что $\bv_2 \perp \bv_1$ и $\norm{\bv_2}=1$. \pause
  \item Компоненту $\bp_3 = X \bv_3$ подберём так, чтобы 
  выборочная дисперсия $\bp_3$ была максимальной при условии, что $\bv_3 \perp \bv_2$, $\bv_3 \perp \bv_1$ и $\norm{\bv_3}=1$. 
      \item \ldots
    \end{enumerate}
    
  \end{block}
  

\end{frame}


\begin{frame}
  \frametitle{Картинка}


\begin{tikzpicture}[
  scale=1.4,
  MyPoints/.style={draw=blue,fill=white,thick},
  Segments/.style={draw=blue!50!red!70,thick},
  MyCircles/.style={green!50!blue!50,thin}, 
  every node/.style={scale=1}
  ]
%	\draw[color=gray,step=1.0,dotted] (-8,-6) grid (8,6);
  \clip (-8,-6) rectangle (8,6);


  %\draw[->, >=stealth] (-1,0)--(6.5,0) node[right]{$x_1$};
  \draw[-{Latex[length=4.5mm, width=2.5mm]}, >=stealth] (0,-5)--(0,5) node[above left]{$x_2$};

  \draw[-{Latex[length=4.5mm, width=2.5mm]}, >=stealth] (-6,0)--(6,0) 
  node[right]{$x_1$};


  %{\verb!->!new, arrowhead = 2mm, line width=4pt}
  %, arrowhead = 3mm
  %, arrowhead = 0.2

  % Feel free to change here coordinates of points A and B
  \pgfmathparse{0}		\let\Xa\pgfmathresult
  \pgfmathparse{0}		\let\Ya\pgfmathresult
  \coordinate (A) at (\Xa,\Ya);

  \pgfmathparse{-1}		\let\Xb\pgfmathresult
  \pgfmathparse{1}		\let\Yb\pgfmathresult
  \coordinate (B) at (\Xb,\Yb);

  \pgfmathparse{1}		\let\Xc\pgfmathresult
  \pgfmathparse{1}		\let\Yc\pgfmathresult
  \coordinate (C) at (\Xc,\Yc);






  \begin{scope}
  \clip[rotate=45] (0, 0) ellipse (4 and 2);
  \foreach \p in {1,...,800}
  { \fill[black, rotate = 45]  (0 + 4*rand,3*rand) circle (0.035);
  }
  \end{scope}
  \pause


  \draw[line width = 0.5mm, ->, vecb] (-4, -4) -- (4, 4) node[above left]{$p_{1}$}; \pause


  \draw[line width = 0.5mm, ->, vecb] (3, -3) -- (-3, 3) node[above right]{$p_{2}$}; 
\tkzMarkRightAngle[size=0.3, vecb, line width = 0.3mm](B,A,C);


  \end{tikzpicture}


  

  

\end{frame}


\begin{frame}
  \frametitle{Это завуалированный $SVD$!}

  Если $X = U\Sigma V^T$, то $P = XV = U\Sigma$. \pause

  \[
    \begin{array}{l}
    \begin{pmatrix}
        \vert &  & \vert \\
        \bp_1 & \ldots & \bp_k \\
        \vert &    & \vert \\
      \end{pmatrix}  = 
      \begin{pmatrix}
        \vert &  & \vert \\
        \bx_1 & \ldots & \bx_k \\
        \vert &    & \vert \\
      \end{pmatrix}
      \begin{pmatrix}
        \vert &  & \vert \\
        \bv_1 & \ldots & \bv_k \\
        \vert &    & \vert \\
      \end{pmatrix} \\
      \begin{pmatrix}
        \vert &  & \vert \\
        \bp_1 & \ldots & \bp_k \\
        \vert &    & \vert \\
      \end{pmatrix}  = 
      \begin{pmatrix}
        \vert &  & \vert \\
        \bu_1 & \ldots & \bu_n \\
        \vert &    & \vert \\
      \end{pmatrix}
      \begin{pmatrix}
        \sigma_1 & 0 & \ldots & 0 \\
         0 & \sigma_2 & \ldots & 0 \\
        \ldots & \ldots & \ldots & \ldots \\
    0 & 0 & \ldots & \sigma_k \\
    0 & 0 & \ldots & 0 \\
    \ldots & \ldots & \ldots & \ldots \\
    0 & 0 & \ldots & 0 \\
      \end{pmatrix} 
    \end{array} \pause
  \]

  % Если вектор $\bu_1$ растянуть в $\sigma_1$ раз, то получится первая главная компонента $\bp_1$. \pause

  % Если вектор $\bu_2$ растянуть в $\sigma_2$ раз, то получится первая главная компонента $\bp_2$. 


%  \ldots

  

\end{frame}



\begin{frame}
  \frametitle{Свойства}

  Главные компоненты находятся из $SVD$ разложения, $\bp_i = \sigma_i \bu_i$. \pause

  Главные компоненты ортогональны. \\

  Величины $\sigma^2_i$ равны квадратам длин $\norm{\bp_i}^2$ 
  и пропорциональны выборочным дисперсиям $\bp_i$.  \\

Значение максимума суммы 
\[
  \norm{\bp_1}^2 + \ldots + \norm{\bp_d}^2 = \sigma^2_1 + \ldots + \sigma^2_d
\]
одинаково при пошаговой и одновременной максимизации.


\end{frame}


\begin{frame}
  \frametitle{Связь с корреляционной матрицей}
  Если $X = U\Sigma V^T$, то корреляционная матрица имеет вид $C=X^TX = V\Sigma^T \Sigma V^T$. \pause

  Векторы весов $\bv_i$, с которыми исходные переменные входят в компоненты, являются
  собственными векторами корреляционной матрицы $C$. \pause

  Сингулярные значения матрицы $X$ в квадрате являются собственными числами 
  корреляционной матрицы $C$, $\lambda_i = \sigma_i^2$.
  

\end{frame} 

% !TEX root = ../linal_lecture_06.tex

\begin{frame} % название фрагмента

\videotitle{PCA: минимизация ошибки приближения}

\end{frame}



\begin{frame}{Краткий план:}
  \begin{itemize}[<+->]
    \item Наилучшая аппроксимация.
    \item Минимизация ошибки приближения.
  \item Свойства главных компонент.
  \end{itemize}

\end{frame}




\begin{frame}
  \frametitle{Метод главных компонент}


  Есть матрица $X$ исходных наблюдений:

  наблюдения отложены по строкам, 
  
  а переменные — по столбцам. 


  Переменных очень много. 


  Хотим иметь небольшое количество переменных, 
  которые бы почти без потерь 
  содержали всю информацию об исходных переменных. 


\end{frame}


\begin{frame}
  \frametitle{Поиск наилучшего приближения}

  Хотим найти матрицу $\hat X$ такую, чтобы
  при заданном ранге $\rank \hat X = d$ матрица $\hat X$ была бы поближе к $X$:

  \[
  \sum_{ij} (x_{ij} - \hat x_{ij})^2 = \trace ((X - \hat X)^T (X - \hat X)) \to \min \pause 
  \]

  Аппроксимацию с рангом $d$ обозначим $\hat X_d$.
  
\end{frame}


\begin{frame}
  \frametitle{Метод главных компонент}

  Все исходные переменные предварительно стандартизируем! 

  Для каждого столбца $\bx$ выполнены условия $\bar x =0$, $sd(\bx)=1$. 


  На базе столбцов $\bx_1$, $\bx_2$, \ldots, $\bx_k$ матрицы $X$ мы создадим 
  $d\leq k$ новых переменных $\bp_1$, $\bp_2$, \ldots, $\bp_d$. 

  Новые переменные будем создавать по-очереди. 

  Новые переменные будем называть \alert{главными компонентами}.


\end{frame}


\begin{frame}
  \frametitle{Минимизация ошибки приближения}


  Главные компоненты $\bp_1$, $\bp_2$, \ldots, $\bp_d$ будут линейными 
  комбинациями столбцов $X$. \pause

  Матрица $\hat X_i$ содержит проекции столбцов матрицы $X$ на $\Span\{ \bp_1, \bp_2, \ldots, \bp_i\}$.


  \begin{block}{Алгоритм}
    \begin{enumerate}
      \item Компоненту $\bp_1 = X \bv_1$ подберём так, чтобы 
      матрица $\hat X_1$ была наилучшей аппроксимацией $X$, $\sum_{ij} (x_{ij} - \hat{x}_{ij})^2 \to\min$. \pause

\item Компоненту $\bp_2 = X \bv_2$ подберём так, чтобы 
матрица $\hat X_2$ была наилучшей аппроксимацией $X$, $\sum_{ij} (x_{ij} - \hat{x}_{ij})^2 \to\min$
 при условии, что $\bv_2 \perp \bv_1$ и $\norm{\bv_2}=1$. 
%\item Компоненту $\bp_3 = X \bv_3$ подберём так, чтобы 
%матрица $\hat X_3$ была наилучшей аппроксимацией $X$, $\sum_{ij} (x_{ij} - \hat_{x}_ij)^2 \to\min$
 % при условии, что $\bv_3 \perp \bv_2,\, \bv_1$ и $\norm{\bv_3}=1$. 
      \item \ldots
    \end{enumerate}
    
  \end{block}
  

\end{frame}


\begin{frame}
  \frametitle{Картинка}


\begin{tikzpicture}[
  scale=1.4,
  MyPoints/.style={draw=blue,fill=white,thick},
  Segments/.style={draw=blue!50!red!70,thick},
  MyCircles/.style={green!50!blue!50,thin}, 
  every node/.style={scale=1}
  ]
%	\draw[color=gray,step=1.0,dotted] (-8,-6) grid (8,6);
  \clip (-8,-6) rectangle (8,6);


  %\draw[->, >=stealth] (-1,0)--(6.5,0) node[right]{$x_1$};
  \draw[-{Latex[length=4.5mm, width=2.5mm]}, >=stealth] (0,-5)--(0,5) node[above left]{$x_2$};

  \draw[-{Latex[length=4.5mm, width=2.5mm]}, >=stealth] (-6,0)--(6,0) 
  node[right]{$x_1$};


  %{\verb!->!new, arrowhead = 2mm, line width=4pt}
  %, arrowhead = 3mm
  %, arrowhead = 0.2

  % Feel free to change here coordinates of points A and B
  \pgfmathparse{0}		\let\Xa\pgfmathresult
  \pgfmathparse{0}		\let\Ya\pgfmathresult
  \coordinate (A) at (\Xa,\Ya);

  \pgfmathparse{-1}		\let\Xb\pgfmathresult
  \pgfmathparse{1}		\let\Yb\pgfmathresult
  \coordinate (B) at (\Xb,\Yb);

  \pgfmathparse{1}		\let\Xc\pgfmathresult
  \pgfmathparse{1}		\let\Yc\pgfmathresult
  \coordinate (C) at (\Xc,\Yc);






  \begin{scope}
  \clip[rotate=45] (0, 0) ellipse (4 and 2);
  \foreach \p in {1,...,800}
  { \fill[black, rotate = 45]  (0 + 4*rand,3*rand) circle (0.035);
  }
  \end{scope}
  \pause


  \draw[line width = 0.5mm, ->, vecb] (-4, -4) -- (4, 4) node[above left]{$p_{1}$}; \pause


  \draw[line width = 0.5mm, ->, vecb] (3, -3) -- (-3, 3) node[above right]{$p_{2}$}; 
\tkzMarkRightAngle[size=0.3, vecb, line width = 0.3mm](B,A,C);


  \end{tikzpicture}


\end{frame}











\begin{frame}
  \frametitle{И это завуалированный $SVD$!}

  Если $X = U\Sigma V^T$, то $P = XV = U\Sigma$. \pause


  Аппроксимация $\hat X_d$ с рангом $\rank \hat X_d = d$ строится так:
  \[
  \hat X_d = U \Sigma_d V^T,  
  \]
  где матрицу $\Sigma_d$ получили из матрицы $\Sigma$ оставив в ней только $d$ штук
  самых больших сингулярных значений и занулив остальные.


\end{frame}



\begin{frame}
  \frametitle{Свойства}

  Главные компоненты находятся из $SVD$ разложения, $\bp_i = \sigma_i \bu_i$. 

  Главные компоненты ортогональны. 

  Величины $\sigma^2_i$ равны квадратам длин $\norm{\bp_i}^2$ 
  и пропорциональны выборочным дисперсиям $\bp_i$.  \pause

  Значение суммы 
  \[
    \norm{\bp_1}^2 + \ldots + \norm{\bp_d}^2 = \sigma^2_1 + \ldots + \sigma^2_d
  \]
  одинаково при пошаговой и одновременной аппроксимации матрицы $X$.


\end{frame}

 

% !TEX root = ../linal_lecture_06.tex

\begin{frame} % название фрагмента

\videotitle{PCA: максимизация $R^2$}

\end{frame}



\begin{frame}{Краткий план:}
  \begin{itemize}[<+->]
    \item Коэффициент детерминации.
    \item Максимизация $R^2$.
    \item Свойства главных компонент.
  \end{itemize}

\end{frame}




\begin{frame}
  \frametitle{Коэффициент детерминации}

  
  Рассмотрим нахождение проекции $\bhy$ вектора $\by$ на
  линейную оболочку $\Span\{ \bx_1, \ldots, \bx_k\}$ столбцов матрицы $X$. \pause

  Предположим, что среди $\bx_i$ есть вектор-константа. \pause 

  \begin{block}{Определение}
    \alert{Коэффициентом детерминации $R^2$} называют квадрат косинуса угла 
    между центрированный вектором $\by'$, $y_i' = y_i - \bar y$,
    и его проекцией $\bhy'$ на столбцы $X$.
    \[
      R^2 = \cos^2 \angle(\bhy', \by') = \frac{\norm{\bhy'}^2}{\norm{\by'}^2} \pause
    \]
  \end{block}

  Коэффициент детерминации — доля «объяснённого разброса». 

\end{frame}

\begin{frame}
\frametitle{Коэффициент детерминации}

  \begin{block}{Утверждение}
  Коэффициент детерминации равен 
    квадрату выборочной корреляции между исходным $\by$ и его проекцией $\bhy$ на столбцы $X$:
    \[
      R^2 = \rho^2 (\bhy, \by) = \frac{\left( \sum_i (y_i - \bar y) (\hat y_i - \bar y) \right)^2}{\sum_i (y_i - \bar y)^2 \sum_i (\hat y_i - \bar y)^2 }
    \]
  \end{block}

\end{frame}




\begin{frame}
  \frametitle{Метод главных компонент}

  Все исходные переменные $\bx_1$, $\bx_2$, \ldots, $\bx_k$ предварительно стандартизируем! 

  Для каждого столбца $\bx$ выполнены условия $\bar x =0$, $sd(\bx)=1$. 


  На базе столбцов $\bx_1$, $\bx_2$, \ldots, $\bx_k$ матрицы $X$ мы создадим 
  $d\leq k$ новых переменных $\bp_1$, $\bp_2$, \ldots, $\bp_d$. 

  Новые переменные будем создавать по-очереди. 

  Новые переменные будем называть \alert{главными компонентами}.

  % PCA — \alert{principal component analysis}.

\end{frame}

\begin{frame}
  \frametitle{Обозначение}


  Обозначим $R^2_i(\bx)$ —
  коэффициент детерминации при проецировании вектора $\bx$ на 
  линейную оболочку первых~$i$~главных компонент, $\Span \{ \bp_1, \bp_2, \ldots, \bp_i  \}$. \pause


  Коэффициент $R^2_i(\bx)$ показывает, насколько хорошо первые~$i$~главных компонент
  предсказывают вектор $\bx$. 

\end{frame}


\begin{frame}
  \frametitle{Максимизация суммы $R^2$}

  Главные компоненты $\bp_1$, $\bp_2$, \ldots, $\bp_d$ будут линейными 
  комбинациями столбцов $X$. \pause

  \begin{block}{Алгоритм}
    \begin{enumerate}
      \item Компоненту $\bp_1 = X \bv_1$ подберём так, чтобы 
     максимизировать сумму $R_1^2(\bx_1) + \ldots + R_1^2(\bx_k)$ 
     при условии, что $\norm{\bv_1}=1$. \pause
\item Компоненту $\bp_2 = X \bv_2$ подберём так, чтобы 
максимизировать сумму $R_2^2(\bx_1) + \ldots + R_2^2(\bx_k)$ 
    при условии, что $\bv_2 \perp \bv_1$ и $\norm{\bv_2}=1$. 
%  \item Компоненту $\bp_3 = X \bv_3$ подберём так, чтобы 
 % выборочная дисперсия $\bp_3$ была максимальной при условии, что $\bv_3 \perp \bv_2,\, \bv_1$ и $\norm{\bv_3}=1$. 
      \item \ldots
    \end{enumerate}
    
  \end{block}
  

\end{frame}



\begin{frame}
  \frametitle{И это завуалированный $SVD$!}

  Если $X = U\Sigma V^T$, то $P = XV = U\Sigma$.

  Главные компоненты находятся из $SVD$ разложения, $\bp_i = \sigma_i \bu_i$. 

  Главные компоненты ортогональны. 

  Величины $\sigma^2_i$ равны квадратам длин $\norm{\bp_i}^2$ 
  и пропорциональны выборочным дисперсиям $\bp_i$.  \pause

Значение суммы 
\[
  \norm{\bp_1}^2 + \ldots + \norm{\bp_d}^2 = \sigma^2_1 + \ldots + \sigma^2_d
\]
одинаково при пошаговой и одновременной максимизации.


\end{frame}


\begin{frame}
  \frametitle{Смысл сингулярных значений}

  Квадраты сингулярных значений показывают увеличение суммарного коэффициента детерминации!

%   \[
%     \begin{array}{rl}
%   \frac{\sigma^2_1}{\sigma^2_1 + \ldots + \sigma^2_k} &= \frac{1}{k} (R_1^2(\bx_1) + \ldots + R_1^2(\bx_k)) \\ \pause 
%   \frac{\sigma^2_1 + \sigma^2_2}{\sigma^2_1 + \ldots + \sigma^2_k} &= \frac{1}{k} (R_2^2(\bx_1) + \ldots + R_2^2(\bx_k)) \\ \pause
%  \frac{\sigma^2_1 + \sigma^2_2 + \sigma^2_3}{\sigma^2_1 + \ldots + \sigma^2_k} &= \frac{1}{k} (R_3^2(\bx_1) + \ldots + R_3^2(\bx_k))  
%     \end{array}
%   \]

\[
 \frac{\sigma^2_1}{\sigma^2_1 + \ldots + \sigma^2_k} = \frac{1}{k} (R_1^2(\bx_1) + \ldots + R_1^2(\bx_k))  \pause 
 \]
 \[
 \frac{\sigma^2_1 + \sigma^2_2}{\sigma^2_1 + \ldots + \sigma^2_k} = \frac{1}{k} (R_2^2(\bx_1) + \ldots + R_2^2(\bx_k)) \pause
 \]
\[
 \frac{\sigma^2_1 + \sigma^2_2 + \sigma^2_3}{\sigma^2_1 + \ldots + \sigma^2_k} = \frac{1}{k} (R_3^2(\bx_1) + \ldots + R_3^2(\bx_k))  
 \]

  

\end{frame}


% \begin{frame}
%   \frametitle{Связь с корреляционной матрицей}
%   Если $X = U\Sigma V^T$, то корреляционная матрица имеет вид $C=X^TX = V\Sigma^T \Sigma V^T$. \pause

%   Векторы весов $\bv_i$, с которыми исходные переменные входят в компоненты, являются
%   собственными векторами корреляционной матрицы $C$. \pause

%   Сингулярные значения матрицы $X$ в квадрате являются собственными числами 
%   корреляционной матрицы $C$, $\lambda_i = \sigma_i^2$.
  

% \end{frame}



\begin{frame}
  \frametitle{Резюме}

  \begin{itemize}[<+->]
  \item Жорданова нормальная форма.
  \item Сингулярное разложение.
  \item PCA: максимизация разброса.
  \item PCA: минимизация ошибки приближения.
  \item PCA: максимизация суммарного $R^2$.
  \item Скринкаст с SVD.
  \item Бонусное видео: геометрическая алгебра.
  \end{itemize}
  \pause
  
  \alert{Большое спасибо героям, прошедшим курс!}
      


\end{frame} 



\begin{frame}
\lecturetitle{Скринкаст: SVD для снижения размерности}
\todo{Это видеофрагмент с доской, слайдов здесь нет :)}
\end{frame}
    


\begin{frame}
\lecturetitle{Бонус: геометрическая алгебра}
\todo{Это видеофрагмент с доской, слайдов здесь нет :)}
\end{frame}
    


\end{document}
