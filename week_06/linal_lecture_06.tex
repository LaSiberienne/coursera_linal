\documentclass[14 pt,xcolor=dvipsnames]{beamer}

\usepackage{epsdice}

\usepackage[absolute,overlay]{textpos}

\usepackage[orientation=portrait,size=custom,width=25.4,height=19.05]{beamerposter}

%25,4 см 19,05 см размеры слайда в powerpoint

\usetheme{metropolis}
\metroset{
  %progressbar=none,
  numbering=none,
  subsectionpage=progressbar,
  block=fill
}

%\usecolortheme{seahorse}

\usepackage{fontspec}
\usepackage{polyglossia}
\setmainlanguage{russian}


\usepackage{fontawesome5} % removed [fixed]
\setmainfont[Ligatures=TeX]{Myriad Pro}
\setsansfont{Myriad Pro}


\usepackage{amssymb,amsmath,amsxtra,amsthm}


\usepackage{unicode-math}
\usepackage{centernot}

\usepackage{graphicx}
\graphicspath{{img/}}

\usepackage{wrapfig}
\usepackage{animate}
\usepackage{tikz}
%\usetikzlibrary{shapes.geometric,patterns,positioning,matrix,calc,arrows,shapes,fit,decorations,decorations.pathmorphing}
\usepackage{pifont}
\usepackage{comment}
\usepackage[font=small,labelfont=bf]{caption}
\captionsetup[figure]{labelformat=empty}
\includecomment{techno}

\usefonttheme[onlymath]{serif}


%Расположение

\setbeamersize{text margin left=15 mm,text margin right=5mm} 
\setlength{\leftmargini}{38 pt}

%\usepackage{showframe}
%\usepackage{enumitem}
%\setlist{leftmargin=5.5mm}


%Цвета от дирекции

\definecolor{dirblack}{RGB}{58, 58, 58}
\definecolor{dirwhite}{RGB}{245, 245, 245}
\definecolor{dirred}{RGB}{149, 55, 53}
\definecolor{dirblue}{RGB}{0, 90, 171}
\definecolor{dirorange}{RGB}{235, 143, 76}
\definecolor{dirlightblue}{RGB}{75, 172, 198}
\definecolor{dirgreen}{RGB}{155, 187, 89}
\definecolor{dircomment}{RGB}{128, 100, 162}

\setbeamercolor{title separator}{bg=dirlightblue!50, fg=dirblue}

%Цвета блоков

\setbeamercolor{block title}{bg=dirblue!30,fg=dirblack}

\setbeamercolor{block title example}{bg=dirlightblue!50,fg=dirblack}

\setbeamercolor{block body example}{bg=dirlightblue!20,fg=dirblack}

\AtBeginEnvironment{exampleblock}{\setbeamercolor{itemize item}{fg=dirblack}}
%\setbeamertemplate{blocks}[rounded][shadow]

% Набор команд для удобства верстки

\newcommand{\RR}{\mathbb{R}}
\newcommand{\ZZ}{\mathbb{Z}}
\newcommand{\la}{\lambda}

% Набор команд для структуризации

%\newcommand{\quest}{\faQuestionCircleO}
%\faPencilSquareO \faPuzzlePiece \faQuestionCircleO  \faIcon*[regular]{file} {\textcolor{dirblue}
%\newcommand{\quest}{\textcolor{dirblue}{\boxed{\textbf{?}}}
\newcommand{\task}{\faIcon{tasks}}
\newcommand{\exmpl}{\faPuzzlePiece}
\newcommand{\dfn}{\faIcon{pen-square}}
\newcommand{\quest}{\textcolor{dirblue}{\faQuestionCircle[regular]}}
\newcommand{\acc}[1]{\textcolor{dirred}{#1}}
\newcommand{\accm}[1]{\textcolor{dirred}{#1}}
\newcommand{\acct}[1]{\textcolor{dirblue}{#1}}
\newcommand{\acctm}[1]{\textcolor{dirblue}{#1}}
\newcommand{\accex}[1]{\textcolor{dirblack}{\bf #1}}
\newcommand{\accexm}[1]{\textcolor{dirblack}{ \mathbf{#1}}}
\newcommand{\acclp}[1]{\textcolor{dirorange}{\it #1}}


\newcommand{\videotitle}[1]{\begin{center}
    \textcolor{dirblue}{#1}

    \todo{название видеофрагмента}
\end{center}}

\newcommand{\lecturetitle}[1]{\begin{center}
    \textcolor{dirblue}{#1}

    \todo{название лекции}
\end{center}}




\newcommand{\todo}[1]{\textcolor{dircomment}{\bf #1}}

\newcommand{\spcbig}{\vspace{-10 pt}}
\newcommand{\spcsmall}{\vspace{-5 pt}}

%\usepackage{listings}
%\lstset{
%xleftmargin=0 pt,
%  basicstyle=\small, 
%  language=Python,
  %tabsize = 2,
%  backgroundcolor=\color{mc!20!white}
%}



%\newcommand{\mypart}[1]{\begin{frame}[standout]{\huge #1}\end{frame}}

\setbeamercolor{background canvas}{bg=}

% frame title setup
\setbeamercolor{frametitle}{bg=,fg=dirblue}
\setbeamertemplate{frametitle}[default][left]

\addtobeamertemplate{frametitle}{\hspace*{-0.5 cm}}{\vspace*{0.25cm}}


%Шрифты
\setbeamerfont{frametitle}{family=\rmfamily,series=\bfseries,size={\fontsize{33}{30}}}
\setbeamerfont{framesubtitle}{family=\rmfamily,series=\bfseries,size={\fontsize{26}{20}}}





\usepackage{physics}
\newcommand{\R}{\mathbb{R}}

\usepackage[outline]{contour}




\usepackage{pgfplots}
\pgfplotsset{compat=newest}

\usepackage{tikz}
\usetikzlibrary{calc}
\usetikzlibrary{quotes,angles}
\usetikzlibrary{arrows}
\usetikzlibrary{arrows.meta}
\usetikzlibrary{positioning,intersections,decorations.markings}
\usetikzlibrary{patterns}

\usepackage{tkz-euclide} 

\newcommand{\grid}{\draw[color=gray,step=1.0,dotted] (-2.1,-2.1) grid (9.6,6.1)}

\newcommand{\ba}{\symbf{a}}
\newcommand{\be}{\symbf{e}}
\newcommand{\bb}{\symbf{b}}
\newcommand{\bc}{\symbf{c}}
\newcommand{\bd}{\symbf{d}}
\newcommand{\bx}{\symbf{x}}
\newcommand{\bv}{\symbf{v}}
\newcommand{\bzero}{\symbf{0}}


\DeclareMathOperator{\Lin}{Span}
\DeclareMathOperator{\Span}{Span}
\DeclareMathOperator{\LL}{L}

%\tikzset{>=latex}

\colorlet{veca}{red}
\colorlet{vecb}{blue}
\colorlet{vecc}{olive}



% ## Матрица линейного оператора


% Видео 1. Линейная комбинация векторов. Линейно зависимые и независимые наборы векторов.

% Видео 2. Линейная оболочка векторов. Базис линейной оболочки. Размерность линейной оболочки векторов. 
% След проекции — просто как особый случай (!)

% Видео 1. Общее понятие линейного пространства. Базис линейного пространства. Размерность.

% Видео 5. Множество значений линейного оператора как линейная оболочка. Ранг линейного оператора: поворот, проекция, отражение. 

% Видео 6. (доска) Матрица как способ записи линейного оператора. 

% Видео 8. (доска) Матрицы поворота, проекции, растяжения. Единичная матрица. 

% Видео 7. (доска) Умножение матрицы на вектор как линейная комбинация столбцов. 

% Видео 9. (доска) Умножение матриц как последовательное применение операторов. 

% Видео 10. (доска) Метод Гаусса решения систем. 

% задача про краски?
% или про коров? 101 корова, убираем любую одну и всегда сможем поделить 50 на 50 на стада с равными массами



\begin{document}

% \maketitle


\begin{frame} % название лекции


\lecturetitle{Матричная запись}

\end{frame}


% !TEX root = ../linal_lecture_02.tex

\begin{frame} % название фрагмента

\videotitle{Линейная оболочка}

\end{frame}



\begin{frame}{Краткий план:}
  \begin{itemize}[<+->]
    \item Вектор — это столбец чисел.
    \item Сложение двух векторов и умножение на число.
    \item Расстояние и косинус угла между векторами.
  \end{itemize}

\end{frame}


\begin{frame}{Линейная комбинация}

\begin{itemize}[<+->]
  \item Определение. Вектор $\bv$ называется \alert{линейной комбинацией} векторов $\bx_1$, $\bx_2$, \ldots, $\bx_k$, 
если его можно представить в виде их суммы с некоторыми действительными весами $\alpha_i$:

\[
  \bv = \alpha_1 \bx_1 + \alpha_2 \bx_2 + \ldots + \alpha_k \bx_k
\]

\item Пример. Вектор $\begin{pmatrix}
  4 \\
  5 \\
\end{pmatrix}$ — это линейная комбинация векторов $\begin{pmatrix}
  1 \\
  0 \\
\end{pmatrix}$ и $\begin{pmatrix}
  1 \\
  1 \\
\end{pmatrix}$:

\[
\begin{pmatrix}
  4 \\
  5 \\
\end{pmatrix} = -1 \begin{pmatrix}
  1 \\
  0 \\
\end{pmatrix} + 5 \begin{pmatrix}
  1 \\
  1 \\
\end{pmatrix}  
\]


\end{itemize}


\end{frame}



\begin{frame}{Любой вектор — линейная комбинация}


\begin{itemize}[<+->]
  \item Любой вектор $\bv \in \R^2$ — линейная комбинация векторов $\begin{pmatrix}
    1 \\
    0 \\
  \end{pmatrix}$ и $\begin{pmatrix}
    0 \\
    1 \\
  \end{pmatrix}$:

\[
\begin{pmatrix}
  v_1 \\
  v_2 \\
\end{pmatrix} = 
v_1 \begin{pmatrix}
    1 \\
    0 \\
  \end{pmatrix} + 
  v_2 \begin{pmatrix}
    0 \\
    1 \\
  \end{pmatrix}
\]

\item Аналогично, для вектора  $\bv \in \R^3$:

\[
\begin{pmatrix}
v_1 \\
v_2 \\
v_3 \\
\end{pmatrix} = 
v_1 \begin{pmatrix}
  1 \\
  0 \\
  0 \\
\end{pmatrix} + 
v_2 \begin{pmatrix}
  0 \\
  1 \\
  0 \\
\end{pmatrix} +
v_3 \begin{pmatrix}
  0 \\
  0 \\
  1 \\
\end{pmatrix} 
\]



\end{itemize}
  

\end{frame}



\begin{frame}
  \frametitle{Линейная зависимость}


  Определение.

  \begin{itemize}[<+->]
    \item Набор $A$ из двух и более векторов называется 
    \alert{линейно зависимым}, если хотя бы один вектор является линейной комбинацией остальных.
    \item Набор $A$ из одного нулевого вектора также называется \alert{линейно зависимым}.
  \end{itemize}
  

\end{frame}


\begin{frame}
  \frametitle{Линейная зависимость: пример}




  \begin{itemize}[<+->]
    \item Набор $A = \left\{ \begin{pmatrix}
      0 \\
      2 \\
    \end{pmatrix}, \begin{pmatrix}
      3 \\
      4 \\
    \end{pmatrix} \right\}$ — линейно независимый.
    \item Набор $A = \left\{ \begin{pmatrix}
      0 \\
      2 \\
      0 \\
    \end{pmatrix}, \begin{pmatrix}
      3 \\
      4 \\
      0 \\
    \end{pmatrix},
    \begin{pmatrix}
      1 \\
      0 \\
      0 \\
    \end{pmatrix} \right\}$ — линейно зависимый:

    \[
      \begin{pmatrix}
        3 \\
        4 \\
        0 \\
      \end{pmatrix} = 2
    \begin{pmatrix}
      0 \\
      2 \\
      0 \\
    \end{pmatrix} + 3
    \begin{pmatrix}
      1 \\
      0 \\
      0 \\
    \end{pmatrix}  
    \]


  \end{itemize}
  

\end{frame}


\begin{frame}
  \frametitle{Линейная зависимость: дубль два}

  Определение-2. Набор векторов $A = \{ \bv_1, \bv_2, \ldots, \bv_k\}$ называется \alert{линейно зависимым},
  если можно найти такие числа $\alpha_1$, $\alpha_2$, \ldots, $\alpha_k$, что
  не все из них равны нулю и 

  \[
  \alpha_1 \bv_1 + \alpha_2 \bv_2 + \ldots + \alpha_k \bv_k = \bzero  
  \]

  
\end{frame}




\end{document}
