% !TEX root = ../linal_lecture_06.tex

\begin{frame} % название фрагмента

\videotitle{Немного статистики}

\end{frame}



\begin{frame}{Краткий план:}
  \begin{itemize}[<+->]
    \item Выборочная дисперсия.
    \item Стандартизация.
    \item Выборочная корреляция.
  \end{itemize}

\end{frame}

\begin{frame}
  \frametitle{Центрирование}


  \begin{block}{Определение}
    Для вектора чисел $\bx = (x_1, x_2, \ldots, x_n)$ \alert{выборочным средним} называют величину
    \[
    \bar x = \frac{x_1 + x_2 + \ldots + x_n}{n}. \pause
    \]
  \end{block}

  \begin{block}{Определение}
    \alert{Центрирование переменной} — переход от набора чисел
    $(x_1, x_2, \ldots, x_n)$ к набору чисел $(x_1 - \bar x, x_2 - \bar x, \ldots, x_n - \bar x)$. \pause
  \end{block}

  Пример. $(4, 2, 3, 3) \to (1, -1, 0, 0)$.

\end{frame}


\begin{frame}
  \frametitle{Свойства центрирования}

  \begin{block}{Утверждение}
Если $\bx'$ — это центрированный вектор $\bx$, $y_i = x_i - \bar x$, то 
$\bar x' = 0$. \pause
\[
\sum_{i=1}^n x'_i = \sum_{i=1}^n (x_i - \bar x) = \sum_{i=1}^n x_i - n\bar x =0  \pause
\]
  \end{block}

  С геометрической точки зрения:
  
  $(\bar x, \bar x, \ldots, \bar x)$ — проекция вектора $\bx$ на $\Span \bv$, где
  $\bv = (1, 1, \ldots, 1)$. \pause


$(x_1 - \bar x, x_2 - \bar x, \ldots, x_n - \bar x)$ — проекция вектора $\bx$ на $\Span^{\perp} \bv$.


\end{frame}



\begin{frame}
  \frametitle{Выборочная дисперсия}

  \begin{block}{Определение}
    \alert{Выборочной дисперсией} набора чисел $(x_1, x_2, \ldots, x_n)$ 
    называют величину $\norm{\bx'}^2 / (n-1)$, где вектор $\bx'$ — это 
    центрированный вектор $\bx$. 
    \[
    \frac{\sum_{i=1}^n(x_i - \bar x)^2}{n-1}.  
    \]
    \pause
  \end{block}

  Выборочная дисперсия вектора показывает «разброс» $x_i$, 
  насколько далеки $x_i$ от своего среднего $\bar x$.
  

\end{frame}


\begin{frame}
  \frametitle{Стандартное отклонение}

  Если $x_i$ измеряется в сантиментрах, 
  то выборочная дисперсия измеряется в квадратных сантиметрах. \pause

  \begin{block}{Определение}
    \alert{Выборочным стандартным отклонением} набора $\bx$ называется 
    корень из выборочной дисперсии 
    \[
      sd(\bx) = \sqrt{\frac{\sum_{i=1}^n(x_i - \bar x)^2}{n-1}}. \pause
    \]
  \end{block}

  \pause 
  Если $x_i$ измеряется в сантиментрах, 
  то выборочное стандартное отклонение измеряется в  сантиметрах. 
  

\end{frame}

\begin{frame}
  \frametitle{Стандартизация}

  Популярным вариантом масштабирования переменной является переход к безразмерной 
  величине по принципу:
  \[
  x_i  \to \frac{x_i - \bar x}{sd(\bx)}.  \pause
  \]

  После стандартизации величина имеет нулевое среднее и единичное стандартное отлонение.


\end{frame}


\begin{frame}
  \frametitle{Выборочная корреляция}

  \begin{block}{Определение}
    \alert{Выборочной корреляцией} двух наборов чисел $\bx$ и $\by$
    называют косинус угла между их центрированными версиями:
    \[
    \rho(\bx, \by) = \cos \angle (\bx', \by') = \frac{\langle \bx', \by' \rangle}{\norm{\bx'} \norm{\by'}},   
    \]
    где $x'_i = x_i - \bar x$ и $y_i' = y_i - \bar y$. \pause
  \end{block}

Не определена, если $\bx$ или $\by$ состоит из одинаковых чисел. \pause
 
Лежит в диапазоне от $-1$ до $1$. 

\end{frame}


\begin{frame}
  \frametitle{Выборочная корреляция}

  \begin{block}{Утверждение}
    Выборочная корреляция равна 
    \[
      \rho(\bx, \by) = \frac{\sum (x_i - \bar x)(y_i - \bar y)}{\sqrt{\sum (x_i - \bar x)^2 \sum (y_i - \bar y)}}.
    \]
  \end{block}

 
  Если после центрирования векторы ортогональны, то выборочная корреляция равна $0$. \pause

  Не чувствительна к масштабированию.

\end{frame}


\begin{frame}
  \frametitle{Выборочная корреляционная матрица}

  \begin{block}{Определение}
  \alert{Выборочной корреляционной матрицей} переменных $\bx_1$, $\bx_2$, \ldots, $\bx_k$ называется
  матрица $C$, элементами которой являются выборочные корреляции,
  \[
  c_{ij} = \rho(\bx_i, \bx_j) \pause
  \] 
  \end{block}

  Если переменные $\bx_1$, $\bx_2$, \ldots, $\bx_k$ стандартизированы, то $C=X^TX$. \pause

  Матрица $C$ симметрична и положительно полуопределена. 

\end{frame}


