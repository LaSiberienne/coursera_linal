% !TEX root = ../linal_lecture_02.tex

\begin{frame} % название фрагмента

\videotitle{Системы линейных уравнений}

\end{frame}



\begin{frame}{Краткий план:}
  \begin{itemize}[<+->]
    \item Однородная система и ядро оператора;
    \item Структура множества решений.
  \end{itemize}

\end{frame}


\begin{frame}{Варианты записи системы}

Скалярный: 
$\begin{cases}
5x_1 + 6x_2 = 8 \\
3x_1 + 7x_2 = 9 \\
\end{cases}$

\pause

Векторный:
$x_1 \begin{pmatrix}
    5 \\
    3 
\end{pmatrix} + 
x_2 \begin{pmatrix}
    6 \\
    7
\end{pmatrix} = 
\begin{pmatrix}
    8 \\
    9
\end{pmatrix}$.
\pause

Матричный:
$\begin{pmatrix}
    5 & 6 \\
    3 & 7 \\
\end{pmatrix} \cdot 
\begin{pmatrix}
    x_1 \\
    x_2 \\
\end{pmatrix} = \begin{pmatrix}
    8 \\
    9 \\
\end{pmatrix}$.


\end{frame}


\begin{frame}{Однородная и неоднородная системы}

\begin{block}{Определение}
    Система уравнений $A\bx = 0$ называется \alert{однородной}.
\end{block}
\pause

Однородная система: 
$\begin{pmatrix}
    5 & 6 \\
    3 & 7 \\
\end{pmatrix} \cdot 
\begin{pmatrix}
    x_1 \\
    x_2 \\
\end{pmatrix} = \begin{pmatrix}
    0 \\
    0 \\
\end{pmatrix}$.
\pause

Неоднородная система: 
$\begin{pmatrix}
    5 & 6 \\
    3 & 7 \\
\end{pmatrix} \cdot 
\begin{pmatrix}
    x_1 \\
    x_2 \\
\end{pmatrix} = \begin{pmatrix}
    8 \\
    9 \\
\end{pmatrix}$.


\end{frame}



\begin{frame}{Ядро оператора}

\begin{block}{Определение}
\alert{Ядром} линейного оператора $\LL: \R^n \to \R^k$ называется множество векторов,
которые под действием $\LL$ превращаются в $\bzero \in \R^k$: 
\[
\ker \LL = \{ \bv \in \R^n \mid \LL \bv = \bzero \}    
\]
\end{block}

\pause
Чтобы найти ядро $\LL$ нужно решить однородную систему $\LL \bv = \bzero$.

\end{frame}



\begin{frame}{Метод Гаусса}

Основная идея: по очереди избавиться от всех неизвестных.
\pause

\begin{block}{Алгоритм}
\begin{enumerate}
    \item Выберем первое уравнение так, чтобы в нём была переменная $x_1$.
    \pause
    \item Вычитаем первое уравнение из остальных так, чтобы в них пропала переменная $x_1$.
    \pause
    \item Зафиксируем первое уравнение и работаем с остальными. 
\end{enumerate}
\end{block}


\pause
В финальной системе в каждом следующем уравнении меньше неизвестных, чем в предыдущем.



\end{frame}



\begin{frame}{Ступенчатый вид}


После применения метода Гаусса система примет ступенчатый вид:
\[
\left[
\begin{array}{ccccc|c}
\textcolor{red}{2} & 0 & 3 & -1 & 5 & 2 \\
0 & \textcolor{red}{1} & 1 & -1 & -2 & 3 \\
0 & 0 & 0 & \textcolor{red}{3} & 0 & -1 \\
\end{array}
\right]
\]

\pause
Неизвестные, лежащие в начале «ступеньки», называются \alert{главными}, а остальные —
\alert{свободными}.

Главные переменные можно выразить через свободные.

\end{frame}





\begin{frame}{Количество решений}
\begin{block}{Утверждение}
Система уравнений $A \bx = \bb$ имеет ноль, одно или бесконечное количество решений. 
\end{block}
\pause

\begin{block}{Доказательство}
После применения метода Гаусса последнее уравнение, в котором хотя бы один коэффициент 
отличен от нуля, окажется одного из трёх видов:
\[
\begin{array}{l}
A: 0x_1 + 0x_2 + 0x_3 + 0x_4 = 7, \text{ нет решений.}\\    
B: 0x_1 + 0x_2 + 0x_3 + 5x_4 = 7, \text{ хотя бы одно решение.} \\
C: 0x_1 + 3x_2 + 2x_3 + 5x_4 = 7, \text{ бесконечное количество.} \\
\end{array}
\]
\end{block}
\pause

В случае C мы получаем в последнем уравнении свободу выбора $x_3$ и $x_4$.

\end{frame}



\begin{frame}{Структура множества решений}

\begin{block}{Утверждение}
Если решений бесконечное множество, то ответ можно записать в виде:
\[
\bx = \ba + \alpha_1 \bv_1 + \ldots + \alpha_k \bv_k,
\]
где $\ba$, $\bv_1$, \ldots, $\bv_k$ — конкретные векторы, а
$\alpha_1$, \ldots, $\alpha_k$ — произвольные числа. 
\end{block}

\pause
Для однородной системы $\ba=\bzero$, 
    а число $k$ является размерностью множества решений, $k=\dim\ker A$.

\end{frame}




\begin{frame}{Резюме}


\begin{itemize}[<+->]
\item Линейная комбинация и линейная оболочка.
\item Абстрактное векторное пространство.
\item Матрица линейного оператора.
\item Ранг оператора. 
\item Умножение матрицы на вектор и на матрицу.
\item Решение системы методом Гаусса.
\item Бонус: задача о шахматной доске.
\end{itemize}
\pause
\alert{Следующая лекция:} определитель и обратная матрица.



\end{frame}
    