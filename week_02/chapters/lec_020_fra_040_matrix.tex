% !TEX root = ../linal_lecture_02.tex

\begin{frame} % название фрагмента

\videotitle{Матрица линейного оператора}

\end{frame}



\begin{frame}{Краткий план:}
  \begin{itemize}[<+->]
    \item Матрица линейного оператора;
    \item Примеры;
    \item Обобщение на векторное пространство.
  \end{itemize}

\end{frame}


\begin{frame}{Как записать линейный оператор?}

Любой вектор $\bv$ представим в виде:
\[
\bv =  \begin{pmatrix}
    v_1 \\
    v_2 \\
    \vdots \\
    v_n 
\end{pmatrix} =  
v_1  \begin{pmatrix}
    1 \\
    0 \\
    \vdots \\
    0 
\end{pmatrix} + v_2  \begin{pmatrix}
    0 \\
    1 \\
    \vdots \\
    0 
\end{pmatrix} + \ldots +
v_n  \begin{pmatrix}
    0 \\
    0 \\
    \vdots \\
    1 
\end{pmatrix}
\]

\pause
По свойству линейности
\[
\LL \bv =  \LL \begin{pmatrix}
    v_1 \\
    v_2 \\
    \vdots \\
    v_n 
\end{pmatrix} =  
v_1  \LL \begin{pmatrix}
    1 \\
    0 \\
    \vdots \\
    0 
\end{pmatrix} + v_2  \LL \begin{pmatrix}
    0 \\
    1 \\
    \vdots \\
    0 
\end{pmatrix} + \ldots +
v_n  \LL \begin{pmatrix}
    0 \\
    0 \\
    \vdots \\
    1 
\end{pmatrix}
\]

\pause
Достаточно понять, что оператор $\LL$ делает с векторами, содержащими 
одну единичку и нули на остальных местах.

\end{frame}


\begin{frame}{Запишем оператор по столбцам!}


Обозначим $\be_i$ — вектор, у которого на $i$-м месте стоит $1$, а на остальных местах — $0$.
\[
\be_i = \begin{pmatrix}
    0 \\
    \vdots \\
    0 \\
    1 \\
    0  \\
    \vdots \\
    0 \\
\end{pmatrix}    
\]

\pause 
\begin{block}{Определение}
\alert{Матрицей линейного оператора} $\LL$ назовём прямоугольную табличку чисел
в которой $i$-ый столбец равен $\LL \be_i$. 
\end{block}
\end{frame}








\begin{frame}{Векторное пространство}

\begin{block}{Определение} 
Множество $V$ произвольных объектов называется \alert{конечномерным векторным пространством}, если:

\begin{itemize}[<+->]
\item множество $V$ можно взаимно однозначно сопоставить пространству $\R^n$;
\item определено сложение двух объектов $\ba$ и $\bb$ из $V$, 
и оно соответствует сложению столбцов из $\R^n$;
\item определено умножение объекта $\ba$ из $V$ на число $\lambda\in \R^n$, 
и оно соответствует умножению столбца $\R^n$ на $\lambda$.
\end{itemize}
\end{block}

\pause
Элементы векторного пространства называют векторами. 
\pause
Векторное пространство также называют \alert{линейным}.

\end{frame}



\begin{frame}{Многочлены}
Множество $V$ всех многочленов от $t$ степени не выше трёх:
\[
V  = \{ at^3 + bt^2 + ct + d \mid a, b, c, d \in \R^n \}
\]

\pause
Взаимно однозначное сопоставление: 
\[5t^3 + 6t^2 - 3t + 2 \leftrightarrow \begin{pmatrix} 
    5 \\
    6 \\
    -3 \\
    2 \\
\end{pmatrix}.
\]
\pause


Сложение двух многочленов и умножение многочлена на число соответствуют операциям над столбцами чисел.
\end{frame}




\begin{frame}{Пример векторного пространства}

Множество $V$ всех функций $f(t)$ равных нулю вне двух данных точек:
\[
V  = \{ f \mid f(t)=0 \text{ для всех } t\neq \pm 1 \}
\]

\pause
Взаимно однозначное сопоставление: 
\[f \leftrightarrow \begin{pmatrix}
    f(-1) \\
    f(1) \\
\end{pmatrix}.
\]
\pause


Сложение двух таких функций и умножение  число соответствуют операциям над столбцами чисел.
\end{frame}
    

\begin{frame}{Аналогия с $\R^n$}

% Сопоставление абстрактного множества $V$ и множества столбцов $\R^n$ дарит нам:
% \pause
\begin{block}{Определение} 
Вектор $\bc$ называется \alert{линейной комбинацией} векторов $\bv_1$, $\bv_2$, \ldots, $\bv_k$, 
если его можно представить в виде их суммы с некоторыми действительными весами $\alpha_i$:
\[
  \bc = \alpha_1 \bv_1 + \alpha_2 \bv_2 + \ldots + \alpha_k \bv_k
\]
\end{block}

\pause
\begin{block}{Определение} 
Множество векторов $M$, содержащее все возможные линейные комбинации векторов $\bv_1$, 
$\bv_2$, \ldots, $\bv_k$, называется их \alert{линейной оболочкой},
\[
  M = \Span\{ \bv_1, \bv_2, \ldots, \bv_k \}
\]
\end{block}
    
\pause 
Полностью аналогично определяются линейно зависимые и независимые наборы векторов. 

\end{frame}


\begin{frame}{Базис и размерность пространства}

\begin{block}{Определение}
\alert{Базисом векторного пространства} $V$ называется любой набор $\{\be_1, \be_2, \ldots, \be_n\}$,
такой что 
\begin{itemize}
    \item $V = \Span \{\be_1, \be_2, \ldots, \be_n\}$;
    \item векторы $\{\be_1, \be_2, \ldots, \be_n\}$ линейно независимы. 
\end{itemize}
\end{block}

\pause
\begin{block}{Определение}
Число  векторов в базисе, $n$, называют \alert{размерностью пространства} $V$, $\dim V = n$.
\end{block}

\end{frame}


\begin{frame}{Продолжаем аналогию}



Пространство $V$ взаимнооднозначно сопоставлено с $\R^n$ и при этом сложение в $V$ 
соответствует сложению в $\R^n$, 
а умножение на число в $V$ соответствует умножению на число в $\R^n$.
 


\begin{block}{Утверждение}
Линейная независимость в $V$ соответствует линейной независимости в $\R^n$.

Базис в $V$ соответствует базису в $\R^n$.

Размерность $V$ равна размерности $\R^n$, $\dim V = \dim \R^n = n$. 
\end{block}


\end{frame}



\begin{frame}{Формальности}

Мы слишком привыкли к свойствам чисел!

\pause

\vspace{40pt}

\begin{block}{Эквивалентное определение}
Множество $V$ называется \alert{векторным пространством}, если выполнено восемь свойств\ldots
\end{block}

\end{frame}

\begin{frame}{Восемь аксиом: сложение}
\begin{enumerate}
    \item При сложении можно расставлять скобки как хочешь (\alert{ассоциативность}):
    \[
    \ba + (\bb + \bc) = (\ba + \bb) + \bc    
    \]
    \item При сложении можно путать лево и право (\alert{коммутативность}):
    \[
    \ba + \bb = \bb + \ba    
    \]
    \item  Существует \alert{нулевой} вектор $\bzero$:
    \[
    \ba + \bzero = \ba    
    \]
    \item Для любого вектора $\ba$ найдется \alert{противоположный} вектор $-\ba$:
    \[
    \ba + (-\ba) = \bzero    
    \]
\end{enumerate}

\end{frame}


\begin{frame}{Восемь аксиом: умножение}
\begin{enumerate}[resume]
    \item[5.] Умножение вектора на число \alert{совместимо} с умножением чисел:
    \[
    \lambda_1 (\lambda_2\ba) = (\lambda_1 \lambda_2 ) \ba
    \]
    \item[6.]  Умножение на \alert{единицу} не меняет вектор:
    \[
    1\cdot \ba= \ba
    \]
    \item[7.] Раскрывать скобки можно (\alert{дистрибутивность умножения}):
    \[
    \lambda (\ba + \bb) = \lambda \ba + \lambda \bb
    \]
    \item[8.] Раскрывать скобки можно по всякому (\alert{дистрибутивность умножения}):
    \[
    (\lambda_1 + \lambda_2) \ba = \lambda_1 \ba + \lambda_2 \ba
    \]
\end{enumerate}

\end{frame}