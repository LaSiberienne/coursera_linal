% !TEX root = ../linal_lecture_02.tex

\begin{frame} % название фрагмента

\videotitle{Линейная оболочка}

\end{frame}



\begin{frame}{Краткий план:}
  \begin{itemize}[<+->]
    \item Линейная оболочка векторов;
    \item Базис линейной оболочки векторов;
    \item Размерность линейной оболочки векторов.
  \end{itemize}

\end{frame}


\begin{frame}{Линейная оболочка}

\begin{block}{Определение} 
Множество векторов $V$, содержащее все возможные линейные комбинации векторов $\bv_1$, 
$\bv_2$, \ldots, $\bv_k$, называется их \alert{линейной оболочкой},
\[
  V = \Span\{ \bv_1, \bv_2, \ldots, \bv_k \}
\]
\end{block}

\end{frame}



\begin{frame}{Линейная оболочка векторов: картинка}




\begin{center}

  \begin{tikzpicture}[
    scale=1.5,
    MyPoints/.style={draw=black,fill=black,thick},
    Segments/.style={draw=blue!50!red!70,thick},
    MyCircles/.style={green!50!blue!50,thin}, 
    every node/.style={scale=1}
    ]

    %\grid;

    \clip (-1.5,-1.5) rectangle (5.5,5.5);

    \begin{scope}[cm={1,1,1.5,0,(0,0)}]
    \draw[draw=blue!30, dashed] (-1.2,-4.2) grid[step=1] (3.5,7);
    \end{scope}

    %{\verb!->!new, arrowhead = 2mm, line width=4pt}
    %, arrowhead = 3mm
    %, arrowhead = 0.2

    % Feel free to change here coordinates of points A and B
    \pgfmathparse{0}		\let\Xa\pgfmathresult
    \pgfmathparse{0}		\let\Ya\pgfmathresult
    \coordinate (A) at (\Xa,\Ya);

    \pgfmathparse{2}		\let\Xb\pgfmathresult
    \pgfmathparse{0.5}		\let\Yb\pgfmathresult
    \coordinate (B) at (\Xb,\Yb);

    \pgfmathparse{2}		\let\Xd\pgfmathresult
    \pgfmathparse{4}		\let\Yd\pgfmathresult
    \coordinate (D) at (\Xd,\Yd);

    \pgfmathparse{4}		\let\Xc\pgfmathresult
    \pgfmathparse{0}		\let\Yc\pgfmathresult
    \coordinate (C) at (\Xc,\Yc);


    \pgfmathparse{1}		\let\Xe\pgfmathresult
    \pgfmathparse{1}		\let\Ye\pgfmathresult
    \coordinate (E) at (\Xe,\Ye);

    \pgfmathparse{2.5}		\let\Xf\pgfmathresult
    \pgfmathparse{0}		\let\Yf\pgfmathresult
    \coordinate (F) at (\Xf,\Yf);

    \pgfmathparse{4}		\let\Xg\pgfmathresult
    \pgfmathparse{1}		\let\Yg\pgfmathresult
    \coordinate (G) at (\Xg,\Yg);




    \draw[-{Latex[length=4.5mm, width=2.5mm]}, >=stealth, thick] (A)--(D) node[above left]{$\bd$};

    \draw[-{Latex[length=4.5mm, width=2.5mm]}, >=stealth, vecb, thick] (A)--(E) node[right]{$\ba$};

    \draw[-{Latex[length=4.5mm, width=2.5mm]}, >=stealth, vecb, thick] (A)--(F) node[below]{$\bb$};

    \draw[-{Latex[length=4.5mm, width=2.5mm]}, >=stealth, veca, thick] (A)--(G) node[right]{$\bc$};


    \draw[black, dashed] (B)--(D);

    \fill[MyPoints]  (0,0) circle (0.8mm);

    %\node [right,darkgray] at (0.5,-2) {$\bc \in \operatorname{Lin} (\ba, \bb) $ }; 

    %\node [right,darkgray] at (0.5,-3) {$\bd \notin \operatorname{Lin} (\ba, \bb) $ }; 



    \end{tikzpicture}
  \end{center}
  
Вектор $\bc$ лежит в плоскости $\Span\{ \ba, \bb \}$.

Вектор $\bd$ не лежит в плоскости $\Span\{ \ba, \bb \}$.


\end{frame}



\begin{frame}
\frametitle{Линейная зависимость}


\begin{block}{Определение}
Набор $A$ из двух и более векторов называется 
\alert{линейно зависимым}, если хотя бы один вектор является линейной комбинацией остальных.


Набор $A = \{\bzero\}$ из одного нулевого вектора также называется \alert{линейно зависимым}.
\end{block}


\end{frame}


\begin{frame}
\frametitle{Линейная зависимость: примеры}



Набор $A = \left\{ \begin{pmatrix}
      0 \\
      2 \\
    \end{pmatrix}, \begin{pmatrix}
      3 \\
      4 \\
    \end{pmatrix} \right\}$ — линейно независимый.

\pause

Набор $A = \left\{ \begin{pmatrix}
      0 \\
      2 \\
      0 \\
    \end{pmatrix}, \begin{pmatrix}
      3 \\
      4 \\
      0 \\
    \end{pmatrix},
    \begin{pmatrix}
      1 \\
      0 \\
      0 \\
    \end{pmatrix} \right\}$ — линейно зависимый:

    \[
      \begin{pmatrix}
        3 \\
        4 \\
        0 \\
      \end{pmatrix} = 2
    \begin{pmatrix}
      0 \\
      2 \\
      0 \\
    \end{pmatrix} + 3
    \begin{pmatrix}
      1 \\
      0 \\
      0 \\
    \end{pmatrix}  
    \]
  

\end{frame}


\begin{frame}
  \frametitle{Линейная зависимость: дубль два}

\begin{block}{Эквивалентное пределение} Набор векторов $A = \{ \bv_1, \bv_2, \ldots, \bv_k\}$ называется \alert{линейно зависимым},
  если можно найти такие веса $\alpha_1$, $\alpha_2$, \ldots, $\alpha_k$, что
  \[
  \alpha_1 \bv_1 + \alpha_2 \bv_2 + \ldots + \alpha_k \bv_k = \bzero,  
  \]
  и при этом хотя бы одно из чисел $\alpha_i$ отлично от $0$. 
\end{block}

\pause

\begin{block}{Доказательство эквивалентности}
Вектор с ненулевым коэффициентом $\alpha_i$ перед ним можно выразить через остальные. 
\pause

Если вектор $\bv_2$ выражен через $\bv_1$ и $\bv_3$, $\bv_2 = \alpha_1 \bv_1 + \alpha_3 \bv_3$, 
то искомая нулевая линейная комбинация имеет вид: $\alpha_1 \bv_1 +(-1)\bv_2 + \alpha_3 \bv_3=\bzero$.
\end{block}

\end{frame}