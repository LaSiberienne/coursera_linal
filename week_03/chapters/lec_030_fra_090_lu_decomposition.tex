% !TEX root = ../linal_lecture_03.tex

\begin{frame} % название фрагмента

\videotitle{LU-разложение}

\end{frame}



\begin{frame}{Краткий план:}
  \begin{itemize}[<+->]
    \item Гауссовские преобразования: другой взгляд;
    \item LU-разложение;
    \item Применение LU-разложения.
  \end{itemize}

\end{frame}



\begin{frame}
    \frametitle{Треугольные матрицы}

    \begin{block}{Определение}
        Квадратная матрица называется \alert{верхнетреугольной},
        если ниже диагонали у неё стоят нулевые числа, например,
        \[
            \mathrm{U} = 
        \begin{pmatrix}
            4 & 5 & -1 & 2 \\
            0 & 0 & 2 & 6 \\
            0 & 0 & 3 & 1 \\
            0 & 0 & 0 & -1 \\
        \end{pmatrix}.    
        \]
    \end{block}
\pause
При перемножении верхнетреугольных матриц получается верхнетреугольная.
\pause
Определитель треугольной матрицы равен произведению диагональных элементов.

\end{frame}


\begin{frame}
\frametitle{Треугольные матрицы}

    \begin{block}{Определение}
        Квадратная матрица называется \alert{нижнетреугольной},
        если ниже диагонали у неё стоят нулевые числа, например,
        \[
            \LL =
        \begin{pmatrix}
            4 & 0 & 0 & 0 \\
            2 & 1 & 0 & 0 \\
            3 & 3 & 0 & 0 \\
            1 & 2 & 2 & -1 \\
        \end{pmatrix}.    
        \]
    \end{block}
    \pause
    При перемножении нижнетреугольных матриц получается нижнетреугольная.
    \pause
    Определитель треугольной матрицы равен произведению диагональных элементов.

\end{frame}




\begin{frame}
    \frametitle{Гауссовские преобразования}

    Рассмотрим систему уравнений в матричном виде:
    \[
    \left(
    \begin{array}{ccc|c}
    1 & 3 & 3  & 10 \\
    -1 & 1 & 1 & 2 \\
    2 & 7 & 1  & 17 \\
    \end{array}
    \right)
    \]
    \pause

    Гауссовские преобразования уравнений системы:
    
    \begin{enumerate}
        \item Домножение строки на ненулевое число;
        \item Перестановка двух строк местами;
        \item Прибавление к данной строке другой строки, домноженной на произвольное $\lambda$.
    \end{enumerate}
    \pause
    Каждое из этих действий можно закодировать умножением на матрицу!

\end{frame}

\begin{frame}
    \frametitle{Домножение строки как матрица}


    Домножим вторую строка матрицы $A$ на $7$:
    \[
    \begin{pmatrix}
        1 & 0 & 0 \\
        0 & \red{7} & 0 \\
        0 & 0 & 1 \\
    \end{pmatrix}   \cdot 
    \begin{pmatrix}
        a & b & c \\
        d & e & f \\
        g & h & i \\
    \end{pmatrix} =\pause
\begin{pmatrix}
    a & b & c \\
    \red{7d} & \red{7e} & \red{7f} \\
    g & h & i \\
\end{pmatrix}   
    \]


Левая матрица задаёт веса строк для правой матрицы!


\end{frame}



\begin{frame}
    \frametitle{Перестановка строк как матрица}

 
Переставим первую и вторую строки матрицы $A$:

    \[
    \begin{pmatrix}
        0 & \red{1} & 0 \\
        \red{1} & 0 & 0 \\
        0 & 0 & 1 \\
    \end{pmatrix}   \cdot 
    \begin{pmatrix}
        a & b & c \\
        d & e & f \\
        g & h & i \\
    \end{pmatrix} =\pause
\begin{pmatrix}
    \red{d} & \red{e} & \red{f} \\
    \red{a} & \red{b} & \red{c} \\
    g & h & i \\
\end{pmatrix}   
    \]

    Левая матрица задаёт веса строк для правой матрицы!


\end{frame}


\begin{frame}
    \frametitle{Прибавление строки как матрица}

    Из второй строки вычтем первую строку с весом $4$:

    \[
    \begin{pmatrix}
        1 & 0 & 0 \\
        \red{-4} & 1 & 0 \\
        0 & 0 & 1 \\
    \end{pmatrix}   \cdot 
    \begin{pmatrix}
        a & b & c \\
        d & e & f \\
        g & h & i \\
    \end{pmatrix} =\pause
\begin{pmatrix}
    a & b & c \\
    d -\red{4a} & e - \red{4b} & f-\red{4c} \\
    g & h & i \\
\end{pmatrix}   
    \]
    
    %\vspace{10pt}
    Левая матрица задаёт веса строк для правой матрицы!
 \pause
 Прибавлению строки к другой строке ниже соответствует нижнетреугольная матрица.

\end{frame}



\begin{frame}
    \frametitle{Вычитание — антоним прибавления}

    

    \[
    \begin{pmatrix}
        1 & 0 & 0 \\
        \red{-4} & 1 & 0 \\
        0 & 0 & 1 \\
    \end{pmatrix}   \cdot 
\begin{pmatrix}
    1 & 0 & 0 \\
    \red{4} & 1 & 0 \\
    0 & 0 & 1 \\
\end{pmatrix} =
\begin{pmatrix}
    1 & 0 & 0 \\
    0 & 1 & 0 \\
    0 & 0 & 1 \\
\end{pmatrix}   
    \]
    
\pause
Левая матрица отвечает за вычитание первой строки из второй с весом $4$.

\pause
Правая матрица отвечает за прибавление первой строки ко второй с весом $4$.


\end{frame}





\begin{frame}
    \frametitle{Переход к треугольной матрице}

    С помощью метода Гаусса мы приводим квадратную матрицу к ступенчатому верхнетреугольному виду:

    \[
    A \overset{g_1}{\to} A_1  \overset{g_2}{\to} A_2 \overset{g_3}{\to} \ldots A_{k-1}\overset{g_k}{\to} \mathrm{U} 
    \]
    \pause

    В матричном виде:
    \[
    \mathrm{U} = G_k \cdot G_{k-1} \cdot \ldots  \cdot  G_2 \cdot G_{1} A
    \]    

\end{frame}


\begin{frame}
    \frametitle{Уменьшим список разрешённых действий!}

    \begin{block}{Алгоритм приведения к ступенчатому виду}
    \begin{enumerate}
        \item Выберем первое уравнение так, чтобы в нём была переменная $x_1$.
        \item Вычитаем первое уравнение из остальных так, чтобы в них пропала переменная $x_1$.
        \item Зафиксируем первое уравнение и работаем с остальными. 
    \end{enumerate}
    \end{block}
    \pause
    Выводы:
    \begin{itemize}
        \item Можно обойтись без домножения строк на число. \pause
        \item Все перестановки строк можно сделать в начале. 
\end{itemize}

    

\end{frame}


\begin{frame}
    \frametitle{$LU$-разложение}

    С помощью метода Гаусса мы приводим квадратную матрицу к ступенчатому верхнетреугольному виду:

    \[
    A \overset{p}{\to} A_1  \overset{\ell_1}{\to} A_2 \overset{\ell_2}{\to} \ldots A_{k-1}\overset{\ell_k}{\to} \mathrm{U} 
    \]
    \pause
    В матричном виде: 
    \[
    \mathrm{U} = \LL_k \cdot \LL_{k-1} \cdot \ldots  \cdot  \LL_2 \cdot \LL_{1} \cdot \mathrm{P} \cdot A.\pause
    \]
    Отменим действия $\LL_i$.
    \[
        \LL_1^{-1} \cdot \LL_2^{-1} \cdot \ldots \cdot \LL_k^{-1} \cdot  \LL_k^{-1} \cdot \mathrm{U} =  \mathrm{P} \cdot A.\pause
    \]
    Гауссовские преобразования эквивалентны разложению
    \[
    \LL\cdot  \mathrm{U} = \mathrm{P} \cdot A.     
    \]

\end{frame}

\begin{frame}
    \frametitle{Зачем нужно $LU$-разложение?}

    Если $LU$-разложение матрицы $A$ получено, то можно очень быстро
    \begin{itemize}
        \item найти определитель матрицы $A$;
        \item решить любую систему $A\bx = \bb$.
    \end{itemize}

    \pause
    \[
    \begin{pmatrix}
        1 & 0 & 0 \\
        2 & 1 & 0 \\
        -3 & 5 & 1 \\
    \end{pmatrix} \cdot 
    \begin{pmatrix}
        -2 & 1 & 3 \\
        0 & -1 & 2 \\
        0 &  0 & 1 \\
    \end{pmatrix} = 
    \begin{pmatrix}
-2 & 1 & 3 \\
-4 & 1 & 8 \\
6 &  -8 & 2 \\        
    \end{pmatrix}
    \]

    \pause
    \[
    \det A = 1\cdot 1\cdot 1\cdot (-2) \cdot (-1) \cdot 1 = 2
    \]    

\end{frame}



\begin{frame}
    \frametitle{Резюме}

    \begin{itemize}[<+->]
    \item Определитель измеряет изменение площади и объёма.
    \item Свойства определителя.
    \item Обращение матрицы методом Гаусса.
    \item Обращение матрицы методом Крамера.
    \item Метод Гаусса как $LU$-разложение.
    \item Бонус: комплексные числа.
    \end{itemize}
    \pause
    \alert{Следующая лекция:} спектральное разложение и диагонализация.
        


\end{frame}