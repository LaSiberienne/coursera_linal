% !TEX root = ../linal_lecture_03.tex

\begin{frame} % название фрагмента

\videotitle{Идея определителя}

\end{frame}



\begin{frame}{Краткий план:}
  \begin{itemize}[<+->]
    \item Определитель на плоскости;
    \item Определитель в пространстве.
  \end{itemize}

\end{frame}




\begin{frame}
    \frametitle{Идея определителя}

    Рассмотрим оператор преобразования плоскости, $\LL: \R^2 \to \R^2$. 

    Пара векторов $\ba$, $\bb$ переходит в пару векторов $\LL\ba$, $\LL\bb$. 

    \pause

    Как меняется площадь параллелограмма образованного двумя векторами?

    \pause

    Меняется ли направление поворота от первого вектора ко второму?
\end{frame}




\begin{frame}
    \frametitle{Идея определителя на картинке}


    \begin{tikzpicture}[
    scale=1.6,
    MyPoints/.style={draw=blue,fill=white,thick},
    Segments/.style={draw=blue!50!red!70,thick},
    MyCircles/.style={green!50!blue!50,thin}, 
    every node/.style={scale=1}
    ]
    %\grid;
    \clip (-5.5,-2.5) rectangle (7.5,6.5);


    %\draw[->, >=stealth] (-1,0)--(6.5,0) node[right]{$x_1$};
    \draw[-{Latex[length=4.5mm, width=2.5mm]}, >=stealth] (0,-1)--(0,5) node[above left]{$x_2$};

    \draw[-{Latex[length=4.5mm, width=2.5mm]}, >=stealth] (-4,0)--(6.5,0) 
    node[right]{$x_1$};


    %{\verb!->!new, arrowhead = 2mm, line width=4pt}
    %, arrowhead = 3mm
    %, arrowhead = 0.2

    % Feel free to change here coordinates of points A and B
    \pgfmathparse{0}		\let\Xa\pgfmathresult
    \pgfmathparse{0}		\let\Ya\pgfmathresult
    \coordinate (A) at (\Xa,\Ya);

    \pgfmathparse{4}		\let\Xb\pgfmathresult
    \pgfmathparse{3}		\let\Yb\pgfmathresult
    \coordinate (B) at (\Xb,\Yb);

    \pgfmathparse{4}		\let\Xc\pgfmathresult
    \pgfmathparse{0}		\let\Yc\pgfmathresult
    \coordinate (C) at (\Xc,\Yc);




    %\node [above right,darkgray] at (1,3.5) {$\det \operatorname{L} = \frac{S(\LL\ba, \LL\bb)}{S(\ba, \bb)} $};



    \pgfmathparse{4}		\let\Xg\pgfmathresult
    \pgfmathparse{0}		\let\Yg\pgfmathresult
    \coordinate (G) at (\Xg,\Yg);

    \pgfmathparse{4}		\let\Xh\pgfmathresult
    \pgfmathparse{0}		\let\Yh\pgfmathresult
    \coordinate (H) at (\Xh,\Yh);

    \pgfmathparse{4}		\let\Xi\pgfmathresult
    \pgfmathparse{0}		\let\Yi\pgfmathresult
    \coordinate (I) at (\Xi,\Yi);



    \begin{scope}[cm={1,1,1.5,0.5,(0,0)}]
    \draw[pattern=north west lines, pattern color=blue!50, draw=none ] (0,0) rectangle (2,2);
    \draw[-{Latex[length=4.5mm, width=2.5mm]}, >=stealth, vecb,thick] (0,0)--(0,2) node[below]{$\ba$};
    \draw[-{Latex[length=4.5mm, width=2.5mm]}, >=stealth, vecb,thick] (0,0)--(2,0) node[above]{$\bb$};
    \end{scope};


    \pgfmathparse{3}		\let\Xd\pgfmathresult
    \pgfmathparse{1}		\let\Yd\pgfmathresult
    \coordinate (D) at (\Xd,\Yd);

    \pgfmathparse{0}		\let\Xe\pgfmathresult
    \pgfmathparse{0}		\let\Ye\pgfmathresult
    \coordinate (E) at (\Xe,\Ye);

    \pgfmathparse{2}		\let\Xf\pgfmathresult
    \pgfmathparse{2}		\let\Yf\pgfmathresult
    \coordinate (F) at (\Xf,\Yf);


    \tkzMarkAngle[size=1, mark = none, arrows=->,line width=1pt, mkcolor=blue ](D,E,F);


    \begin{scope}[cm={-1,2,-1.5,0.5,(0,0)}]
    \draw[pattern=north west lines,pattern color=black!50, draw=none ] (0,0) rectangle (0.9,2);
    \draw[-{Latex[length=4.5mm, width=2.5mm]}, >=stealth, thick] (0,0)--(0,2) node[below]{$\operatorname{L} \ba$};
    \draw[-{Latex[length=4.5mm, width=2.5mm]}, >=stealth, thick] (0,0)--(0.9,0) node[above]{$\operatorname{L} \bb$};

    \end{scope}



    \pgfmathparse{-1}		\let\Xg\pgfmathresult
    \pgfmathparse{2}		\let\Yg\pgfmathresult
    \coordinate (G) at (\Xg,\Yg);

    \pgfmathparse{-3}		\let\Xi\pgfmathresult
    \pgfmathparse{1}		\let\Yi\pgfmathresult
    \coordinate (I) at (\Xi,\Yi);

    \tkzMarkAngle[size=1, mark = none, arrows=<-,line width=1pt, mkcolor=blue ](G,E,I);

    \node [right,darkgray] at (-2.5,1.5) {$\operatorname{LF}$ }; 

    \node [right,darkgray] at (2,1.5) {$\operatorname{F}$ }; 


    %\tkzMarkAngle[size=1, mark = none, arrows=->,line width=1.5pt, mkcolor=red ](B,A,E);



    \end{tikzpicture}

        
%    \[
%    \det \LL  = \frac{S(\LL\ba, \LL\bb)}{S(\ba, \bb)}    
%    \]
    

\end{frame}




\begin{frame}
    \frametitle{Ориентированная площадь}


    \begin{block}{Определение}
        Возьмём площадь параллелограмма со сторонами $\ba$ и $\bb$.
        Если поворот от первого вектора ко второму идёт по часовой стрелке, то дополнительно домножим площадь на $(-1)$.

        Полученное число назовём \alert{ориентированной площадью} параллелограмма и обозначим $S(\ba, \bb)$.    
    \end{block}

    \pause

    Важен порядок векторов: 
    \[
        S(\ba, \bb) = - S(\bb, \ba).  
    \]



    
\end{frame}



\begin{frame}
    \frametitle{Идея определителя}

\begin{block}{Определение}
    Возьмём любые два вектора $\ba$ и $\bb$ c $S(\ba, \bb)\neq 0$.

    \alert{Определитель} оператора $\LL:\R^2 \to \R^2$ показывает во сколько раз изменяется
    ориентированная площадь
    \[
    \det \LL = \frac{S(\LL\ba, \LL\ba)}{S(\ba, \bb)}    
    \]
\end{block}    
    

\end{frame}




\begin{frame}
    \frametitle{Определитель симметрии}


    Рассмотрим оператор $\LL : \begin{pmatrix}
      a_1 \\
      a_2 \\
    \end{pmatrix} \to 
    \begin{pmatrix}
      a_2 \\
      a_1 \\
    \end{pmatrix}$.
    

    \pause

    Площадь параллелограмма не изменяется. 


    Меняется направление поворота от первого вектора ко второму. 

    \pause

    \[
    \det \LL = \frac{S(\LL \ba, \LL \bb )}{S(\ba, \bb )} = -1
    \]

\end{frame}






\begin{frame}
    \frametitle{Определитель растягивания компонент}


    Рассмотрим оператор $\LL : \begin{pmatrix}
      a_1 \\
      a_2 \\
    \end{pmatrix} \to 
    \begin{pmatrix}
      2a_1 \\
      -3a_2 \\
    \end{pmatrix}$.
    

    \pause

    Одна сторона растягивается в два раза, вторая — в три раза.


    Меняется направление поворота от первого вектора ко второму. 

    \pause

    \[
    \det \LL = \frac{S(\LL \ba, \LL \bb )}{S(\ba, \bb )} = (-1)\cdot 2\cdot 3 = -6
    \]

\end{frame}




\begin{frame}
    \frametitle{Определитель поворота}


    Оператор $\Rot: \R^2 \to \R^2$ вращает плоскость на $30^{\circ}$ против часовой стрелки.

    \pause

    При вращении не изменяется площадь параллелограмма.

    При вращении не изменяется направление поворота от первого вектора ко второму.

    \pause

    \[
    \det \Rot = \frac{\Rot(\LL \ba, \LL \bb )}{\Rot(\ba, \bb )} = 1     
    \]

\end{frame}


\begin{frame}
    \frametitle{Определитель проекции}


    Оператор $\HH: \R^2 \to \R^2$ проецирует векторы на прямую $\ell$.  

    \pause

    При проекции любой параллелограмм «складывается» в отрезок нулевой площади.

    \pause

    \[
    \det \HH = \frac{S(\HH \ba, \HH \bb )}{S(\ba, \bb )} = 0     
    \]

\end{frame}



\begin{frame}
    \frametitle{Чем прекрасна ориентированная площадь?}

    \begin{block}{Утверждение}
        Ориентированная площадь $S(\ba, \bb)$ линейна по каждому аргументу:
        \[
        S(\lambda \ba, \bb) = \lambda S(\ba, \bb), \quad S(\ba + \bb, \bc) = S(\ba, \bc) + S(\bb, \bc)    
        \]
    \end{block}

    \pause

    здесь картинка.

    

\end{frame}



\begin{frame}
    \frametitle{Корректность идеи определителя}

    Величина $\det \LL = \frac{S(\LL\ba, \LL\bb)}{S(\ba, \bb)}$ не зависит от выбора $\ba$ и $\bb$!


    \pause
    \begin{block}{Идея доказательства}
        Обозначим $\be_1 = \begin{pmatrix}
            1 \\
            0 \\
        \end{pmatrix}$, $\be_2 = \begin{pmatrix}
            0 \\
            1 \\
        \end{pmatrix}$.
        \pause

        Возьмём $\ba = 5\be_1 + 7\be_2$. Найдём $S(\LL\ba, \LL\be_2)/{S(\ba, \be_2)}$:
        \pause
        \[
        \frac{S(\LL (5\be_1 + 7\be_2), \LL \be_2)}{S(5\be_1 + 7\be_2,\be_2)} =
         \frac{S(\LL 5\be_1 , \LL \be_2) + S(\LL 7\be_2 , \LL \be_2)}{S(5\be_1,\be_2) + S(7\be_2,\be_2)} =
        \]
        \pause
        \[
         =\frac{5S(\LL \be_1 , \LL \be_2) + 0}{5S(\be_1,\be_2) + 0} =
         \frac{S(\LL \be_1 , \LL \be_2)}{S(\be_1,\be_2)}
        \]
    \end{block}

    

\end{frame}



\begin{frame}
    \frametitle{Ещё один взгляд на определитель}
    
    Обозначим $\be_1 = \begin{pmatrix}
        1 \\
        0 \\
    \end{pmatrix}$, $\be_2 = \begin{pmatrix}
        0 \\
        1 \\
    \end{pmatrix}$.

    \pause

    \begin{block}{Определение}
        Преобразуем параллелограмм, образованный векторами $\be_1$ и $\be_2$, с помощью оператора $\LL$. 
        
        Определитель линейного оператора $\LL:\R^2 \to \R^2$ равен ориентированной 
        площади полученного параллелограмма.
        \[
        \det \LL = S(\LL\be_1, \LL\be_2)    
        \]
        
    \end{block}
    
\end{frame}


\begin{frame}
    \frametitle{Определитель в пространстве}

    Идея: заменим ориентированную площадь параллелограмма $S(\ba, \bb)$ 
    на ориентированный объём параллелепипеда $S(\ba, \bb, \bc)$.
    
    \pause

\begin{block}{Определение}
    Возьмём любые три вектора $\ba$, $\bb$ и $\bc$, для которых $S(\ba, \bb, \bc)\neq 0$.

    \alert{Определитель} оператора $\LL:\R^3 \to \R^3$ показывает во сколько раз изменяется
    ориентированный объём
    \[
    \det \LL = \frac{S(\LL\ba, \LL\ba, \LL\bc)}{S(\ba, \bb, \bc)}    
    \]
\end{block} 


\end{frame}



\begin{frame}
    \frametitle{А что такое ориентированный объём?}

\pause 

Обозначим $\be_1 = \begin{pmatrix}
    1 \\
    0 \\
    0 \\
\end{pmatrix}$, $\be_2 = \begin{pmatrix}
    0 \\
    1 \\
    0
\end{pmatrix}$, $\be_3 = \begin{pmatrix}
    0 \\
    0 \\
    1 \\
\end{pmatrix}$.

% Скажем, что $S(\be_1, \be_2, \be_3) = 1$.

% Сопоставим большой, указательный и средний палец этим векторам.

\pause

\begin{block}{Определение}
    Рассмотрим параллелепипед, образованный $\ba$, $\bb$ и $\bc$.

    \pause

    С помощью поворота: 

    Cовместим вектор $\be_1$ с вектором $\ba$;

    Затем вектор $\be_2$ «положим» в плоскость $\ba$, $\bb$.

    \pause

    \alert{Ориентированный объём} $S(\ba, \bb, \bc)$ объявим отрицательным,
    если векторы $\be_3$ и $\bc$ смотрят в разные полупространства.
    
\end{block}



    

\end{frame}


\begin{frame}
    \frametitle{Определитель растягивания компонент}


    Рассмотрим оператор $\LL : \begin{pmatrix}
      a_1 \\
      a_2 \\
      a_3 \\
    \end{pmatrix} \to 
    \begin{pmatrix}
      2a_1 \\
      3a_2 \\
      -5a_3 \\
    \end{pmatrix}$.
    

    \pause

    Одна сторона растягивается в два раза, вторая — в три раза, третья — в пять.


    Первые два вектора не изменяют направления при преобразовании.

    
    Третий вектор меняет полупространство, в котором он лежит относительно первых двух. 

    \pause

    \[
    \det \LL = \frac{S(\LL \ba, \LL \bb, \LL \bc )}{S(\ba, \bb, \bc )} = (-1)\cdot 2\cdot 3\cdot 5  = - 30
    \]

\end{frame}



\begin{frame}
    \frametitle{Определитель проекции}


    Оператор $\HH : \R^3 \to \R^3$ проецирует векторы на плоскость $\alpha$. 

    \pause

    Любой параллелепипед «схлопывается» в плоскую фигуру нулевого объёма.

    \pause

    \[
    \det \HH = \frac{S(\HH \ba, \HH \bb, \HH \bc )}{S(\ba, \bb, \bc )} = 0
    \]

\end{frame}


