% !TEX root = ../linal_lecture_03.tex

\begin{frame} % название фрагмента

\videotitle{Идея определителя}

\end{frame}



\begin{frame}{Краткий план:}
  \begin{itemize}[<+->]
    \item Определитель на плоскости;
    \item Определитель в пространстве.
  \end{itemize}

\end{frame}




\begin{frame}
    \frametitle{Идея определителя}

    Рассмотрим оператор преобразования плоскости, $\LL: \R^2 \to \R^2$. 

    Пара векторов $\ba$, $\bb$ переходит в пару векторов $\LL\ba$, $\LL\bb$. 

    \pause

    Как меняется площадь параллелограмма образованного двумя векторами?

    \pause

    Меняется ли направление поворота от первого вектора ко второму?
\end{frame}




\begin{frame}
    \frametitle{Идея определителя на картинке}


    \begin{center}
    \begin{tikzpicture}[
    scale=1.6,
    MyPoints/.style={draw=blue,fill=white,thick},
    Segments/.style={draw=blue!50!red!70,thick},
    MyCircles/.style={green!50!blue!50,thin}, 
    every node/.style={scale=1}
    ]
    %\grid;
    \clip (-5.5,-2.5) rectangle (7.5,6.5);


    %\draw[->, >=stealth] (-1,0)--(6.5,0) node[right]{$x_1$};
    \draw[-{Latex[length=4.5mm, width=2.5mm]}, >=stealth] (0,-1)--(0,5) node[above left]{$x_2$};

    \draw[-{Latex[length=4.5mm, width=2.5mm]}, >=stealth] (-4,0)--(6.5,0) 
    node[right]{$x_1$};


    %{\verb!->!new, arrowhead = 2mm, line width=4pt}
    %, arrowhead = 3mm
    %, arrowhead = 0.2

    % Feel free to change here coordinates of points A and B
    \pgfmathparse{0}		\let\Xa\pgfmathresult
    \pgfmathparse{0}		\let\Ya\pgfmathresult
    \coordinate (A) at (\Xa,\Ya);

    \pgfmathparse{4}		\let\Xb\pgfmathresult
    \pgfmathparse{3}		\let\Yb\pgfmathresult
    \coordinate (B) at (\Xb,\Yb);

    \pgfmathparse{4}		\let\Xc\pgfmathresult
    \pgfmathparse{0}		\let\Yc\pgfmathresult
    \coordinate (C) at (\Xc,\Yc);




    %\node [above right,darkgray] at (1,3.5) {$\det \operatorname{L} = \frac{S(\LL\ba, \LL\bb)}{S(\ba, \bb)} $};



    \pgfmathparse{4}		\let\Xg\pgfmathresult
    \pgfmathparse{0}		\let\Yg\pgfmathresult
    \coordinate (G) at (\Xg,\Yg);

    \pgfmathparse{4}		\let\Xh\pgfmathresult
    \pgfmathparse{0}		\let\Yh\pgfmathresult
    \coordinate (H) at (\Xh,\Yh);

    \pgfmathparse{4}		\let\Xi\pgfmathresult
    \pgfmathparse{0}		\let\Yi\pgfmathresult
    \coordinate (I) at (\Xi,\Yi);



    \begin{scope}[cm={1,1,1.5,0.5,(0,0)}]
    \draw[pattern=north west lines, pattern color=blue!50, draw=none ] (0,0) rectangle (2,2);
    \draw[-{Latex[length=4.5mm, width=2.5mm]}, >=stealth, vecb,thick] (0,0)--(0,2) node[below]{$\ba$};
    \draw[-{Latex[length=4.5mm, width=2.5mm]}, >=stealth, vecb,thick] (0,0)--(2,0) node[above]{$\bb$};
    \end{scope};


    \pgfmathparse{3}		\let\Xd\pgfmathresult
    \pgfmathparse{1}		\let\Yd\pgfmathresult
    \coordinate (D) at (\Xd,\Yd);

    \pgfmathparse{0}		\let\Xe\pgfmathresult
    \pgfmathparse{0}		\let\Ye\pgfmathresult
    \coordinate (E) at (\Xe,\Ye);

    \pgfmathparse{2}		\let\Xf\pgfmathresult
    \pgfmathparse{2}		\let\Yf\pgfmathresult
    \coordinate (F) at (\Xf,\Yf);


    \tkzMarkAngle[size=1, mark = none, arrows=->,line width=1pt, mkcolor=blue ](D,E,F);


    \begin{scope}[cm={-1,2,-1.5,0.5,(0,0)}]
    \draw[pattern=north west lines,pattern color=black!50, draw=none ] (0,0) rectangle (0.9,2);
    \draw[-{Latex[length=4.5mm, width=2.5mm]}, >=stealth, thick] (0,0)--(0,2) node[below]{$\operatorname{L} \ba$};
    \draw[-{Latex[length=4.5mm, width=2.5mm]}, >=stealth, thick] (0,0)--(0.9,0) node[above]{$\operatorname{L} \bb$};

    \end{scope}



    \pgfmathparse{-1}		\let\Xg\pgfmathresult
    \pgfmathparse{2}		\let\Yg\pgfmathresult
    \coordinate (G) at (\Xg,\Yg);

    \pgfmathparse{-3}		\let\Xi\pgfmathresult
    \pgfmathparse{1}		\let\Yi\pgfmathresult
    \coordinate (I) at (\Xi,\Yi);

    \tkzMarkAngle[size=1, mark = none, arrows=<-,line width=1pt, mkcolor=blue ](G,E,I);

%    \node [right,darkgray] at (-2.5,1.5) {$\operatorname{LF}$ }; 

 %   \node [right,darkgray] at (2,1.5) {$\operatorname{F}$ }; 


    %\tkzMarkAngle[size=1, mark = none, arrows=->,line width=1.5pt, mkcolor=red ](B,A,E);



    \end{tikzpicture}
\end{center}
        
%    \[
%    \det \LL  = \frac{S(\LL\ba, \LL\bb)}{S(\ba, \bb)}    
%    \]
    

\end{frame}




\begin{frame}
    \frametitle{Ориентированная площадь}


    \begin{block}{Определение}
        Возьмём площадь параллелограмма со сторонами $\ba$ и $\bb$.
        Если поворот от первого вектора ко второму идёт по часовой стрелке, то дополнительно домножим площадь на $(-1)$.

        Полученное число назовём \alert{ориентированной площадью} параллелограмма и обозначим $S(\ba, \bb)$.    
    \end{block}

    \pause

    Важен порядок векторов: 
    \[
        S(\ba, \bb) = - S(\bb, \ba).  
    \]
    
\end{frame}

\begin{frame}
    \frametitle{Ориентированная площадь}

    \begin{minipage}[H]{0.48\linewidth}


        \begin{tikzpicture}[
        scale=1.75,
        MyPoints/.style={draw=blue,fill=white,thick},
        Segments/.style={draw=blue!50!red!70,thick},
        MyCircles/.style={green!50!blue!50,thin}, 
        every node/.style={scale=0.75}
        ]
        %\grid;
        \clip (-1,-2.5) rectangle (5, 3.5);


        %\draw[->, >=stealth] (-1,0)--(6.5,0) node[right]{$x_1$};
        \draw[-{Latex[length=4.5mm, width=2.5mm]}, >=stealth] (0,-1)--(0,3) node[above left]{$x_2$};

        \draw[-{Latex[length=4.5mm, width=2.5mm]}, >=stealth] (-1,0)--(4.5,0) 
        node[right]{$x_1$};


        %{\verb!->!new, arrowhead = 2mm, line width=4pt}
        %, arrowhead = 3mm
        %, arrowhead = 0.2

        % Feel free to change here coordinates of points A and B
        \pgfmathparse{0}		\let\Xa\pgfmathresult
        \pgfmathparse{0}		\let\Ya\pgfmathresult
        \coordinate (A) at (\Xa,\Ya);

        \pgfmathparse{4}		\let\Xb\pgfmathresult
        \pgfmathparse{3}		\let\Yb\pgfmathresult
        \coordinate (B) at (\Xb,\Yb);

        \pgfmathparse{4}		\let\Xc\pgfmathresult
        \pgfmathparse{0}		\let\Yc\pgfmathresult
        \coordinate (C) at (\Xc,\Yc);




        \pgfmathparse{4}		\let\Xg\pgfmathresult
        \pgfmathparse{0}		\let\Yg\pgfmathresult
        \coordinate (G) at (\Xg,\Yg);

        \pgfmathparse{4}		\let\Xh\pgfmathresult
        \pgfmathparse{0}		\let\Yh\pgfmathresult
        \coordinate (H) at (\Xh,\Yh);

        \pgfmathparse{4}		\let\Xi\pgfmathresult
        \pgfmathparse{0}		\let\Yi\pgfmathresult
        \coordinate (I) at (\Xi,\Yi);



        \begin{scope}[cm={1,1,1.5,0.5,(0,0)}]
        \draw[pattern=north west lines, pattern color=blue!50, draw=none ] (0,0) rectangle (2,2);
        \draw[-{Latex[length=4.5mm, width=2.5mm]}, >=stealth, vecb,thick] (0,0)--(0,2) node[below]{$\ba$};
        \draw[-{Latex[length=4.5mm, width=2.5mm]}, >=stealth, vecb,thick] (0,0)--(2,0) node[above]{$\bb$};
        \end{scope};


        \pgfmathparse{3}		\let\Xd\pgfmathresult
        \pgfmathparse{1}		\let\Yd\pgfmathresult
        \coordinate (D) at (\Xd,\Yd);

        \pgfmathparse{0}		\let\Xe\pgfmathresult
        \pgfmathparse{0}		\let\Ye\pgfmathresult
        \coordinate (E) at (\Xe,\Ye);

        \pgfmathparse{2}		\let\Xf\pgfmathresult
        \pgfmathparse{2}		\let\Yf\pgfmathresult
        \coordinate (F) at (\Xf,\Yf);


        \tkzMarkAngle[size=1, mark = none, arrows=->,line width=1pt, mkcolor=blue ](D,E,F);





        \pgfmathparse{-1}		\let\Xg\pgfmathresult
        \pgfmathparse{2}		\let\Yg\pgfmathresult
        \coordinate (G) at (\Xg,\Yg);

        \pgfmathparse{-3}		\let\Xi\pgfmathresult
        \pgfmathparse{1}		\let\Yi\pgfmathresult
        \coordinate (I) at (\Xi,\Yi);


        \node [above right,darkgray] at (1,-1) {$S(\ba,\bb)>0$};



        %\tkzMarkAngle[size=1, mark = none, arrows=->,line width=1.5pt, mkcolor=red ](B,A,E);



        \end{tikzpicture}




    \end{minipage}
    \begin{minipage}[H]{0.48\linewidth}


        \begin{tikzpicture}[
        scale=1.75,
        MyPoints/.style={draw=blue,fill=white,thick},
        Segments/.style={draw=blue!50!red!70,thick},
        MyCircles/.style={green!50!blue!50,thin}, 
        every node/.style={scale=0.75}
        ]
        %\grid;
        \clip (-1,-2.5) rectangle (5, 3.5);


        %\draw[->, >=stealth] (-1,0)--(6.5,0) node[right]{$x_1$};
        \draw[-{Latex[length=4.5mm, width=2.5mm]}, >=stealth] (0,-1)--(0,3) node[above left]{$x_2$};

        \draw[-{Latex[length=4.5mm, width=2.5mm]}, >=stealth] (-1,0)--(4.5,0) 
        node[right]{$x_1$};


        %{\verb!->!new, arrowhead = 2mm, line width=4pt}
        %, arrowhead = 3mm
        %, arrowhead = 0.2

        % Feel free to change here coordinates of points A and B
        \pgfmathparse{0}		\let\Xa\pgfmathresult
        \pgfmathparse{0}		\let\Ya\pgfmathresult
        \coordinate (A) at (\Xa,\Ya);

        \pgfmathparse{4}		\let\Xb\pgfmathresult
        \pgfmathparse{3}		\let\Yb\pgfmathresult
        \coordinate (B) at (\Xb,\Yb);

        \pgfmathparse{4}		\let\Xc\pgfmathresult
        \pgfmathparse{0}		\let\Yc\pgfmathresult
        \coordinate (C) at (\Xc,\Yc);




        \pgfmathparse{4}		\let\Xg\pgfmathresult
        \pgfmathparse{0}		\let\Yg\pgfmathresult
        \coordinate (G) at (\Xg,\Yg);

        \pgfmathparse{4}		\let\Xh\pgfmathresult
        \pgfmathparse{0}		\let\Yh\pgfmathresult
        \coordinate (H) at (\Xh,\Yh);

        \pgfmathparse{4}		\let\Xi\pgfmathresult
        \pgfmathparse{0}		\let\Yi\pgfmathresult
        \coordinate (I) at (\Xi,\Yi);



        \begin{scope}[cm={1,1,1.5,0.5,(0,0)}]
        \draw[pattern=north west lines, pattern color=blue!50, draw=none ] (0,0) rectangle (2,2);
        \draw[-{Latex[length=4.5mm, width=2.5mm]}, >=stealth, vecb,thick] (0,0)--(0,2) node[below]{$\ba$};
        \draw[-{Latex[length=4.5mm, width=2.5mm]}, >=stealth, vecb,thick] (0,0)--(2,0) node[above]{$\bb$};
        \end{scope};


        \pgfmathparse{3}		\let\Xd\pgfmathresult
        \pgfmathparse{1}		\let\Yd\pgfmathresult
        \coordinate (D) at (\Xd,\Yd);

        \pgfmathparse{0}		\let\Xe\pgfmathresult
        \pgfmathparse{0}		\let\Ye\pgfmathresult
        \coordinate (E) at (\Xe,\Ye);

        \pgfmathparse{2}		\let\Xf\pgfmathresult
        \pgfmathparse{2}		\let\Yf\pgfmathresult
        \coordinate (F) at (\Xf,\Yf);


        \tkzMarkAngle[size=1, mark = none, arrows=<-,line width=1pt, mkcolor=blue ](D,E,F);





        \pgfmathparse{-1}		\let\Xg\pgfmathresult
        \pgfmathparse{2}		\let\Yg\pgfmathresult
        \coordinate (G) at (\Xg,\Yg);

        \pgfmathparse{-3}		\let\Xi\pgfmathresult
        \pgfmathparse{1}		\let\Yi\pgfmathresult
        \coordinate (I) at (\Xi,\Yi);


        \node [above right,darkgray] at (1,-1) {$S(\bb,\ba)<0$};



        %\tkzMarkAngle[size=1, mark = none, arrows=->,line width=1.5pt, mkcolor=red ](B,A,E);



        \end{tikzpicture}

    \end{minipage}
    

    

\end{frame}



\begin{frame}
    \frametitle{Идея определителя}

\begin{block}{Определение}
    Возьмём любые два вектора $\ba$ и $\bb$, для которых $S(\ba, \bb)\neq 0$.

    \alert{Определитель} оператора $\LL:\R^2 \to \R^2$ показывает во сколько раз изменяется
    ориентированная площадь
    \[
    \det \LL = \frac{S(\LL\ba, \LL\ba)}{S(\ba, \bb)}    
    \]
\end{block}    
    

\end{frame}






\begin{frame}
\frametitle{Определитель отражения}

Оператор $\LL : \begin{pmatrix}
  a_1 \\
  a_2 \\
\end{pmatrix} \to 
\begin{pmatrix}
  a_2 \\
  a_1 \\
\end{pmatrix}$ отражает относительно $x_1= x_2$.\pause




\begin{center}


\begin{tikzpicture}[
scale=1.8,
MyPoints/.style={draw=blue,fill=white,thick},
Segments/.style={draw=blue!50!red!70,thick},
MyCircles/.style={green!50!blue!50,thin}, 
every node/.style={scale=1}
]
%\grid;
\clip (-1,-2.5) rectangle (6.5, 6.5);


%\draw[->, >=stealth] (-1,0)--(6.5,0) node[right]{$x_1$};
\draw[-{Latex[length=4.5mm, width=2.5mm]}, >=stealth] (0,-1)--(0,6) node[above left]{$x_2$};

\draw[-{Latex[length=4.5mm, width=2.5mm]}, >=stealth] (-1,0)--(6,0) 
node[right]{$x_1$};


%{\verb!->!new, arrowhead = 2mm, line width=4pt}
%, arrowhead = 3mm
%, arrowhead = 0.2

% Feel free to change here coordinates of points A and B
\pgfmathparse{0}		\let\Xa\pgfmathresult
\pgfmathparse{0}		\let\Ya\pgfmathresult
\coordinate (A) at (\Xa,\Ya);

\pgfmathparse{4}		\let\Xb\pgfmathresult
\pgfmathparse{3}		\let\Yb\pgfmathresult
\coordinate (B) at (\Xb,\Yb);

\pgfmathparse{4}		\let\Xc\pgfmathresult
\pgfmathparse{0}		\let\Yc\pgfmathresult
\coordinate (C) at (\Xc,\Yc);




\pgfmathparse{4}		\let\Xg\pgfmathresult
\pgfmathparse{0}		\let\Yg\pgfmathresult
\coordinate (G) at (\Xg,\Yg);

\pgfmathparse{4}		\let\Xh\pgfmathresult
\pgfmathparse{0}		\let\Yh\pgfmathresult
\coordinate (H) at (\Xh,\Yh);

\pgfmathparse{4}		\let\Xi\pgfmathresult
\pgfmathparse{0}		\let\Yi\pgfmathresult
\coordinate (I) at (\Xi,\Yi);



\begin{scope}[cm={1.25,0.75,1.75,0.25,(0,0)}]
\draw[pattern=north west lines, pattern color=blue!50, draw=none ] (0,0) rectangle (2,2);
\draw[-{Latex[length=4.5mm, width=2.5mm]}, >=stealth, vecb,thick] (0,0)--(0,2) node[below]{$\ba$};
\draw[-{Latex[length=4.5mm, width=2.5mm]}, >=stealth, vecb,thick] (0,0)--(2,0) node[above]{$\bb$};
\end{scope};

\begin{scope}[cm={0.75,1.25,0.25,1.75,(0,0)}]
\draw[pattern=north west lines, pattern color=blue!50, draw=none ] (0,0) rectangle (2,2);
\draw[-{Latex[length=4.5mm, width=2.5mm]}, >=stealth, vecb,thick] (0,0)--(0,2) node[above]{$\operatorname{L}\ba$};
\draw[-{Latex[length=4.5mm, width=2.5mm]}, >=stealth, vecb,thick] (0,0)--(2,0) node[right]{$\operatorname{L}\bb$};
\end{scope};


\pgfmathparse{0.75}		\let\Xd\pgfmathresult
\pgfmathparse{1.25}		\let\Yd\pgfmathresult
\coordinate (D) at (\Xd,\Yd);

\pgfmathparse{0}		\let\Xe\pgfmathresult
\pgfmathparse{0}		\let\Ye\pgfmathresult
\coordinate (E) at (\Xe,\Ye);

\pgfmathparse{0.25}		\let\Xf\pgfmathresult
\pgfmathparse{1.75}		\let\Yf\pgfmathresult
\coordinate (F) at (\Xf,\Yf);


\tkzMarkAngle[size=1, mark = none, arrows=<-,line width=1pt, mkcolor=blue ](D,E,F);



\pgfmathparse{1.25}		\let\Xd\pgfmathresult
\pgfmathparse{0.75}		\let\Yd\pgfmathresult
\coordinate (D) at (\Xd,\Yd);

\pgfmathparse{0}		\let\Xe\pgfmathresult
\pgfmathparse{0}		\let\Ye\pgfmathresult
\coordinate (E) at (\Xe,\Ye);

\pgfmathparse{1.75}		\let\Xf\pgfmathresult
\pgfmathparse{0.25}		\let\Yf\pgfmathresult
\coordinate (F) at (\Xf,\Yf);


\tkzMarkAngle[size=1, mark = none, arrows=->,line width=1pt, mkcolor=blue ](F,E,D);




\pgfmathparse{-1}		\let\Xg\pgfmathresult
\pgfmathparse{-1}		\let\Yg\pgfmathresult
\coordinate (G) at (\Xg,\Yg);

\pgfmathparse{6}		\let\Xi\pgfmathresult
\pgfmathparse{6}		\let\Yi\pgfmathresult
\coordinate (I) at (\Xi,\Yi);



\draw[dashed] (G)--(I) node[near end, below right]{$x_1=x_2$};







%\tkzMarkAngle[size=1, mark = none, arrows=->,line width=1.5pt, mkcolor=red ](B,A,E);



\end{tikzpicture}



    \end{center}

\end{frame}


\begin{frame}
\frametitle{Определитель отражения}
    

Оператор $\LL : \begin{pmatrix}
  a_1 \\
  a_2 \\
\end{pmatrix} \to 
\begin{pmatrix}
  a_2 \\
  a_1 \\
\end{pmatrix}$ отражает относительно $x_1= x_2$.


\pause



    Площадь параллелограмма не изменяется. 


    Меняется направление поворота от первого вектора ко второму. 

    \pause

    \[
    \det \LL = \frac{S(\LL \ba, \LL \bb )}{S(\ba, \bb )} = -1
    \]

\end{frame}






\begin{frame}
    \frametitle{Определитель растягивания компонент}


    Рассмотрим оператор $\LL : \begin{pmatrix}
      a_1 \\
      a_2 \\
    \end{pmatrix} \to 
    \begin{pmatrix}
      2a_1 \\
      -3a_2 \\
    \end{pmatrix}$.
    

    \pause

    Первый базисный вектор $\be_1$ растягивается в два раза.
    
    Второй базисный вектор $\be_2$ растягивается в три раза и отражается.


    Меняется направление поворота от первого вектора ко второму. 

    \pause

    \[
    \det \LL = \frac{S(\LL \ba, \LL \bb )}{S(\ba, \bb )} = (-1)\cdot 2\cdot 3 = -6
    \]

\end{frame}




\begin{frame}
    \frametitle{Определитель поворота}


    Оператор $\Rot: \R^2 \to \R^2$ вращает плоскость.

    \pause

  \begin{center}


      \begin{tikzpicture}[
      scale=1.5,
      MyPoints/.style={draw=blue,fill=white,thick},
      Segments/.style={draw=blue!50!red!70,thick},
      MyCircles/.style={green!50!blue!50,thin}, 
      every node/.style={scale=1}
      ]
      %\grid;
      \clip (-7,-2.5) rectangle (6.5, 5.5);



      %{\verb!->!new, arrowhead = 2mm, line width=4pt}
      %, arrowhead = 3mm
      %, arrowhead = 0.2

      % Feel free to change here coordinates of points A and B
      \pgfmathparse{0}		\let\Xa\pgfmathresult
      \pgfmathparse{0}		\let\Ya\pgfmathresult
      \coordinate (A) at (\Xa,\Ya);




      \pgfmathparse{2}		\let\Xb\pgfmathresult
      \pgfmathparse{0.5}		\let\Yb\pgfmathresult
      \coordinate (B) at (\Xb,\Yb);

      \pgfmathparse{0.5}		\let\Xd\pgfmathresult
      \pgfmathparse{1.5}		\let\Yd\pgfmathresult
      \coordinate (D) at (\Xd,\Yd);

      \pgfmathparse{130}		\let\angle\pgfmathresult;
      \pgfmathparse{sqrt(10)}		\let\rad\pgfmathresult;


      \pgfmathparse{\Xb*cos(\angle)  - \Yb*sin(\angle)}		\let\Xe\pgfmathresult
      \pgfmathparse{\Xb*sin(\angle)  + \Yb*cos(\angle)}		\let\Ye\pgfmathresult
      \coordinate (E) at (\Xe,\Ye);

      \pgfmathparse{\Xd*cos(\angle)  - \Yd*sin(\angle)}		\let\Xf\pgfmathresult
      \pgfmathparse{\Xd*sin(\angle)  + \Yd*cos(\angle)}		\let\Yf\pgfmathresult
      \coordinate (F) at (\Xf,\Yf);

      \tkzMarkAngle[size=1, mark = none, arrows=->,line width=1pt, mkcolor=blue ](D,A,F);


      \begin{scope}[cm={\Xb,\Yb,\Xd,\Yd,(0,0)}]
      \draw[pattern=north west lines, pattern color=blue!50, draw=none ] (0,0) rectangle (2,2);
      \draw[-{Latex[length=4.5mm, width=2.5mm]}, >=stealth, vecb,thick] (0,0)--(0,2) node[above left]{$\bb$};
      \draw[-{Latex[length=4.5mm, width=2.5mm]}, >=stealth, vecb,thick] (0,0)--(2,0) node[below right]{$\ba$};
      \end{scope};

      \begin{scope}[cm={\Xe,\Ye,\Xf,\Yf,(0,0)}]
      \draw[pattern=north west lines, pattern color=blue!50, draw=none ] (0,0) rectangle (2,2);
      \draw[-{Latex[length=4.5mm, width=2.5mm]}, >=stealth, vecb,thick] (0,0)--(0,2) node[below right]{$\operatorname{R}\bb$};
      \draw[-{Latex[length=4.5mm, width=2.5mm]}, >=stealth, vecb,thick] (0,0)--(2,0) node[above right]{$\operatorname{R}\ba$};
      \end{scope};














      %\tkzMarkAngle[size=1, mark = none, arrows=->,line width=1.5pt, mkcolor=red ](B,A,E);



      \end{tikzpicture}


  \end{center}


\end{frame}



\begin{frame}
    \frametitle{Определитель поворота}


    Оператор $\Rot: \R^2 \to \R^2$ вращает плоскость.

    \pause

    При вращении не изменяется площадь параллелограмма.

    При вращении не изменяется направление поворота от первого вектора ко второму.

    \pause

    \[
    \det \Rot = \frac{S(\Rot \ba, \Rot \bb )}{S(\ba, \bb )} = 1     
    \]

\end{frame}





\begin{frame}
    \frametitle{Определитель проекции}


    Оператор $\HH: \R^2 \to \R^2$ проецирует векторы на прямую $\ell$.   \pause

    \begin{center}

        \begin{tikzpicture}[
        scale=2,
        MyPoints/.style={draw=blue,fill=white,thick},
        Segments/.style={draw=blue!50!red!70,thick},
        MyCircles/.style={green!50!blue!50,thin}, 
        every node/.style={scale=1.2}
        ]
        %\draw[color=gray,step=1.0,dotted] (-2.1,-5.1) grid (7.6,2.1);
        \clip (-1.5,-5.5) rectangle (6.5,2.5);

        \pgfmathparse{3}		\let\Xa\pgfmathresult
        \pgfmathparse{-3}		\let\Ya\pgfmathresult
        \coordinate (A) at (\Xa,\Ya);

        \pgfmathparse{6}		\let\Xb\pgfmathresult
        \pgfmathparse{0}		\let\Yb\pgfmathresult
        \coordinate (B) at (\Xb,\Yb);

        \pgfmathparse{-1}		\let\Xc\pgfmathresult
        \pgfmathparse{1}		\let\Yc\pgfmathresult
        \coordinate (C) at (\Xc,\Yc);

        \pgfmathparse{4}		\let\Xd\pgfmathresult
        \pgfmathparse{-4}		\let\Yd\pgfmathresult
        \coordinate (D) at (\Xd,\Yd);

        \pgfmathparse{0}		\let\Xe\pgfmathresult
        \pgfmathparse{0}		\let\Ye\pgfmathresult
        \coordinate (E) at (\Xe,\Ye);

        \pgfmathparse{3.5}		\let\Xf\pgfmathresult
        \pgfmathparse{-0.5}		\let\Yf\pgfmathresult
        \coordinate (F) at (\Xf,\Yf);

        \pgfmathparse{4}		\let\Xg\pgfmathresult
        \pgfmathparse{0}		\let\Yg\pgfmathresult
        \coordinate (G) at (\Xg,\Yg);

        \pgfmathparse{4.5}		\let\Xh\pgfmathresult
        \pgfmathparse{1.5}		\let\Yh\pgfmathresult
        \coordinate (H) at (\Xh,\Yh);



        \begin{scope}[cm={1.5,0.5,1.5,-0.5,(0,0)}]
        \draw[pattern=north west lines, pattern color=blue!50, draw=none ] (0,0) rectangle (2,2);
        \draw[-{Latex[length=4.5mm, width=2.5mm]}, >=stealth, vecb,thick] (0,0)--(0,2) node[below]{$\ba$};
        \draw[-{Latex[length=4.5mm, width=2.5mm]}, >=stealth, vecb,thick] (0,0)--(2,0) node[above]{$\bb$};
        \end{scope};


        \draw[ black,dashed] (C)--(D) node[below right]{$\ell$};

        \draw[dashed] (A)--(B);

        \draw[vecb,line width=0.5mm] (E)--(A) ;



        \node [above right] at (0, 5) {$\operatorname{H} \bv = \bb$}; 



        \tkzMarkRightAngle[size=0.3](B,A,C);


        \end{tikzpicture}





    \end{center}



\end{frame}




\begin{frame}
    \frametitle{Определитель проекции}


    Оператор $\HH: \R^2 \to \R^2$ проецирует векторы на прямую $\ell$.  



    \pause

    При проекции любой параллелограмм «складывается» в отрезок нулевой площади.


    \[
    \det \HH = \frac{S(\HH \ba, \HH \bb )}{S(\ba, \bb )} = 0     
    \]

\end{frame}




\begin{frame}
    \frametitle{Чем прекрасна ориентированная площадь?}

    \begin{block}{Утверждение}
        Ориентированная площадь $S(\ba, \bb)$ линейна по каждому аргументу:
        \[
        S(\lambda \ba, \bb) = \lambda S(\ba, \bb), \quad S(\ba + \bb, \bc) = S(\ba, \bc) + S(\bb, \bc)    
        \]
    \end{block}
    \pause
  \begin{center}
      \begin{tikzpicture}[
      scale=1.5,
      MyPoints/.style={draw=blue,fill=white,thick},
      Segments/.style={draw=blue!50!red!70,thick},
      MyCircles/.style={green!50!blue!50,thin}, 
      every node/.style={scale=1}
      ]
      %\grid;
      \clip (-1,-2.5) rectangle (6.5, 4.5);





      %{\verb!->!new, arrowhead = 2mm, line width=4pt}
      %, arrowhead = 3mm
      %, arrowhead = 0.2

      % Feel free to change here coordinates of points A and B
      \pgfmathparse{0}		\let\Xa\pgfmathresult
      \pgfmathparse{0}		\let\Ya\pgfmathresult
      \coordinate (A) at (\Xa,\Ya);

      \pgfmathparse{4}		\let\Xb\pgfmathresult
      \pgfmathparse{3}		\let\Yb\pgfmathresult
      \coordinate (B) at (\Xb,\Yb);

      \pgfmathparse{4}		\let\Xc\pgfmathresult
      \pgfmathparse{0}		\let\Yc\pgfmathresult
      \coordinate (C) at (\Xc,\Yc);




      \pgfmathparse{4}		\let\Xg\pgfmathresult
      \pgfmathparse{0}		\let\Yg\pgfmathresult
      \coordinate (G) at (\Xg,\Yg);

      \pgfmathparse{4}		\let\Xh\pgfmathresult
      \pgfmathparse{0}		\let\Yh\pgfmathresult
      \coordinate (H) at (\Xh,\Yh);

      \pgfmathparse{4}		\let\Xi\pgfmathresult
      \pgfmathparse{0}		\let\Yi\pgfmathresult
      \coordinate (I) at (\Xi,\Yi);



      \begin{scope}[cm={0.5,-1,1.5,0,(0,0)}]
      \draw[pattern=north west lines, pattern color=blue!50, draw=none ] (0,0) rectangle (2,2);
      \draw[-{Latex[length=4.5mm, width=2.5mm]}, >=stealth, vecb,thick] (0,0)--(0,2);
      \draw[{Latex[length=4.5mm, width=2.5mm]}-, >=stealth, vecb,thick] (0,0)--(2,0) node[midway, left]{$\ba$};
      \draw[{Latex[length=4.5mm, width=2.5mm]}-, >=stealth, vecb,thick] (0,2)--(2,2) ;
      \draw[-{Latex[length=4.5mm, width=2.5mm]}, >=stealth, vecb,thick] (2,0)--(2,2) node[midway, below]{$\bc$};

      \end{scope};

      \begin{scope}[cm={1.5,0,1,2,(0,0)}]
      \draw[pattern=north west lines, pattern color=blue!50, draw=none ] (0,0) rectangle (2,2);
      \draw[-{Latex[length=4.5mm, width=1.5mm]}, >=stealth, vecb,thick] (0,0)--(0,2) node[midway, above left]{$\bb$};
      \draw[-{Latex[length=4.5mm, width=1.5mm]}, >=stealth, vecb,thick] (2,0)--(2,2) ;
      \draw[-{Latex[length=4.5mm, width=2.5mm]}, >=stealth, vecb,thick] (0,2)--(2,2) node[midway, above ]{$\bc$};
      \end{scope};


      \draw[-{Latex[length=4.5mm, width=1.5mm]}, >=stealth, vecb,thick] (1,-2)--(2,4);


      \draw[-{Latex[length=4.5mm, width=1.5mm]}, >=stealth, vecb,thick] (4,-2)--(5,4) node[midway, right ]{$\ba+\bb$};














      %\tkzMarkAngle[size=1, mark = none, arrows=->,line width=1.5pt, mkcolor=red ](B,A,E);



      \end{tikzpicture}


  \end{center}



    

\end{frame}



\begin{frame}
    \frametitle{Корректность идеи определителя}

    Величина $\det \LL = \frac{S(\LL\ba, \LL\bb)}{S(\ba, \bb)}$ не зависит от выбора $\ba$ и $\bb$!


    \pause
    \begin{block}{Идея доказательства}
        Обозначим $\be_1 = \begin{pmatrix}
            1 \\
            0 \\
        \end{pmatrix}$, $\be_2 = \begin{pmatrix}
            0 \\
            1 \\
        \end{pmatrix}$.
        \pause

        Возьмём $\ba = 5\be_1 + 7\be_2$. Найдём $S(\LL\ba, \LL\be_2)/{S(\ba, \be_2)}$:
        \pause
        \[
        \frac{S(\LL (5\be_1 + 7\be_2), \LL \be_2)}{S(5\be_1 + 7\be_2,\be_2)} =
         \frac{S(\LL 5\be_1 , \LL \be_2) + S(\LL 7\be_2 , \LL \be_2)}{S(5\be_1,\be_2) + S(7\be_2,\be_2)} =
        \]
        \pause
        \[
         =\frac{5S(\LL \be_1 , \LL \be_2) + 0}{5S(\be_1,\be_2) + 0} =
         \frac{S(\LL \be_1 , \LL \be_2)}{S(\be_1,\be_2)}
        \]
    \end{block}

    

\end{frame}



\begin{frame}
    \frametitle{Ещё один взгляд на определитель}
    
    Обозначим $\be_1 = \begin{pmatrix}
        1 \\
        0 \\
    \end{pmatrix}$, $\be_2 = \begin{pmatrix}
        0 \\
        1 \\
    \end{pmatrix}$.

    \pause

    \begin{block}{Определение}
        Преобразуем параллелограмм, образованный векторами $\be_1$ и $\be_2$, с помощью оператора $\LL$. 
        
        \alert{Определитель линейного оператора} $\LL:\R^2 \to \R^2$ равен ориентированной 
        площади полученного параллелограмма.
        \[
        \det \LL = S(\LL\be_1, \LL\be_2)    
        \]
        
    \end{block}
    
\end{frame}


\begin{frame}
    \frametitle{Определитель в пространстве}

    Заменим ориентированную площадь параллелограмма $S(\ba, \bb)$ 
    на ориентированный объём параллелепипеда $S(\ba, \bb, \bc)$.
    
    \pause

\begin{block}{Определение}
    Возьмём любые три вектора $\ba$, $\bb$ и $\bc$, для которых $S(\ba, \bb, \bc)\neq 0$.

    \alert{Определитель} оператора $\LL:\R^3 \to \R^3$ показывает во сколько раз изменяется
    ориентированный объём
    \[
    \det \LL = \frac{S(\LL\ba, \LL\ba, \LL\bc)}{S(\ba, \bb, \bc)}    
    \]
\end{block} 


\end{frame}



\begin{frame}
    \frametitle{А что такое ориентированный объём?}

\pause 

Обозначим $\be_1 = \begin{pmatrix}
    1 \\
    0 \\
    0 \\
\end{pmatrix}$, $\be_2 = \begin{pmatrix}
    0 \\
    1 \\
    0
\end{pmatrix}$, $\be_3 = \begin{pmatrix}
    0 \\
    0 \\
    1 \\
\end{pmatrix}$.

% Скажем, что $S(\be_1, \be_2, \be_3) = 1$.

% Сопоставим большой, указательный и средний палец этим векторам.

\pause

\begin{block}{Определение}
    Рассмотрим параллелепипед, образованный $\ba$, $\bb$ и $\bc$.

    \pause

    С помощью поворота: 

    Cовместим вектор $\be_1$ с вектором $\ba$;

    Затем вектор $\be_2$ «положим» в плоскость $\ba$, $\bb$.

    \pause

    \alert{Ориентированный объём} $S(\ba, \bb, \bc)$ объявим отрицательным,
    если векторы $\be_3$ и $\bc$ смотрят в разные полупространства.
    
\end{block}


\end{frame}


\begin{frame}
    \frametitle{Определитель в пространстве}


    \begin{center}
    \begin{tikzpicture}[
    scale=1.4,
    MyPoints/.style={draw=blue,fill=white,thick},
    Segments/.style={draw=blue!50!red!70,thick},
    MyCircles/.style={green!50!blue!50,thin}, 
    every node/.style={scale=1.2}
    ]
    %\draw[color=gray,step=1.0,dotted] (-5.5,-6.5) grid (5.5,6.5); 
    \clip (-5.5,-6.9) rectangle (9.1,6.5);

    %{\verb!->!new, arrowhead = 2mm, line width=4pt}
    %, arrowhead = 3mm
    %, arrowhead = 0.2

    % Feel free to change here coordinates of points A and B
    \pgfmathparse{0}		\let\Xa\pgfmathresult
    \pgfmathparse{0}		\let\Ya\pgfmathresult
    \coordinate (A) at (\Xa,\Ya);

    \pgfmathparse{0}		\let\Xb\pgfmathresult
    \pgfmathparse{3}		\let\Yb\pgfmathresult
    \coordinate (B) at (\Xb,\Yb);

    \pgfmathparse{2}		\let\Xc\pgfmathresult
    \pgfmathparse{2}		\let\Yc\pgfmathresult
    \coordinate (C) at (\Xc,\Yc);

    \pgfmathparse{4}		\let\Xd\pgfmathresult
    \pgfmathparse{0}		\let\Yd\pgfmathresult
    \coordinate (D) at (\Xd,\Yd);

    \pgfmathparse{2}		\let\Xe\pgfmathresult
    \pgfmathparse{5}		\let\Ye\pgfmathresult
    \coordinate (E) at (\Xe,\Ye);

    \pgfmathparse{6}		\let\Xf\pgfmathresult
    \pgfmathparse{5}		\let\Yf\pgfmathresult
    \coordinate (F) at (\Xf,\Yf);

    \pgfmathparse{4}		\let\Xg\pgfmathresult
    \pgfmathparse{3}		\let\Yg\pgfmathresult
    \coordinate (G) at (\Xg,\Yg);

    \pgfmathparse{6}		\let\Xg\pgfmathresult
    \pgfmathparse{2}		\let\Yg\pgfmathresult
    \coordinate (H) at (\Xg,\Yg);


    \draw[-{Latex[length=4.5mm, width=2.5mm]}, >=stealth, vecb,  thick] (A)--(B) node[midway, left]{$\ba$};

    \draw[-{Latex[length=4.5mm, width=2.5mm]}, >=stealth, vecb,  thick] (A)--(C) node[midway, above]{$\bb$};

    \draw[-{Latex[length=4.5mm, width=2.5mm]}, >=stealth, vecb,  thick] (A)--(D) node[midway, below]{$\bc$};

    \draw[vecb, dashed] (C)--(E);
    \draw[vecb, dashed] (B)--(E);
    \draw[vecb, dashed] (E)--(F);
    \draw[vecb, dashed] (G)--(F);
    \draw[vecb, dashed] (G)--(B);
    \draw[vecb, dashed] (G)--(D);
    \draw[vecb, dashed] (D)--(H);
    \draw[vecb, dashed] (F)--(H);
    \draw[vecb, dashed] (C)--(H);

    \begin{scope}[cm={0.1,-0.9,-1,0,(0,0)}]
    \draw[-{Latex[length=4.5mm, width=2.5mm]}, >=stealth,   thick] (0,0)--(0,3) node[midway, above]{$\operatorname{L} \ba$};
    \draw[-{Latex[length=4.5mm, width=2.5mm]}, >=stealth,   thick] (0,0)--(2,2) node[midway, above left ]{$\operatorname{L} \bb$};
    \draw[-{Latex[length=4.5mm, width=2.5mm]}, >=stealth,   thick] (0,0)--(4,0) node[midway, right]{$\operatorname{L} \bc$};
    \draw[vecb, dashed] (2,2)--(2,5);
    \draw[vecb, dashed] (0,3)--(2,5);
    \draw[vecb, dashed] (2,5)--(6,5);
    \draw[vecb, dashed] (4,3)--(6,5);
    \draw[vecb, dashed] (4,3)--(0,3);
    \draw[vecb, dashed] (4,3)--(4,0);
    \draw[vecb, dashed] (4,0)--(6,2);
    \draw[vecb, dashed] (6,5)--(6,2);
    \draw[vecb, dashed] (2,2)--(6,2);
    \end{scope}



    %\node [right,darkgray] at (-5,-6) {$\operatorname{LF}$ }; 

    %\node [right,darkgray] at (5,5.5) {$\operatorname{F}$ }; 



    %\draw[-{Latex[length=4.5mm, width=2.5mm]}, >=stealth, thick] (A)--(B) node[left]{$\ba$};
    %
    %\draw[-{Latex[length=4.5mm, width=2.5mm]}, >=stealth, thick] (A)--(C) node[above]{$\bb$};
    %
    %\draw[-{Latex[length=4.5mm, width=2.5mm]}, >=stealth, thick] (A)--(D) node[above]{$\bc$};

    \node [above right,darkgray] at (1.3,-3.5) {$\det \operatorname{L} = \dfrac{S(\LL\ba,\LL\bb,\LL\bc)}{S(\ba,\bb,\bc)} $};





    \end{tikzpicture}

\end{center}        
    

\end{frame}




\begin{frame}
    \frametitle{Определитель растягивания компонент}


    Рассмотрим оператор $\LL : \begin{pmatrix}
      a_1 \\
      a_2 \\
      a_3 \\
    \end{pmatrix} \to 
    \begin{pmatrix}
      2a_1 \\
      3a_2 \\
      -5a_3 \\
    \end{pmatrix}$.
    

    \pause

    Одна сторона растягивается в два раза, вторая — в три раза, третья — в пять.


    Первые два вектора, $\be_1$ и $\be_2$, не изменяют направления при преобразовании.

    
    Третий вектор, $\be_3$, меняет полупространство, в котором он лежит относительно первых двух. 

    \pause

    \[
    \det \LL = \frac{S(\LL \ba, \LL \bb, \LL \bc )}{S(\ba, \bb, \bc )} = (-1)\cdot 2\cdot 3\cdot 5  = - 30
    \]

\end{frame}



\begin{frame}
    \frametitle{Определитель проекции}


    Оператор $\HH : \R^3 \to \R^3$ проецирует векторы на плоскость $\alpha$. 

    \pause

    Любой параллелепипед «складывается» в плоскую фигуру нулевого объёма.

    \pause

    \[
    \det \HH = \frac{S(\HH \ba, \HH \bb, \HH \bc )}{S(\ba, \bb, \bc )} = 0
    \]

\end{frame}


