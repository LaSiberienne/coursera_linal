% !TEX root = ../linal_lecture_03.tex

\begin{frame} % название фрагмента

\videotitle{Свойства определителя}

\end{frame}



\begin{frame}{Краткий план:}
  \begin{itemize}[<+->]
    \item Ориентированный объём в $\R^n$;
    \item Свойства определителя;
    \item Явная формула для определителя.
  \end{itemize}

\end{frame}




\begin{frame}
    \frametitle{Формализация ориентированного объёма}

    %Преобразование $\LL\R^n\to\R^n$, желаемые свойства $\det \LL$:

    Вектор $\be_i$ содержит на $i$-м месте единицу, а на остальных — нули.
    \pause

    \begin{enumerate}
        \item Верный гипер-объём базового гипер-кубика:
        \[
        S(\be_1, \be_2, \ldots, \be_n) = 1   \pause 
        \]
        \item Линейность по каждому аргументу:
        \[
            S(\red{\ba + \bb}, \bv_2, \bv_3, \ldots, \bv_n) =  S(\red{\ba}, \bv_2, \bv_3, \ldots, \bv_n) + 
        \]
        \[
            + S(\red{\bb}, \bv_2, \bv_3, \ldots, \bv_n)          
        \]
        \[
            S(\red{\lambda}\bv_1, \bv_2, \bv_3, \ldots, \bv_n) = \red{ \lambda } S(\bv_1, \bv_2, \bv_3, \ldots, \bv_n) \pause
        \]
        \item Антисимметричность: 
        \[
          S(\red{\bv_1}, \red{\bv_2}, \bv_3, \ldots, \bv_n) = - S(\red{\bv_2}, \red{\bv_1}, \bv_3, \ldots, \bv_n)  
        \]
    \end{enumerate}
\end{frame}



\begin{frame}
    \frametitle{Определитель во всей $n$-мерности}

    \begin{block}{Определение}
        Возьмём векторы $\bv_1$, \ldots, $\bv_n$, для которых $S(\bv_1, \ldots, \bv_n)\neq 0$.

        \alert{Определитель} оператора $\LL:\R^n \to \R^n$ показывает во сколько раз изменяется
        ориентированный гипер-объём произвольного параллелепипеда:
        \[
            \det \LL = \frac{S(\LL\bv_1, \ldots, \LL\bv_n)}{S(\bv_1, \ldots, \bv_n)}    
        \]
    \end{block}  
    
    \pause

\begin{block}{Определение}

    \alert{Определитель} оператора $\LL:\R^n \to \R^n$ показывает во сколько раз изменяется
    ориентированный гипер-объём базового гипер-кубика:
    \[
        \det \LL = S(\LL\be_1, \ldots, \LL\be_n)
    \]
\end{block}  


\end{frame}



\begin{frame}
    \frametitle{Определитель матрицы}

    \begin{block}{Определение}
        \alert{Определителем матрицы} называется определитель соответствующего линейного оператора. 
    \end{block}
    
    \pause

    В матрице $\LL$ $i$-й столбец равен $\LL\be_i$, поэтому 
    \[
    \det \LL = S(\col_1 \LL, \col_2 \LL, \ldots, \col_n \LL)    
    \]

    \pause
    \begin{block}{Утверждение}
        Определитель матрицы можно считать по строкам: 
    \[
    \det \LL = S(\row_1 \LL, \row_2 \LL, \ldots, \row_n \LL)    
    \]    
    \end{block}

    Определитель обозначают $\det \LL$ или $|\LL|$.

\end{frame}





\begin{frame}
    \frametitle{Быстрые признаки равенства нулю}

    \begin{enumerate}
        \item Если среди векторов есть два одинаковых, то 
        гипер-объём параллелепипеда равен нулю. 
        \[
            S(\ba, \ba, \bv_3, \ldots, \bv_n) = 0
        \]
        \pause
        \item Если среди векторов есть один нулевой, то 
        гипер-объём параллелепипеда равен нулю. 
        \[
            S(\bzero, \bv_2, \bv_3, \ldots, \bv_n) = 0
        \]
        \end{enumerate}

\end{frame}

\begin{frame}
\frametitle{Принцип Кавальери}

        «Cкашивание» параллелепипеда вбок не изменяет гипер-объём:
        \[
            S(\ba, \bb, \bv_3, \ldots, \bv_n) = S(\ba + \bb, \bb, \bv_3, \ldots, \bv_n)
        \]
    \pause
    \begin{center}
        \begin{tikzpicture}[
        scale=1.7,
        MyPoints/.style={draw=blue,fill=white,thick},
        Segments/.style={draw=blue!50!red!70,thick},
        MyCircles/.style={green!50!blue!50,thin}, 
        every node/.style={scale=1.2}
        ]
        \clip (-3.5,-1.5) rectangle (5.5,4.5);


        %%\draw[->, >=stealth] (-1,0)--(6.5,0) node[right]{$x_1$};
        %\draw[-{Latex[length=4.5mm, width=2.5mm]}, >=stealth] (0,-1)--(0,5) node[above left]{$x_2$};
        %
        %\draw[-{Latex[length=4.5mm, width=2.5mm]}, >=stealth] (-1,0)--(6.5,0) 
        %node[right]{$x_1$};

        % Feel free to change here coordinates of points A and B
        \pgfmathparse{0}		\let\Xa\pgfmathresult
        \pgfmathparse{0}		\let\Ya\pgfmathresult
        \coordinate (A) at (\Xa,\Ya);

        \pgfmathparse{1}		\let\Xb\pgfmathresult
        \pgfmathparse{4}		\let\Yb\pgfmathresult
        \coordinate (B) at (\Xb,\Yb);

        \pgfmathparse{3}		\let\Xc\pgfmathresult
        \pgfmathparse{0}		\let\Yc\pgfmathresult
        \coordinate (C) at (\Xc,\Yc);

        \pgfmathparse{4}		\let\Xd\pgfmathresult
        \pgfmathparse{4}		\let\Yd\pgfmathresult
        \coordinate (D) at (\Xd,\Yd);


        \pgfmathparse{-2}		\let\Xe\pgfmathresult
        \pgfmathparse{4}		\let\Ye\pgfmathresult
        \coordinate (E) at (\Xe,\Ye);



        \draw[-{Latex[length=4.5mm, width=2.5mm]}, >=stealth, vecb,thick] (A)--(B);

        \draw[-{Latex[length=4.5mm, width=2.5mm]}, >=stealth, vecb,thick] (C)--(D) node[midway,left]{$\ba$};


        \draw[-{Latex[length=4.5mm, width=2.5mm]}, >=stealth, veca,thick] (C)--(A) node[midway,below]{$\bb$};

        \draw[-{Latex[length=4.5mm, width=2.5mm]}, >=stealth, veca,thick] (D)--(B) ;

        \draw[-{Latex[length=4.5mm, width=2.5mm]}, >=stealth, veca,thick] (B)--(E) ;


        \draw[-{Latex[length=4.5mm, width=1.5mm]}, >=stealth, vecc,thick] (C)--(B) node[midway,above, sloped]{$\ba+\bb$};

        \draw[-{Latex[length=4.5mm, width=1.5mm]}, >=stealth, vecc,thick] (A)--(E) ;

        % \node [above right,darkgray] at (1,-1.5) {$S(\ba,\bb)=S(\ba+\bb,\bb)$};


        \end{tikzpicture}


    \end{center}


    
\end{frame}



\begin{frame}
    \frametitle{Принцип Кавальери на матрице}
    
    Единственным ненулевым элементом столбца можно «скосить» всю строку:

    \[
    \begin{vmatrix}
        0 & 3 & -2 \\
        \red{4} & 2 & 7 \\
        0 & 1 & -5 \\
    \end{vmatrix} =  \pause 
\begin{vmatrix}
    0 & 3 & -2 \\
    \red{4} & \blue{0} & 7 \\
    0 & 1 & -5 \\
\end{vmatrix}  = \pause 
\begin{vmatrix}
    0 & 3 & -2 \\
    \red{4} & \blue{0} & \blue{0} \\
    0 & 1 & -5 \\
\end{vmatrix}  
    \]

    \pause
    Единственным ненулевым элементом строки можно «скосить» весь столбец:
    
    \[
    \begin{vmatrix}
        3 & 3 & -2 \\
        \red{4} & 0 & 0 \\
        2 & 1 & -5 \\
    \end{vmatrix} = \pause 
\begin{vmatrix}
    \blue{0} & 3 & -2 \\
\red{4} & 0 & 0 \\
2 & 1 & -5 \\
\end{vmatrix}  = \pause 
\begin{vmatrix}
    \blue{0} & 3 & -2 \\
\red{4} & 0 & 0 \\
\blue{0} & 1 & -5 \\
\end{vmatrix}  
    \]
    
\end{frame}


\begin{frame}
    \frametitle{Определитель и ранг}

    \begin{block}{Утверждение}
        Для матрицы $\LL$ размера $n\times n$ четыре свойства эквиваленты:

        \begin{enumerate}
            \item Определитель равен нулю, $\det \LL = 0$.
            \pause
            \item Столбцы матрицы линейно зависимы.
            \pause
            \item Строки матрицы линейно зависимы.
            \pause
            \item Ранг матрицы меньше числа столбцов, $\rank \LL < n$.
        \end{enumerate}
    \end{block}


    

\end{frame}

\begin{frame}
    \frametitle{Определитель композиции}
    \begin{block}{Утверждение}
        Определитель композиции $A$ и $B$ равен произведению определителей:
        \[
          \det (AB) = \det A \det B  
        \]
    \end{block}
    \pause
    \begin{block}{Следствие}

        \[
        \det A \det A^{-1} = \det (A \cdot A^{-1}) = \det \Id = 1
        \]
    \end{block}

\end{frame}





\begin{frame}
    \frametitle{Спокойствие, только спокойствие!}

    \begin{block}{Утверждение}
        Свойства нормировки, линейности по аргументам и антисимметричности однозначно определяют функцию гипер-объёма $S(\bv_1, \ldots, \bv_n)$.
    \end{block}

    \pause

\begin{block}{Утверждение}
    Отношение гипер-объёмов $\det \LL = \frac{S(\LL\bv_1, \ldots, \LL\bv_n)}{S(\bv_1, \ldots, \bv_n)}$ не зависит от выбора $\bv_1$, \ldots, $\bv_n$.
\end{block}


\end{frame}




\begin{frame}
    \frametitle{Формула с перестановками}

    \begin{block}{Определение}
        \alert{Перестановкой} называют последовательность из $n$ чисел, 
        в которой каждое число от $1$ до $n$ встречается ровно один раз.
    \end{block}

    \pause
    Примеры: $(12345)$, $(32145)$, $(21354)$.

    \pause
    \begin{block}{Определение}
    Перестановку называют \alert{чётной}, если требуется чётное число смен местами двух чисел,
    чтобы привести перестановку к $(1234\ldots n)$.

    Если $\sigma$ — чётная перестановка, то пишут $\sign \sigma  = 1$,
    для нечётной пишут $\sign\sigma = -1$.
\end{block}
\pause
Примеры: 

$\sign(12345) = 1$, $\sign(32145)=-1$, $\sign(21354)=1$.


\end{frame}



\begin{frame}
\frametitle{Расстановка ладей!}

    Рассмотрим квадратную матрицу. 

    Перестановку $\sigma$ будем трактовать как инструкцию, какой элемент взять из очередной строки матрицы.
    
    \[
    (3124) \sim \begin{pmatrix}
    . & . & * & . \\
    * & . & . & . \\
    . & * & . & . \\
    . & . & . & * \\
    \end{pmatrix}    
    \]

    \pause

    С помощью $p(\sigma)$ обозначим произведение этих элементов. 

    Например, $p(3124) = a_{13} \cdot a_{21} \cdot a_{32} \cdot a_{44}$.


\end{frame}


\begin{frame}
\frametitle{Явная формула}

\begin{block}{Утверждение}
Трём свойствам определителя (нормировке, линейности, антисимметричности) удовлетворяет единственная функция
\[
    \det \LL = \sum_{\sigma} \sign (\sigma) \cdot p(\sigma).
\]

Перестановку $\sigma$ трактуем как инструкцию, какой элемент взять из очередной строки матрицы.

С помощью $p(\sigma)$ обозначено произведение этих элементов. 

\end{block}


\end{frame}


\begin{frame}
    \frametitle{Иллюстрация для $2\times 2$}

    \[
    \det \begin{pmatrix}
        a & b \\
        c & d \\
    \end{pmatrix} = \pause
    +\underset{\red{\sign(12)=1}}{\begin{pmatrix}
        a &  \\
         & d \\
    \end{pmatrix}} - 
    \underset{\textcolor{blue}{\sign(21)=-1}}{\begin{pmatrix}
         & b \\
        c &  \\
    \end{pmatrix}} = \pause  ad - bc
    \]

\end{frame}



\begin{frame}
    \frametitle{Иллюстрация для $3\times 3$}

    \[
    \det \begin{pmatrix}
        a & b & c \\
        d & e & f \\
        g & h & i \\
    \end{pmatrix} = \pause
    \]
\[
= +\underset{\red{\sign(123)=1}}{\begin{pmatrix}
    a &  &  \\
     & e &  \\
     &  & i \\
\end{pmatrix}} +
\underset{\red{\sign(312)=1}}{\begin{pmatrix}
     &  & c \\
    d &  &  \\
     & h &  \\
\end{pmatrix}} +
\underset{\red{\sign(231)=1}}{\begin{pmatrix}
     & b &  \\
     &  & f \\
    g &  &  \\
\end{pmatrix}}
\]
\[
- \underset{\textcolor{blue}{\sign(321)=-1}}{\begin{pmatrix}
     &  & c \\
     & e &  \\
    g &  &  \\
\end{pmatrix}} 
- \underset{\textcolor{blue}{\sign(213)=-1}}{\begin{pmatrix}
     & b & \\
    d &  &  \\
     &  & i \\
\end{pmatrix}} 
- \underset{\textcolor{blue}{\sign(132)=-1}}{\begin{pmatrix}
    a &  &  \\
     &  & f \\
     & h &  \\
\end{pmatrix}}=
\]
\pause
\[
= aei + cdh + bfg - ceg - bdi - afh
\]




\end{frame}



