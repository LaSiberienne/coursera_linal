% !TEX root = ../linal_lecture_03.tex

\begin{frame} % название фрагмента

\videotitle{Определитель и транспонирование}

\end{frame}



\begin{frame}{Краткий план:}
  \begin{itemize}[<+->]
    \item Транспонирование матрицы;
    \item Определитель и транспонирование;
    \item Разложение определителя по строке.
  \end{itemize}

\end{frame}




\begin{frame}
    \frametitle{Конструктивный подход к транспонированию}

    Определение транспонирования оператора основано на свойстве 
    \[
        \langle \LL \ba, \bb \rangle = \langle \ba, \LL^T\bb\rangle.
    \]

    \pause

    Возьмём, к примеру, $\ba = \be_2$ и $\bb = \be_3$:
    \[
        \langle \col_2 \LL, \be_3 \rangle = \langle \be_2, \col_3 \LL^T \rangle
    \]
    \pause 
    \[
        \LL_{32} = \LL^T_{23}        
    \]
    \pause

    Транспонирование меняет местами строки и столбцы матрицы!


\end{frame}


\begin{frame}
    \frametitle{Транспонирование матрицы}

    Пример:
    \[
    \LL = \begin{pmatrix}
        1 & 2 & 3 & 4 \\
        5 & 6 & 7 & 8 \\
        9 & 10 & 11 & 12 \\
    \end{pmatrix}    
    \]
    \pause
    \[
    \LL^T = \begin{pmatrix}
        1 & 5 & 9 \\
        2 & 6 & 10 \\
        3 & 7 & 11 \\
        4 & 8 & 12 \\
    \end{pmatrix}    
    \]
    \pause
    Заметим, что $\LL^{TT}=\LL$.
\end{frame}


\begin{frame}
    \frametitle{Транспонирование и определитель}

    Явная формула определителя:
    \[    
    \det \LL = \sum_{\sigma} \sign(\sigma) p(\sigma) 
    \]
    \pause
    %
    Перестановка диктует, какой элемент выбрать в каждой строке:
    \[
    (3124) \sim \begin{pmatrix}
    . & . & * & . \\
    * & . & . & . \\
    . & * & . & . \\
    . & . & . & * \\
    \end{pmatrix}
    \]
    \pause
    \begin{block}{Утверждение}
        Если в матрице выбран один элемент в каждой строке и в каждом столбце, то при транспонировании это свойство сохраняется.
    \end{block}

\end{frame}

\begin{frame}
\frametitle{Транспонирование и определитель}

\begin{block}{Утверждение}
    Чётности перестановок, кодирующих координаты элементов по строкам и по столбцам, одинаковые.
\end{block}
    %
    \[
        \begin{pmatrix}
    . & . & \textcolor{red}{a} & . \\
    \textcolor{red}{b} & . & . & . \\
    . & c & . & . \\
    . & . & . & d \\
    \end{pmatrix}
    \]   
    %
    \[
        (\col_1 \leftrightarrow \col_3) \sim (\row_1 \leftrightarrow \row_2)
    \]
    \pause
    \[
        \sign(3124) = \sign(2314)
    \]
\end{frame}



\begin{frame}
    \frametitle{Транспонирование и определитель}

Перестановка $\sigma$ выбирает элемент в каждой строке:
\[    
\det \LL = \sum_{\sigma} \sign(\sigma) p(\sigma) 
\]

Перестановка $\sigma$ выбирает элемент в каждом столбце:
\[    
\det \LL^T = \sum_{\sigma} \sign(\sigma) p^T(\sigma)
\]

\pause

\begin{block}{Утверждение}
\[
    \det \LL = \det \LL^T
\]
\end{block}

\end{frame}


\begin{frame}
    \frametitle{Разложение по столбцу на примере}
    Возьмём аддитивность:
    \[
\begin{vmatrix}
    -1 & \red{2} & 3 \\
    4 & \red{5} & 6 \\
    7 & \red{8} & 9 \\
\end{vmatrix}  =       
\begin{vmatrix}
    -1 & \red{2} & 3 \\
    4 & 0 & 6 \\
    7 & 0 & 9 \\
\end{vmatrix} +
\begin{vmatrix}
    -1 & 0 & 3 \\
    4 & \red{5} & 6 \\
    7 & 0 & 9 \\
\end{vmatrix} +
\begin{vmatrix}
    -1 & 0 & 3 \\
    4 & 0 & 6 \\
    7 & \red{8} & 9 \\
\end{vmatrix} \pause
    \]
    Добавим немного принципа Кавальери:
\[
    \begin{vmatrix}
        -1 & \red{2} & 3 \\
        4 & \red{5} & 6 \\
        7 & \red{8} & 9 \\
    \end{vmatrix}  =       
    \begin{vmatrix}
        \blue{0} & \red{2} & \blue{0} \\
        4 & 0 & 6 \\
        7 & 0 & 9 \\
    \end{vmatrix} +
    \begin{vmatrix}
        -1 & 0 & 3 \\
\blue{0} & \red{5} & \blue{0} \\
        7 & 0 & 9 \\
    \end{vmatrix} +
    \begin{vmatrix}
        -1 & 0 & 3 \\
        4 & 0 & 6 \\
\blue{0} & \red{8} & \blue{0} \\
    \end{vmatrix} \pause
\]
Взболтаем и переставим столбцы:
\[
    \begin{vmatrix}
        -1 & \red{2} & 3 \\
        4 & \red{5} & 6 \\
        7 & \red{8} & 9 \\
    \end{vmatrix}  =       
    (-1)^1 \begin{vmatrix}
        \red{2} & 0   & 0 \\
        0 & 4  & 6 \\
        0 & 7  & 9 \\
    \end{vmatrix} +
    (-1)^2 \begin{vmatrix}
        \red{5} & 0 & 0 \\
       0 & -1 & 3 \\
        0 & 7 & 9 \\
    \end{vmatrix} +
    (-1)^3\begin{vmatrix}
        \red{8} & 0 & 0 \\
        0 & -1 & 3 \\
        0 & 4 & 6 \\
    \end{vmatrix} 
\]
\end{frame}


\begin{frame}
\frametitle{Разложение по столбцу на примере}
Взболтаем и переставим столбцы:
\[
    \begin{vmatrix}
        -1 & \red{2} & 3 \\
        4 & \red{5} & 6 \\
        7 & \red{8} & 9 \\
    \end{vmatrix}  =       
    (-1)^1 \begin{vmatrix}
        \red{2} & 0   & 0 \\
        0 & 4  & 6 \\
        0 & 7  & 9 \\
    \end{vmatrix} +
    (-1)^2 \begin{vmatrix}
        \red{5} & 0 & 0 \\
       0 & -1 & 3 \\
        0 & 7 & 9 \\
    \end{vmatrix} +
    (-1)^3\begin{vmatrix}
        \red{8} & 0 & 0 \\
        0 & -1 & 3 \\
        0 & 4 & 6 \\
    \end{vmatrix} \pause
\]
Снизим размерность:
\[
    \begin{vmatrix}
        -1 & \red{2} & 3 \\
        4 & \red{5} & 6 \\
        7 & \red{8} & 9 \\
    \end{vmatrix}  =       
    (-1)^1 \red{2} \begin{vmatrix}
         4  & 6 \\
         7  & 9 \\
    \end{vmatrix} +
    (-1)^2 \red{5}\begin{vmatrix}
        -1 & 3 \\
         7 & 9 \\
    \end{vmatrix} +
    (-1)^3 \red{8} \begin{vmatrix}
         -1 & 3 \\
         4 & 6 \\
    \end{vmatrix} 
\]




\end{frame}


\begin{frame}
    \frametitle{Разложение по столбцу}

    Выберем любой столбец и «пробежимся» вдоль него!

    \[
        \begin{vmatrix}
            * & \red{a_{12}} & * \\
            * & \red{a_{22}} & * \\
            * & \red{a_{32}} & * \\
        \end{vmatrix}  =      
    \]
    \[ 
       =(-1)^{1+2} \red{a_{12}} \det A_{12}^{\cross} +
        (-1)^{2+2} \red{a_{22}} \det A_{22}^{\cross} +
        (-1)^{3+2} \red{a_{32}} \det A_{32}^{\cross}
    \]
    Матрица $A_{ij}^{\cross}$ получается из исходной $A$ вычеркиванием строки $i$ и
    столбца $j$.
    \pause
    \begin{block}{Утверждение}       
    \[
    \det \LL = \sum_{i=1}^{n} (-1)^{i+j} a_{ij} \det A_{ij}^{\cross},
    \]
    \end{block}

    
\end{frame}


\begin{frame}
    \frametitle{Разложение по строке}

    Можно раскладывать и по строке $i$:
\begin{block}{Утверждение}
\[
\det A = \sum_{j=1}^{n} (-1)^{i+j} a_{ij} \det A_{ij}^{\cross},
\]
\end{block}

    \begin{block}{Определение}
        \alert{Алгебраическим дополнением} элемента $a_{ij}$ матрицы $A$ называют величину
        \[
        A_{ij} = (-1)^{i+j} \det A_{ij}^{\cross},    
        \]
    \end{block}

    Матрица $A_{ij}^{\cross}$ получается из исходной $A$ вычеркиванием строки $i$ и
    столбца $j$.

\end{frame}