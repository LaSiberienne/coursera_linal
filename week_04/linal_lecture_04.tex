\documentclass[14 pt,xcolor=dvipsnames]{beamer}

\usepackage{epsdice}

\usepackage[absolute,overlay]{textpos}

\usepackage[orientation=portrait,size=custom,width=25.4,height=19.05]{beamerposter}

%25,4 см 19,05 см размеры слайда в powerpoint

\usetheme{metropolis}
\metroset{
  %progressbar=none,
  numbering=none,
  subsectionpage=progressbar,
  block=fill
}

%\usecolortheme{seahorse}

\usepackage{fontspec}
\usepackage{polyglossia}
\setmainlanguage{russian}


\usepackage{fontawesome5} % removed [fixed]
\setmainfont[Ligatures=TeX]{Myriad Pro}
\setsansfont{Myriad Pro}


\usepackage{amssymb,amsmath,amsxtra,amsthm}


\usepackage{unicode-math}
\usepackage{centernot}

\usepackage{graphicx}
\graphicspath{{img/}}

\usepackage{wrapfig}
\usepackage{animate}
\usepackage{tikz}
%\usetikzlibrary{shapes.geometric,patterns,positioning,matrix,calc,arrows,shapes,fit,decorations,decorations.pathmorphing}
\usepackage{pifont}
\usepackage{comment}
\usepackage[font=small,labelfont=bf]{caption}
\captionsetup[figure]{labelformat=empty}
\includecomment{techno}

\usefonttheme[onlymath]{serif}


%Расположение

\setbeamersize{text margin left=15 mm,text margin right=5mm} 
\setlength{\leftmargini}{38 pt}

%\usepackage{showframe}
%\usepackage{enumitem}
%\setlist{leftmargin=5.5mm}


%Цвета от дирекции

\definecolor{dirblack}{RGB}{58, 58, 58}
\definecolor{dirwhite}{RGB}{245, 245, 245}
\definecolor{dirred}{RGB}{149, 55, 53}
\definecolor{dirblue}{RGB}{0, 90, 171}
\definecolor{dirorange}{RGB}{235, 143, 76}
\definecolor{dirlightblue}{RGB}{75, 172, 198}
\definecolor{dirgreen}{RGB}{155, 187, 89}
\definecolor{dircomment}{RGB}{128, 100, 162}

\setbeamercolor{title separator}{bg=dirlightblue!50, fg=dirblue}

%Цвета блоков

\setbeamercolor{block title}{bg=dirblue!30,fg=dirblack}

\setbeamercolor{block title example}{bg=dirlightblue!50,fg=dirblack}

\setbeamercolor{block body example}{bg=dirlightblue!20,fg=dirblack}

\AtBeginEnvironment{exampleblock}{\setbeamercolor{itemize item}{fg=dirblack}}
%\setbeamertemplate{blocks}[rounded][shadow]

% Набор команд для удобства верстки

\newcommand{\RR}{\mathbb{R}}
\newcommand{\ZZ}{\mathbb{Z}}
\newcommand{\la}{\lambda}

% Набор команд для структуризации

%\newcommand{\quest}{\faQuestionCircleO}
%\faPencilSquareO \faPuzzlePiece \faQuestionCircleO  \faIcon*[regular]{file} {\textcolor{dirblue}
%\newcommand{\quest}{\textcolor{dirblue}{\boxed{\textbf{?}}}
\newcommand{\task}{\faIcon{tasks}}
\newcommand{\exmpl}{\faPuzzlePiece}
\newcommand{\dfn}{\faIcon{pen-square}}
\newcommand{\quest}{\textcolor{dirblue}{\faQuestionCircle[regular]}}
\newcommand{\acc}[1]{\textcolor{dirred}{#1}}
\newcommand{\accm}[1]{\textcolor{dirred}{#1}}
\newcommand{\acct}[1]{\textcolor{dirblue}{#1}}
\newcommand{\acctm}[1]{\textcolor{dirblue}{#1}}
\newcommand{\accex}[1]{\textcolor{dirblack}{\bf #1}}
\newcommand{\accexm}[1]{\textcolor{dirblack}{ \mathbf{#1}}}
\newcommand{\acclp}[1]{\textcolor{dirorange}{\it #1}}


\newcommand{\videotitle}[1]{\begin{center}
    \textcolor{dirblue}{#1}

    \todo{название видеофрагмента}
\end{center}}

\newcommand{\lecturetitle}[1]{\begin{center}
    \textcolor{dirblue}{#1}

    \todo{название лекции}
\end{center}}




\newcommand{\todo}[1]{\textcolor{dircomment}{\bf #1}}

\newcommand{\spcbig}{\vspace{-10 pt}}
\newcommand{\spcsmall}{\vspace{-5 pt}}

%\usepackage{listings}
%\lstset{
%xleftmargin=0 pt,
%  basicstyle=\small, 
%  language=Python,
  %tabsize = 2,
%  backgroundcolor=\color{mc!20!white}
%}



%\newcommand{\mypart}[1]{\begin{frame}[standout]{\huge #1}\end{frame}}

\setbeamercolor{background canvas}{bg=}

% frame title setup
\setbeamercolor{frametitle}{bg=,fg=dirblue}
\setbeamertemplate{frametitle}[default][left]

\addtobeamertemplate{frametitle}{\hspace*{-0.5 cm}}{\vspace*{0.25cm}}


%Шрифты
\setbeamerfont{frametitle}{family=\rmfamily,series=\bfseries,size={\fontsize{33}{30}}}
\setbeamerfont{framesubtitle}{family=\rmfamily,series=\bfseries,size={\fontsize{26}{20}}}





\usepackage{physics}
\newcommand{\R}{\mathbb{R}}
\newcommand{\CC}{\mathbb{C}}

\newcommand{\Rot}{\mathrm{R}}
\newcommand{\HH}{\mathrm{H}}
\newcommand{\Id}{\mathrm{I}}



\usepackage[outline]{contour}


\usepackage{pgfplots}
\pgfplotsset{compat=newest}

\usepackage{tikz}
\usetikzlibrary{calc}
\usetikzlibrary{quotes,angles}
\usetikzlibrary{arrows}
\usetikzlibrary{arrows.meta}
\usetikzlibrary{positioning,intersections,decorations.markings}
\usetikzlibrary{patterns}

\usepackage{tkz-euclide} 

\newcommand{\grid}{\draw[color=gray,step=1.0,dotted] (-2.1,-2.1) grid (9.6,6.1)}

\newcommand{\ba}{\symbf{a}}
\newcommand{\be}{\symbf{e}}
\newcommand{\bb}{\symbf{b}}
\newcommand{\bc}{\symbf{c}}
\newcommand{\bd}{\symbf{d}}
\newcommand{\bx}{\symbf{x}}
\newcommand{\by}{\symbf{y}}
\newcommand{\bff}{\symbf{f}} % \bf is already def
\newcommand{\bv}{\symbf{v}}
\newcommand{\bzero}{\symbf{0}}
\newcommand{\red}[1]{\textcolor{red}{#1}}
\newcommand{\green}[1]{\textcolor{green}{#1}}
\newcommand{\blue}[1]{\textcolor{blue}{#1}}


\DeclareMathOperator{\eig}{Eig}

\DeclareMathOperator{\Lin}{Span}
\DeclareMathOperator{\col}{col}
\DeclareMathOperator{\row}{row}

\DeclareMathOperator{\adj}{adj}

\DeclareMathOperator{\sign}{sign}

\DeclareMathOperator{\charp}{char}

\DeclareMathOperator{\Span}{Span}
\DeclareMathOperator{\Image}{Image}


\DeclareMathOperator{\LL}{L}

%\tikzset{>=latex}

\colorlet{veca}{red}
\colorlet{vecb}{blue}
\colorlet{vecc}{olive}


\tikzset{cross/.style={cross out, draw=black, minimum size=2*(#1-\pgflinewidth), inner sep=0pt, outer sep=0pt},
%default radius will be 1pt. 
cross/.default={5pt}}





\begin{document}

% \maketitle


\begin{frame} % название лекции


\lecturetitle{Спектральное разложение}

\end{frame}


% !TEX root = ../linal_lecture_04.tex

\begin{frame} % название фрагмента

\videotitle{Собственные числа и векторы}

\end{frame}



\begin{frame}{Краткий план:}
  \begin{itemize}[<+->]
    \item Собственные числа и собственные векторы матрицы.
    \item Характеристический многочлен.
    \item Алгебраическая кратность.
  \end{itemize}

\end{frame}


\begin{frame}{От оператора к матрице}

\begin{block}{Определение}
Если для оператора $\LL: \R^n \to \R^n$ найдётся такой ненулевой вектор $\bv$, 
что $\LL \bv=\lambda \cdot \bv$, где $\lambda \in \R$, то:
  \begin{itemize}
    \item вектор $\bv$ называется \alert{собственным вектором};
    \item число $\lambda$ называется \alert{собственным числом}.
  \end{itemize} 
\end{block}

\begin{center}
\begin{tikzpicture}[
scale=1.6,
MyPoints/.style={draw=blue,fill=white,thick},
Segments/.style={draw=blue!50!red!70,thick},
MyCircles/.style={green!50!blue!50,thin}, 
every node/.style={scale=1.2}
]
%\grid;
% \clip (-.5,-.5) rectangle (6.5,5.5);


%%\draw[->, >=stealth] (-1,0)--(6.5,0) node[right]{$x_1$};
%\draw[-{Latex[length=4.5mm, width=2.5mm]}, >=stealth] (0,-1)--(0,5) node[above left]{$x_2$};
%
%\draw[-{Latex[length=4.5mm, width=2.5mm]}, >=stealth] (-1,0)--(6.5,0) 
%node[right]{$x_1$};

% Feel free to change here coordinates of points A and B
\pgfmathparse{0}		\let\Xa\pgfmathresult
\pgfmathparse{0}		\let\Ya\pgfmathresult
\coordinate (A) at (\Xa,\Ya);

\pgfmathparse{5}		\let\Xb\pgfmathresult
\pgfmathparse{0}		\let\Yb\pgfmathresult
\coordinate (B) at (\Xb,\Yb);

\pgfmathparse{2.5}		\let\Xc\pgfmathresult
\pgfmathparse{5}		\let\Yc\pgfmathresult
\coordinate (C) at (\Xc,\Yc);

\pgfmathparse{1.5}		\let\Xd\pgfmathresult
\pgfmathparse{3}		\let\Yd\pgfmathresult
\coordinate (D) at (\Xd,\Yd);

\pgfmathparse{2}		\let\Xe\pgfmathresult
\pgfmathparse{1.5}		\let\Ye\pgfmathresult
\coordinate (E) at (\Xe,\Ye);



% Let I be the midpoint of [AB]
\pgfmathparse{(\Xb+\Xa)/2} \let\XI\pgfmathresult
\pgfmathparse{(\Yb+\Ya)/2} \let\YI\pgfmathresult
\coordinate (I) at (\XI,\YI);	


\draw[-{Latex[length=4.5mm, width=2.5mm]}, >=stealth, darkgray,thick] (A)--(B) node[above]{$\operatorname{L} \bb \neq \lambda \bb$};


\draw[-{Latex[length=4.5mm, width=2.5mm]}, >=stealth, vecb,thick] (A)--(E) node[midway,below]{$\bb$};


\draw[-{Latex[length=4.5mm, width=2.5mm]}, >=stealth, darkgray,thick] (A)--(C) node[above]{$\operatorname{L} \ba = \lambda \ba$};

\draw[-{Latex[length=4.5mm, width=2.5mm]}, >=stealth, veca,thick] (A)--(D) node[midway,left]{$\ba$};


\end{tikzpicture}
\end{center}    


\end{frame}
        



\begin{frame}
    \frametitle{Собственные числа и векторы матрицы}

    \begin{block}{Определение}
        \alert{Собственными числами} и \alert{собственными векторами} матрицы размера $n\times n$ называются собственные числа и векторы 
        соответствующего линейного оператора.
    \end{block}
    \pause

    Для абстрактного векторного пространства $V$ матрица $\LL_{\be \be}$ линейного оператора $\LL: V\to V$ зависит от выбора базиса $\be$.
    При этом выбор базиса $\be$ никак не влияет на собственные числа и собственные векторы. 
\end{frame}


\begin{frame}
    \frametitle{Количество собственных векторов}

    Из уравнения $\LL \bv = \lambda \bv$ находим вектор $\bv$ и число $\lambda$.

    \pause

    Если найдётся один собственный вектор $\bv \neq \bzero$, то любой вектор $\bv' = c \cdot \bv$ также будет собственным:
    \pause
    \[
    \LL \bv' = \LL c \bv = c \LL \bv = c \lambda \bv = \lambda \bv'.
    \]

    \pause
    Система уравнений $\LL \bv = \lambda \bv$ должна иметь бесконечное количество решений!


\end{frame}


\begin{frame}
    \frametitle{Как найти собственные числа?}

    Перепишем систему $\LL \bv = \lambda \bv$ в виде $(\LL - \lambda \Id) \bv = \bzero$.
    
    \pause

    Система имеет бесконечное количество решений, если и только если $\det (\LL - \lambda \Id) = 0$.

    \pause

    \begin{block}{Алгоритм}
        \begin{enumerate}
            \item Из уравнения $\det (\LL - \lambda \Id) = 0$ находим собственные числа $\lambda_1$, \ldots, $\lambda_k$. \pause
            \item Для каждого $\lambda_i$ решаем систему $(\LL - \lambda_i \Id) \bv = \bzero$ относительно $\bv$, то есть находим все собственные векторы. 
        \end{enumerate}
    \end{block}

    
\end{frame}


 \begin{frame}
     \frametitle{Характеристический многочлен}

     \begin{block}{Определение}
         Многочлен $\charp_{\LL}(\lambda) = \det (\LL - \lambda \Id)$ называется
         \alert{характеристическим многочленом} линейного оператора $\LL$.
     \end{block}

     \pause

     Характеристическим многочленом матрицы называется характеристический многочлен соответствующего линейного оператора.
    
 \end{frame}


 \begin{frame}{Характеристический многочлен: пример}

    Рассмотрим матрицу $A = \begin{pmatrix}
        4 & 6 & 0 \\
        6 & 4 & 0 \\
        0 & 0 & 7 \\
    \end{pmatrix}.$

    \pause
    \[
    \charp_A(\lambda) = \det (A - \lambda \Id) = \begin{vmatrix}
4-\lambda & 6 & 0 \\
6 & 4-\lambda & 0 \\
0 & 0 & 7-\lambda \\        
    \end{vmatrix} = \pause
    \]
    \[ = (7-\lambda) \begin{vmatrix}
4-\lambda & 6  \\
6 & 4-\lambda \\
\end{vmatrix} = (7-\lambda)((4-\lambda)^2 - 36) = \pause
\]
\[
 = -(\lambda - 7)(\lambda + 2)(\lambda - 10) = 
 -\lambda^3  + 15\lambda^2   -36 \lambda  - 140  
    \]

 \end{frame}


 \begin{frame}
     \frametitle{Характеристический многочлен}

     По характеристическому многочлену можно найти: \pause

     \begin{enumerate}
         \item Собственные числа $A$ из уравнения $\charp_{A}(\lambda) = 0$.
         \[
            \charp_{A}(\lambda)  = -(\lambda - 7)(\lambda + 2)(\lambda - 10)    
         \]
         \[
         \lambda_1 = 7, \; \lambda_2 = -2, \; \lambda_3 = 10.    
         \]
         
         \pause
         \item Определитель $A$ из равенства $\charp_{A}(0) = \det (A - 0\cdot \Id)$.
         \[
            \charp_{A}(\lambda) =-\lambda^3  + 15\lambda^2   -36 \lambda  - 140 
         \]
         \[
         \det A = \charp_A(0)=-140.
         \]

     \end{enumerate}
 
 
 \end{frame}


 \begin{frame}
     \frametitle{Алгебраическая кратность}

     \begin{block}{Утверждение}
    По основной теореме алгебры любой многочлен $f$ с действительными коэффициентами можно единственным образом представить в виде:
    \[
    f(x) = (x-x_1)^{k_1} \cdot \ldots  \cdot (x-x_p)^{k_p} g(x),
    \]
    где $x_1$, \ldots, $x_p \in \R$ — различные корни многочлена $f$, а многочлен $g$ действительных корней не имеет. 
         
     \end{block}
 
     \pause
     \begin{block}{Определение}
        Число $k_i$ называется \alert{алгебраической кратностью} корня $x_i$.
     \end{block}
 
 \end{frame}


 \begin{frame}
     \frametitle{Алгебраическая кратность: пример}

     Если $\charp_A(\lambda) = -(\lambda - 7)^2(\lambda + 3)$, то 
     собственное число $\lambda = 7$ имеет алгебраическую кратность $2$, 
     а собственное число $\lambda = -3$ имеет алгебраическую кратность $1$.
 
     \pause
     Если $\LL : \R^n \to \R^n$, то сумма алгебраических кратностей $k_i$ 
     действительных собственных чисел $\lambda_i \in \R$
     не превосходит $n$:
    \[
    \sum_{i=1}^p k_i \leq n.
    \]

 \end{frame}



% \begin{frame}
% \frametitle{Немного о комплексных числах}


% \end{frame}


% \begin{frame}
%     \frametitle{Немного о комплексных числах}

%     \begin{block}{Утверждение}
%     По основной теореме алгебры любой многочлен $f$ с действительными коэффициентами можно единственным образом представить в виде:
%     \[
%     f(x) =a (x-x_1)^{k_1}\ldots (x-x_p)^{k_p},
%     \]
%     где $x_1$, \ldots, $x_p \in \mathbb{C}$ — различные корни многочлена $f$. 

%      \end{block}
%     \pause
%     Любой оператор $\LL: \R^n \to \R^n$ имеет собственные числа, если допустить, что $\lambda \in \mathbb{C}$.

%     \pause
% Если $\LL : \R^n \to \R^n$, то сумма алгебраических кратностей $k_i$ 
%  собственных чисел $\lambda_i \in \mathbb{C}$
%  равна $n$, $\sum_{i=1}^p k_i = n$.

    

% \end{frame}

 
\begin{frame}
    \frametitle{Теорема Гамильтона-Кэли}

    \begin{block}{Утверждение}
        Если подставить матрицу $A$ в характеристический многочлен $\charp_A(\lambda)$, то получится матрица из нулей,
        \[
        \charp_A(A) = \bzero;     
        \]
    \end{block}

%    \pause

 %   Эта теорема позволяет, например, снижать степень матрицы. 

    \pause
    \vspace{10pt}
    Пример. Если $\charp_A(\lambda) = \lambda^2 - 3\lambda + 8$, то $A^2 - 3A + 8\Id = \bzero$ и 
    $A^2 = 3A - 8\Id$. 
    

\end{frame}

\begin{frame}
\lecturetitle{Нахождение собственных чисел и векторов}
\todo{Это видеофрагмент с доской, слайдов здесь нет :)}
\end{frame}
    

% !TEX root = ../linal_lecture_04.tex

\begin{frame} % название фрагмента

\videotitle{Диагонализация матрицы}

\end{frame}



\begin{frame}{Краткий план:}
  \begin{itemize}[<+->]
    \item Собственные векторы как линейное пространство. 
    \item Геометрическая кратность собственных чисел.
    \item Диагонализация матрицы.
  \end{itemize}

\end{frame}


\begin{frame}
    \frametitle{Множество собственных векторов}

    Оператор $\LL :\R^n \to \R^n$ имеет собственное число $\lambda \in \R$.
    
    Рассмотрим множество $\eig_\lambda$ — множество всех собственных векторов, растягивающихся 
    в $\lambda$ раз, дополненное нулевым вектором $\bzero$:
    \[
        \eig_\lambda \LL = \{ \bv \mid \LL \bv = \lambda \bv \}.
    \]

    \pause

    \begin{block}{Утверждение}
        Множество $\eig_{\lambda} \LL$ является векторным пространством:
        \pause

        Если вектор $\bv$ растягивается в $\lambda$ раз, то и вектор $t \bv$ растягивается в $\lambda$ раз.
        \pause

       Если векторы $\ba$ и $\bb$ растягивается в $\lambda$ раз, 
       то и их сумма $\bc = \ba + \bb$ растягивается в $\lambda$ раз.
    \end{block}

    

\end{frame}


\begin{frame}
    \frametitle{Геометрическая кратность}

    \begin{block}{Определение}
        Размерность пространства $\eig_\lambda \LL$ называется
        \alert{геометрической кратностью} собственного числа $\lambda \in \R$.        
    \end{block}

    \pause
    \begin{block}{Эквивалентное определение}
        Максимальное количество линейно независимых собственных векторов,
        соответствующих собственному числу $\lambda  \in \R$ называют его 
        \alert{геометрической кратностью}.        
    \end{block}

\end{frame}


\begin{frame}
\frametitle{Разные кратности связаны!}

\begin{block}{Утверждение}
    Геометрическая кратность собственного числа $\lambda\in \R$ не превосходит его алгебраической кратности и не меньше единицы.     
\end{block}

\pause
Пример. У матрицы $A$ характеристический многочлен равен $\charp_A(\lambda) = -(\lambda- 7)(\lambda - 9)^2$. \pause

Числу $\lambda =7$ соответствует ровно один линейно независимый собственный вектор.\pause

Числу $\lambda =9$ соответствуют один или два линейно независимых собственных вектора. 

\end{frame}


\begin{frame}
    \frametitle{Независимость собственных векторов}

    \begin{block}{Утверждение}
        Если векторы набора $A = \{\bv_1$, $\bv_2$, \ldots, $\bv_k\}$ относятся к различным
        собственным числам, то набор $A$ линейно независимый.
    \end{block}
    \pause
    \begin{block}{Идея доказательства}
        Пусть вектора $\bv_1$, $\bv_2$ и $\bv_3$ растягиваются в $2$, $3$ и $8$ раз соответственно, 
        и $\bv_3 = 7\bv_1 - 4 \bv_2$.\pause

        Домножим $A$ на обе части равенства, $8\bv_3 = 2 \cdot 7\bv_1 - 3\cdot 4\bv_2$.\pause

        Поделим на большее собственное число, $\bv_3 = \frac{2}{8} \cdot 7\bv_1 - \frac{3}{8}\cdot 4\bv_2$. \pause
        
        Повторим бесконечно много раз, $\bv_3 = \bzero$.
        Противоречие.
        
    \end{block}

    

\end{frame}




\begin{frame}
    \frametitle{Базис из собственных векторов}
    Векторы отвечающие различным собственным числам независимы.
    \pause 

    В каждом пространстве $\eig_{\lambda_i}\LL$ найдётся базис из
     $\gamma_i = \dim \eig_{\lambda_i}\LL$ собственных векторов.
     \pause

    \begin{block}{Утверждение}
Если $\sum_i \gamma_i = n$, то в $\R^n$ существует базис 
из $n$ векторов, являющихся собственными векторами оператора $\LL$.            
    \end{block}

\end{frame}


\begin{frame}
    \frametitle{Диагонализация: обозначения}
    Допустим, у оператора $\LL:\R^n\to\R^n$ нашлось $n$ линейно независимых 
    собственных векторов $\{\bv_1, \bv_2, \ldots, \bv_n\}$, которым соответствуют
    собственные числа $\{ \lambda_1, \lambda_2, \ldots, \lambda_n\}$.
    \pause

    Запишем все собственные векторы в матрицу $P$ столбцами друг за другом. 

    А в матрицу $D$ поместим все собственные числа на главную диагональ.



    \[
        P = \begin{pmatrix}
            \vert & \vert &   & \vert \\
            \bv_1 & \bv_2 & \ldots & \bv_n \\
            \vert & \vert &  & \vert \\
        \end{pmatrix}, \;
        D = \begin{pmatrix}
            \lambda_1 & 0 & \ldots & 0 \\
            0 & \lambda_2 & \ldots & 0 \\
            0 & 0 & \cdots &  0 \\
            0 & 0 & \ldots & \lambda_n \\
        \end{pmatrix}
    \]
\end{frame}    



\begin{frame}
\frametitle{Диагонализация: мне повезёт!}
    

    \begin{block}{Утверждение}
        Если у оператора $\LL:\R^n\to\R^n$ нашлось $n$ линейно независимых 
        собственных векторов $\{\bv_1, \bv_2, \ldots, \bv_n\}$,
        то $\LL$ представим в виде
        \[
        \LL = PDP^{-1}.    
        \]
    \end{block}
    \pause

    \begin{block}{Доказательство}
        Заметим, что $P \be_i = \bv_i$, и $\LL P \be_i = \lambda_i P \be_i$.
        \pause

        Домножаем на $P^{-1}$ и получаем $P^{-1}\LL P \be_i = \lambda_i \be_i$.
        \pause 

        Диагональная матрица растягивает базисные вектора, $P^{-1}\LL P \be_i = D \be_i$.
        \pause
        \[
        D = P^{-1} \LL P, \text{ или } \LL = P D P^{-1}    
        \]
    
    \end{block}
    

\end{frame}    





\begin{frame}
\lecturetitle{Диагонализация матрицы}
\todo{Это видеофрагмент с доской, слайдов здесь нет :)}
\end{frame}

% !TEX root = ../linal_lecture_04.tex

\begin{frame} % название фрагмента

\videotitle{След матрицы}

\end{frame}



\begin{frame}{Краткий план:}
  \begin{itemize}[<+->] 
    \item Сумма диагональных элементов.
    \item Свойства следа.
  \end{itemize}

\end{frame}


\begin{frame}{След квадратной матрицы}
    \begin{block}{Определение}
        \alert{Следом квадратной матрицы} $\LL$ называют сумму её диагональных элементов. 
        \[
            \trace \LL = \ell_{11} + \ell_{22} + \ldots + \ell_{nn}
        \]
    \end{block}

    \pause
    Пример. $\trace \begin{pmatrix}
        4 & 6 \\
        9 & 1
    \end{pmatrix} = 4 + 1 = 5$.

\end{frame}





\begin{frame}
    \frametitle{Основное свойство следа}

    \onslide<1->{\begin{block}{Утверждение}
        Если матрицы $A$ и $B$ имеют размер $n\times k$, то
        \[
        \trace A^T B = \sum_{ij} a_{ij} b_{ij} = \trace B^T A
        \]

    \end{block}}

    \begin{overlayarea}{\textwidth}{0.4\textheight}

    \only<2>{Пример. $A = \begin{pmatrix}
        a_1 & a_2 \\
        a_3 & a_4 \\ 
    \end{pmatrix},  \;
    B = \begin{pmatrix}
        b_1 & b_2 \\
        b_3 & b_4 \\ 
    \end{pmatrix}$.

    \[
    \trace A^T B = a_1 b_1 + a_2 b_2 + a_3 b_3 + a_4 b_4    
    \]} %
    \only<3>{\begin{block}{Доказательство}
        \[
        \trace A^T B = \sum_i \langle \row_i A^T, \col_i B \rangle =
        \]
        \[ 
         = \sum_i \langle \col_i A, \col_i B \rangle = \sum_{ij} a_{ij} b_{ij}
        \]
    \end{block}}
    \end{overlayarea}


\end{frame}


% \begin{frame}
%     \frametitle{Основное свойство следа}

% \begin{block}{Утверждение}
%     Если матрицы $A$ и $B$ имеют размер $n\times k$, то
%     \[
%     \trace A^T B = \sum_{ij} a_{ij} b_{ij} = \trace B^T A
%     \]

% \end{block}
% \pause

    
    

% \end{frame}


\begin{frame}
    \frametitle{И ещё немного свойств}

    Если $A$ имеет размер $n\times k$, а $B$ — размер $k\times n$, то:
    \[
    \trace AB = \trace BA    
    \]
    \pause

    След — линейный оператор, превращающий матрицы размера $n\times n$ в числа!
    \pause

    \[
    \trace \lambda A = \lambda \trace A    
    \]


    \[
    \trace (A+B) = \trace A + \trace B
    \]


\end{frame}


\begin{frame}
    \frametitle{Зачем нужен след?}

    \pause
    Элегантно позволяет записывать сложные выражения.
    
    \[
    \sum_{ij} a_{ij}^2 = \trace A^T A    
    \]

    \pause
    
    Упрощает теоретические выкладки. 

\end{frame}






% !TEX root = ../linal_lecture_04.tex

\begin{frame} % название фрагмента

\videotitle{Вокруг собственных чисел}

\end{frame}



\begin{frame}{Краткий план:}
  \begin{itemize}[<+->]
    \item Реинкарнация теоремы Виета. 
    \item Обратимость и собственные числа.
    \item Собственные числа проектора.
  \end{itemize}

\end{frame}




\begin{frame}
    \frametitle{Вид характеристического многочлена}


    \begin{block}{Утверждение}
        Характеристический многочлен $\charp_A(\lambda)$ матрицы $A$ размера $n\times n$ имеет вид
        \[
            \charp_A(\lambda) = (-1)^n (\lambda^n - \trace A \lambda^{n-1}) + \ldots + \det A
        \]
        
    \end{block}
    \pause

    Пример. 
    \[
        A= \begin{pmatrix}
            4 & 5 \\
            1 & 2 \\
        \end{pmatrix}, \; \charp_A(\lambda) = \lambda^2 - 6 \lambda + 3
    \]
    \pause
    Пример.
\[
    B= \begin{pmatrix}
        4 & 5 & 0 \\
        1 & 2 & 0 \\
        4 & 3 & 7 \\
    \end{pmatrix}, \; \charp_B(\lambda) = -\lambda^3 + 13 \lambda^2 + \ldots + 21
\]


\end{frame}


\begin{frame}{Реинкарнация теоремы Виета}
\begin{block}{Утверждение}
    Если у матрицы $A$ размера $n\times n$ ровно $n$ действительных собственных чисел $\lambda_1$, 
    $\lambda_2$, \ldots, $\lambda_n$, то \pause
    \[
    \trace A = \sum_{i=1}^n \lambda_i; \pause
    \]
    \[
    \det A = \prod_{i=1}^n \lambda_i.   
    \]
\end{block}

\end{frame}

\begin{frame}
    \frametitle{Пополним критерий вырожденности!}

    Матрица $A$ размера $n\times n$ называется \alert{вырожденной}, если:

    \begin{enumerate}
        \item $\det A = 0$; 
        \item Система $A \bx = \bzero$ имеет бесконечное количество решений; 
        \item Система $A \bx = \bb$ имеет ноль или бесконечное количество решений; 
        \item $\rank A < n$; 
        \item Столбцы $A$ линейно зависимы; 
        \item Строки $A$ линейно зависимы; 
        \item $A^{-1}$ не существует; 
        \item \alert{У матрицы $A$ есть $\lambda=0$.}
    \end{enumerate}


\end{frame}



\begin{frame}
    \frametitle{Собственные числа проектора}

    Оператор $\HH: \R^n \to \R^n$ проецирует $\R^n$ на некоторую линейную оболочку $M$. \pause
    
    \begin{block}{Утверждение}
        Собственные числа проектора $H$ равны $0$ или $1$. \pause
        
        Собственными векторами c $\lambda=0$ будут векторы, ортогональные $M$. \pause

        Собственными векторами c $\lambda=1$ будут векторы из  $M$. \pause

        У проектора ровно $n$ линейно независимых собственных векторов. 
    \end{block}

    

\end{frame}


\begin{frame}
\frametitle{Ранг и след проектора}

Оператор $\HH: \R^n \to \R^n$ проецирует $\R^n$ на некоторую линейную оболочку $M$. \pause

    Ранг проектора — число элементов в базисе $M$. \pause

    След проектора — кратность собственного числа $\lambda=1$. \pause

    \begin{block}{Утверждение}
        Для проектора $\HH$ след и ранг равны размерности множества, на которое проецирует $\HH$,
        \[
        \rank \HH = \trace \HH.    
        \]
    \end{block}


\end{frame}


% !TEX root = ../linal_lecture_04.tex

\begin{frame} % название фрагмента

\videotitle{Комплексные собственные числа}

\end{frame}



\begin{frame}{Краткий план:}
  \begin{itemize}[<+->] 
    \item Комплексные числа.
    \item Основная теорема алгебры.
    \item Снова след и определитель.
  \end{itemize}

\end{frame}



\begin{frame}{Комплексные числа как мистика}

    \begin{block}{}
        Множество $\CC$ вида
        \[
        \CC = \{ a+ bi \mid a, b \in \R \}    
        \]
        с естественным сложением и умножением по правилу $i^2 = -1$ называется множеством \alert{комплексных чисел}.
    \end{block}

    \pause
    Пример.
    \[
    (5 + 6i) + (2 + i) = 7 + 7i \pause 
    \]
    \[
        (5 + 6i)(2+i) = 10 + 17i + 6i^2 = 10 - 6 + 17i = 4 + 17i \pause    
    \]
    \[
    \frac{5+6i}{2-i } =     \frac{(5+6i)(2+i)}{(2-i)(2+i)} = \frac{4+17i}{4 - i^2}= \frac{4}{5} + \frac{17}{5}i
    \]

\end{frame}

\begin{frame}{Комплексные числа как операторы}

    \begin{block}{Идея}
    Комплексное число $a + bi$ — способ записывать повороты плоскости,
    растяжения плоскости и композиции этих действий.    
    \end{block}

    \pause
    $a+ bi \; \leftrightarrow \;$ преобразование плоскости! 

\end{frame}


\begin{frame}{Комплексные числа как операторы}

\begin{center}


\begin{tikzpicture}[
scale=1.8,
MyPoints/.style={draw=blue,fill=white,thick},
Segments/.style={draw=blue!50!red!70,thick},
MyCircles/.style={green!50!blue!50,thin}, 
every node/.style={scale=1}
]
%\draw[color=gray,step=1.0,dotted] (-1.9,-0.9) grid (5.5,6.5); 
\clip (-1,-1.5) rectangle (6.5,3.5);

%{\verb!->!new, arrowhead = 2mm, line width=4pt}
%, arrowhead = 3mm
%, arrowhead = 0.2


%\draw[->, >=stealth] (-1,0)--(6.5,0) node[right]{$x_1$};
\draw[-{Latex[length=4.5mm, width=2.5mm]}, >=stealth] (0,-0.5)--(0,3) node[left]{$\operatorname{Im}$};

\draw[-{Latex[length=4.5mm, width=2.5mm]}, >=stealth] (-0.5,0)--(6,0) 
node[right]{$\operatorname{Re}$};



% Feel free to change here coordinates of points A and B
\pgfmathparse{0}		\let\Xa\pgfmathresult
\pgfmathparse{0}		\let\Ya\pgfmathresult
\coordinate (A) at (\Xa,\Ya);

\pgfmathparse{0}		\let\Xb\pgfmathresult
\pgfmathparse{2}		\let\Yb\pgfmathresult
\coordinate (B) at (\Xb,\Yb);

\pgfmathparse{4}		\let\Xc\pgfmathresult
\pgfmathparse{2}		\let\Yc\pgfmathresult
\coordinate (C) at (\Xc,\Yc);

\pgfmathparse{4}		\let\Xd\pgfmathresult
\pgfmathparse{0}		\let\Yd\pgfmathresult
\coordinate (D) at (\Xd,\Yd);

\pgfmathparse{1}		\let\Xe\pgfmathresult
\pgfmathparse{0}		\let\Ye\pgfmathresult
\coordinate (E) at (\Xe,\Ye);



\draw[-{Latex[length=4.5mm, width=2.5mm]}, >=stealth,  veca] (A)--(C) node[above right]{$\ba+\bb i$};


\draw (A)--(B) node[left]{$\bb$} ;

\draw (A)--(D) node[below]{$\ba$};

\draw[-{Latex[length=4.5mm, width=2.5mm]}, >=stealth,  vecb] (A)--(E) node[below]{$\left(\begin{array}{l}1 \\ 0\end{array}\right)$};


\draw[black, dashed, thick] (B)--(C);
\draw[black, dashed, thick] (D)--(C);



\end{tikzpicture}

    \end{center}
        
\pause

Число $a+bi$ кодирует преобразование плоскости $\begin{pmatrix}
    1 \\
    0 \\
\end{pmatrix} \to \begin{pmatrix}
    a \\
    b \\
\end{pmatrix}$.

\end{frame}


\begin{frame}{Никакой мистики!}


Поворот на $90^{\circ}$:
\[
 \begin{pmatrix}
    1 \\
    0 \\
\end{pmatrix} \to \begin{pmatrix}
    0 \\
    1 \\
\end{pmatrix}   \leftrightarrow 0 + 1\cdot i = i \pause
\]

    Растягивание в 7 раз:
\[
 \begin{pmatrix}
    1 \\
    0 \\
\end{pmatrix} \to \begin{pmatrix}
    7 \\
    0 \\
\end{pmatrix}   \leftrightarrow 7 + 0\cdot i = 7 \pause
\]
       
    Растягивание в $\sqrt{2}$ раз и 
    вращение на $45^\circ$:
\[
 \begin{pmatrix}
    1 \\
    0 \\
\end{pmatrix} \to \begin{pmatrix}
    1 \\
    1 \\
\end{pmatrix}   \leftrightarrow 1 + 1\cdot i = 1 + i
\]
\end{frame}







\begin{frame}{Никакой мистики!}


    Поворот на $90^{\circ}$:
    \[
     \begin{pmatrix}
        1 \\
        0 \\
    \end{pmatrix} \to \begin{pmatrix}
        0 \\
        1 \\
    \end{pmatrix}   \leftrightarrow 0 + 1\cdot i = i 
    \]

    Два поворота подряд на $90^{\circ}$:
    \[
    \begin{pmatrix}
        1 \\
        0 \\
    \end{pmatrix} \to \begin{pmatrix}
        -1 \\
        0 \\
    \end{pmatrix} \leftrightarrow  -1 + 0i = -1  \pause
    \]

    Если повернуться на $90^{\circ}$, а затем повернуться ещё на $90^{\circ}$, то
        развернёшься в обратную сторону, $\; i \cdot i = -1$.

\end{frame}



\begin{frame}{Комплексные числа как операторы}

\begin{block}{Определение}
    Множество $\CC$ преобразований плоскости, включающее повороты плоскости, 
    растяжения плоскости в произвольное количество раз и композиции этих двух действий,
    называется множеством \alert{комплексных чисел}.
\end{block}

\pause
Растягивание в 7 раз $\leftrightarrow 7$.

\pause
Поворот на $90^{\circ} \; \leftrightarrow i$.

\pause
Для $z\in\CC$ определяют:

Модуль $\abs{z}$ — во сколько раз изменяется длина вектора. 

Аргумент $\arg{z}$ — на сколько изменяется угол вектора. 

\end{frame}
    


% \begin{frame}{Комплексные числа как матрицы}

% \begin{block}{Эквивалентное пределение}
%     Множество $\CC$ матриц вида $\begin{pmatrix}
%         r\cos \phi & -r\sin \phi \\
%         r\sin \phi & r\cos \phi \\
%     \end{pmatrix}$ называется множеством \alert{комплексных чисел}.
% \end{block}

% \pause 
% Все матрицы в $\CC$ имеют вид $\begin{pmatrix}
%     a  & -b  \\
%     b  & a  \\
% \end{pmatrix}$.

% \pause 
% Для краткости вместо матрицы 
% $\begin{pmatrix}
%     a  & -b  \\
%     b  & a  \\
% \end{pmatrix}$ пишут $a+bi$.

% \end{frame}


% \begin{frame}
%     \frametitle{Комплексные числа как образ $\be_1$}

    
%     \[
%         \begin{pmatrix}
%         a  & -b  \\
%         b  & a  \\
%     \end{pmatrix} \cdot \begin{pmatrix}
%         1 \\
%         0 
%     \end{pmatrix} = 
%     \begin{pmatrix}
%         a \\
%         b 
%     \end{pmatrix}
%     \]


% \end{frame}

    





\begin{frame}
    \frametitle{Основная теорема алгебры}
    \begin{block}{Утверждение}
        Любой многочлен $f(z)$ степени $n$ имеет ровно $n$ корней,
        если считать корни $z\in \CC$ с учётом алгебраической кратности.
\pause
\[
f(z) = a(z-z_1)(z-z_2)\cdot \ldots \cdot (z-z_n)    
\]
    \end{block}    

    \pause
    \begin{block}{Следствие}
        У любой квадратной матрицы размера $n\times n$ найдётся ровно $n$ 
        комплексных собственных чисел $\lambda \in \CC$ с учётом алгебраической кратности.
    \end{block}


\end{frame}




\begin{frame}
    \frametitle{След линейного оператора}

    \begin{block}{Определение}
        \alert{Следом} линейного оператора $\LL:\R^n\to\R^n$ называют сумму всех его комплексных собственных чисел $\lambda_i \in \CC$,
        \[
        \trace \LL = \lambda_1 + \lambda_2 + \ldots + \lambda_n.
        \]
    \end{block}

    \pause
    Пример. Если $\charp_{A}(\lambda) = -(\lambda-1)(\lambda-5)^2$, то $\trace A = 1 + 5 + 5 = 11$.

    \pause
    Пример. Если $\charp_{A}(\lambda) = -(\lambda-1)(\lambda-2+3i)(\lambda-2-3i)$, то $\trace A = 1 + (2-3i) + (2+3i) = 5$.

\end{frame}



\begin{frame}
    \frametitle{Определитель линейного оператора}

    \begin{block}{Утверждение}
        Определитель линейного оператора $\LL:\R^n\to\R^n$ равен произведению всех его комплексных собственных чисел $\lambda_i \in \CC$,
        \[
        \det \LL = \lambda_1 \cdot \lambda_2 \cdot \ldots \cdot \lambda_n.
        \]
    \end{block}


\end{frame}





\begin{frame}
\lecturetitle{Нахождение проектора}
\todo{Это видеофрагмент с доской, слайдов здесь нет :)}
\end{frame}
    

\begin{frame}
\lecturetitle{Прогнозирование с помощью мнк}
\todo{Это скринкаст, слайдов здесь нет :)}
\end{frame}
    

\begin{frame}
\lecturetitle{Бонус: задача про Чабана и 101 овцу}
\todo{Это видеофрагмент с доской, слайдов здесь нет :)}
\end{frame}
    


\end{document}
