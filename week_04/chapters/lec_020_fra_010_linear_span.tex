% !TEX root = ../linal_lecture_04.tex

\begin{frame} % название фрагмента

\videotitle{Линейная оболочка}

\end{frame}



\begin{frame}{Краткий план:}
  \begin{itemize}[<+->]
    \item Вектор — это столбец чисел.
    \item Сложение двух векторов и умножение на число.
    \item Расстояние и косинус угла между векторами.
  \end{itemize}

\end{frame}


\begin{frame}{Линейная комбинация}

\begin{itemize}[<+->]
  \item Определение. Вектор $\bv$ называется \alert{линейной комбинацией} векторов $\bx_1$, $\bx_2$, \ldots, $\bx_k$, 
если его можно представить в виде их суммы с некоторыми действительными весами $\alpha_i$:

\[
  \bv = \alpha_1 \bx_1 + \alpha_2 \bx_2 + \ldots + \alpha_k \bx_k
\]

\item Пример. Вектор $\begin{pmatrix}
  4 \\
  5 \\
\end{pmatrix}$ — это линейная комбинация векторов $\begin{pmatrix}
  1 \\
  0 \\
\end{pmatrix}$ и $\begin{pmatrix}
  1 \\
  1 \\
\end{pmatrix}$:

\[
\begin{pmatrix}
  4 \\
  5 \\
\end{pmatrix} = -1 \begin{pmatrix}
  1 \\
  0 \\
\end{pmatrix} + 5 \begin{pmatrix}
  1 \\
  1 \\
\end{pmatrix}  
\]


\end{itemize}


\end{frame}



\begin{frame}{Любой вектор — линейная комбинация}


\begin{itemize}[<+->]
  \item Любой вектор $\bv \in \R^2$ — линейная комбинация векторов $\begin{pmatrix}
    1 \\
    0 \\
  \end{pmatrix}$ и $\begin{pmatrix}
    0 \\
    1 \\
  \end{pmatrix}$:

\[
\begin{pmatrix}
  v_1 \\
  v_2 \\
\end{pmatrix} = 
v_1 \begin{pmatrix}
    1 \\
    0 \\
  \end{pmatrix} + 
  v_2 \begin{pmatrix}
    0 \\
    1 \\
  \end{pmatrix}
\]

\item Аналогично, для вектора  $\bv \in \R^3$:

\[
\begin{pmatrix}
v_1 \\
v_2 \\
v_3 \\
\end{pmatrix} = 
v_1 \begin{pmatrix}
  1 \\
  0 \\
  0 \\
\end{pmatrix} + 
v_2 \begin{pmatrix}
  0 \\
  1 \\
  0 \\
\end{pmatrix} +
v_3 \begin{pmatrix}
  0 \\
  0 \\
  1 \\
\end{pmatrix} 
\]



\end{itemize}
  

\end{frame}



\begin{frame}
  \frametitle{Линейная зависимость}


  Определение.

  \begin{itemize}[<+->]
    \item Набор $A$ из двух и более векторов называется 
    \alert{линейно зависимым}, если хотя бы один вектор является линейной комбинацией остальных.
    \item Набор $A$ из одного нулевого вектора также называется \alert{линейно зависимым}.
  \end{itemize}
  

\end{frame}


\begin{frame}
  \frametitle{Линейная зависимость: пример}




  \begin{itemize}[<+->]
    \item Набор $A = \left\{ \begin{pmatrix}
      0 \\
      2 \\
    \end{pmatrix}, \begin{pmatrix}
      3 \\
      4 \\
    \end{pmatrix} \right\}$ — линейно независимый.
    \item Набор $A = \left\{ \begin{pmatrix}
      0 \\
      2 \\
      0 \\
    \end{pmatrix}, \begin{pmatrix}
      3 \\
      4 \\
      0 \\
    \end{pmatrix},
    \begin{pmatrix}
      1 \\
      0 \\
      0 \\
    \end{pmatrix} \right\}$ — линейно зависимый:

    \[
      \begin{pmatrix}
        3 \\
        4 \\
        0 \\
      \end{pmatrix} = 2
    \begin{pmatrix}
      0 \\
      2 \\
      0 \\
    \end{pmatrix} + 3
    \begin{pmatrix}
      1 \\
      0 \\
      0 \\
    \end{pmatrix}  
    \]


  \end{itemize}
  

\end{frame}


\begin{frame}
  \frametitle{Линейная зависимость: дубль два}

  Определение-2. Набор векторов $A = \{ \bv_1, \bv_2, \ldots, \bv_k\}$ называется \alert{линейно зависимым},
  если можно найти такие числа $\alpha_1$, $\alpha_2$, \ldots, $\alpha_k$, что
  не все из них равны нулю и 

  \[
  \alpha_1 \bv_1 + \alpha_2 \bv_2 + \ldots + \alpha_k \bv_k = \bzero  
  \]

  
\end{frame}