% !TEX root = ../linal_lecture_04.tex

\begin{frame} % название фрагмента

\videotitle{Комплексные собственные числа}

\end{frame}



\begin{frame}{Краткий план:}
  \begin{itemize}[<+->] 
    \item Комплексные числа.
    \item Основная теорема алгебры.
  \end{itemize}

\end{frame}



\begin{frame}{Комплексные числа как мистика}

    \begin{block}{}
        Множество $\CC$ вида
        \[
        \CC = \{ a+ bi \mid a, b \in \R \}    
        \]
        с естественным сложением и умножением по правилу $i^2 = -1$ называется множеством \alert{комплексных чисел}.
    \end{block}

    \pause
    Пример.
    \[
    (5 + 6i) + (2 + i) = 7 + 7i \pause 
    \]
    \[
        (5 + 6i)(2+i) = 10 + 17i + 6i^2 = 10 - 6 + 17i = 4 + 17i \pause    
    \]
    \[
    \frac{5+6i}{2-i } =     \frac{(5+6i)(2+i)}{(2-i)(2+i)} = \frac{4+17i}{4 - i^2}= \frac{4}{5} + \frac{17}{5}i
    \]

\end{frame}

\begin{frame}{Комплексные числа как операторы}

    \begin{block}{Идея}
    Комплексное число $a + bi$ — способ записывать повороты плоскости,
    растяжения плоскости и композиции этих действий.    
    \end{block}

    \pause
    $a+ bi \; \leftrightarrow \;$ преобразование плоскости! 

\end{frame}


\begin{frame}{Комплексные числа как операторы}

\begin{center}


\begin{tikzpicture}[
scale=1.8,
MyPoints/.style={draw=blue,fill=white,thick},
Segments/.style={draw=blue!50!red!70,thick},
MyCircles/.style={green!50!blue!50,thin}, 
every node/.style={scale=1}
]
%\draw[color=gray,step=1.0,dotted] (-1.9,-0.9) grid (5.5,6.5); 
\clip (-1,-1.5) rectangle (6.5,3.5);

%{\verb!->!new, arrowhead = 2mm, line width=4pt}
%, arrowhead = 3mm
%, arrowhead = 0.2


%\draw[->, >=stealth] (-1,0)--(6.5,0) node[right]{$x_1$};
\draw[-{Latex[length=4.5mm, width=2.5mm]}, >=stealth] (0,-0.5)--(0,3) node[left]{$\operatorname{Im}$};

\draw[-{Latex[length=4.5mm, width=2.5mm]}, >=stealth] (-0.5,0)--(6,0) 
node[right]{$\operatorname{Re}$};



% Feel free to change here coordinates of points A and B
\pgfmathparse{0}		\let\Xa\pgfmathresult
\pgfmathparse{0}		\let\Ya\pgfmathresult
\coordinate (A) at (\Xa,\Ya);

\pgfmathparse{0}		\let\Xb\pgfmathresult
\pgfmathparse{2}		\let\Yb\pgfmathresult
\coordinate (B) at (\Xb,\Yb);

\pgfmathparse{4}		\let\Xc\pgfmathresult
\pgfmathparse{2}		\let\Yc\pgfmathresult
\coordinate (C) at (\Xc,\Yc);

\pgfmathparse{4}		\let\Xd\pgfmathresult
\pgfmathparse{0}		\let\Yd\pgfmathresult
\coordinate (D) at (\Xd,\Yd);

\pgfmathparse{1}		\let\Xe\pgfmathresult
\pgfmathparse{0}		\let\Ye\pgfmathresult
\coordinate (E) at (\Xe,\Ye);



\draw[-{Latex[length=4.5mm, width=2.5mm]}, >=stealth,  veca] (A)--(C) node[above right]{$\ba+\bb i$};


\draw (A)--(B) node[left]{$\bb$} ;

\draw (A)--(D) node[below]{$\ba$};

\draw[-{Latex[length=4.5mm, width=2.5mm]}, >=stealth,  vecb] (A)--(E) node[below]{$\left(\begin{array}{l}1 \\ 0\end{array}\right)$};


\draw[black, dashed, thick] (B)--(C);
\draw[black, dashed, thick] (D)--(C);



\end{tikzpicture}

    \end{center}
        
\pause

Число $a+bi$ кодирует преобразование плоскости $\begin{pmatrix}
    1 \\
    0 \\
\end{pmatrix} \to \begin{pmatrix}
    a \\
    b \\
\end{pmatrix}$.

\end{frame}


\begin{frame}{Никакой мистики!}


Поворот на $90^{\circ}$:
\[
 \begin{pmatrix}
    1 \\
    0 \\
\end{pmatrix} \to \begin{pmatrix}
    0 \\
    1 \\
\end{pmatrix}   \leftrightarrow 0 + 1\cdot i = i \pause
\]

    Растягивание в 7 раз:
\[
 \begin{pmatrix}
    1 \\
    0 \\
\end{pmatrix} \to \begin{pmatrix}
    7 \\
    0 \\
\end{pmatrix}   \leftrightarrow 7 + 0\cdot i = 7 \pause
\]
       
    Растягивание в $\sqrt{2}$ раз и 
    вращение на $45^\circ$:
\[
 \begin{pmatrix}
    1 \\
    0 \\
\end{pmatrix} \to \begin{pmatrix}
    1 \\
    1 \\
\end{pmatrix}   \leftrightarrow 1 + 1\cdot i = 1 + i
\]
\end{frame}







\begin{frame}{Никакой мистики!}


    Поворот на $90^{\circ}$:
    \[
     \begin{pmatrix}
        1 \\
        0 \\
    \end{pmatrix} \to \begin{pmatrix}
        0 \\
        1 \\
    \end{pmatrix}   \leftrightarrow 0 + 1\cdot i = i 
    \]

    Два поворота подряд на $90^{\circ}$:
    \[
    \begin{pmatrix}
        1 \\
        0 \\
    \end{pmatrix} \to \begin{pmatrix}
        -1 \\
        0 \\
    \end{pmatrix} \leftrightarrow  -1 + 0i = -1  \pause
    \]

    Если повернуться на $90^{\circ}$, а затем повернуться ещё на $90^{\circ}$, то
        развернёшься в обратную сторону, $\; i \cdot i = -1$.

\end{frame}



\begin{frame}{Комплексные числа как операторы}

\begin{block}{Определение}
    Множество $\CC$ преобразований плоскости, включающее повороты плоскости, 
    растяжения плоскости в произвольное количество раз и композиции этих двух действий,
    называется множеством \alert{комплексных чисел}.
\end{block}

\pause
Растягивание в 7 раз $\leftrightarrow 7$.

\pause
Поворот на $90^{\circ} \; \leftrightarrow i$.

\pause
Для $z\in\CC$ определяют:

Модуль $\abs{z}$ — во сколько раз изменяется длина вектора. 

Аргумент $\arg{z}$ — на сколько изменяется угол вектора. 

\end{frame}
    


% \begin{frame}{Комплексные числа как матрицы}

% \begin{block}{Эквивалентное пределение}
%     Множество $\CC$ матриц вида $\begin{pmatrix}
%         r\cos \phi & -r\sin \phi \\
%         r\sin \phi & r\cos \phi \\
%     \end{pmatrix}$ называется множеством \alert{комплексных чисел}.
% \end{block}

% \pause 
% Все матрицы в $\CC$ имеют вид $\begin{pmatrix}
%     a  & -b  \\
%     b  & a  \\
% \end{pmatrix}$.

% \pause 
% Для краткости вместо матрицы 
% $\begin{pmatrix}
%     a  & -b  \\
%     b  & a  \\
% \end{pmatrix}$ пишут $a+bi$.

% \end{frame}


% \begin{frame}
%     \frametitle{Комплексные числа как образ $\be_1$}

    
%     \[
%         \begin{pmatrix}
%         a  & -b  \\
%         b  & a  \\
%     \end{pmatrix} \cdot \begin{pmatrix}
%         1 \\
%         0 
%     \end{pmatrix} = 
%     \begin{pmatrix}
%         a \\
%         b 
%     \end{pmatrix}
%     \]


% \end{frame}

    





\begin{frame}
    \frametitle{Основная теорема алгебры}
    \begin{block}{Утверждение}
        Любой многочлен $f(z)$ степени $n$ имеет ровно $n$ корней,
        если считать корни $z\in \CC$ с учётом алгебраической кратности.
\pause
\[
f(z) = a(z-z_1)(z-z_2)\cdot \ldots \cdot (z-z_n)    
\]
    \end{block}    

    \pause
    \begin{block}{Следствие}
        У любой квадратной матрицы размера $n\times n$ найдётся ровно $n$ 
        комплексных собственных чисел $\lambda \in \CC$ с учётом алгебраической кратности.
    \end{block}


\end{frame}




\begin{frame}
    \frametitle{След линейного оператора}

    \begin{block}{Определение}
        \alert{Следом} линейного оператора $\LL:\R^n\to\R^n$ называют сумму всех его комплексных собственных чисел $\lambda_i \in \CC$,
        \[
        \trace \LL = \lambda_1 + \lambda_2 + \ldots + \lambda_n.
        \]
    \end{block}

    \pause
    Пример. Если $\charp_{A}(\lambda) = -(\lambda-1)(\lambda-5)^2$, то $\trace A = 1 + 5 + 5 = 11$.

    \pause
    Пример. Если $\charp_{A}(\lambda) = -(\lambda-1)(\lambda-2+3i)(\lambda-2-3i)$, то $\trace A = 1 + (2-3i) + (2+3i) = 5$.

\end{frame}



\begin{frame}
    \frametitle{Определитель линейного оператора}

    \begin{block}{Утверждение}
        Определитель линейного оператора $\LL:\R^n\to\R^n$ равен произведению всех его комплексных собственных чисел $\lambda_i \in \CC$,
        \[
        \det \LL = \lambda_1 \cdot \lambda_2 \cdot \ldots \cdot \lambda_n.
        \]
    \end{block}


\end{frame}

