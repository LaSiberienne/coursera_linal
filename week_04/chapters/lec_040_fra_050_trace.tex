% !TEX root = ../linal_lecture_04.tex

\begin{frame} % название фрагмента

\videotitle{След матрицы}

\end{frame}



\begin{frame}{Краткий план:}
  \begin{itemize}[<+->] 
    \item Сумма диагональных элементов.
    \item Свойства следа.
  \end{itemize}

\end{frame}


\begin{frame}{След квадратной матрицы}
    \begin{block}{Определение}
        \alert{Следом квадратной матрицы} $\LL$ называют сумму её диагональных элементов. 
        \[
            \trace \LL = \ell_{11} + \ell_{22} + \ldots + \ell_{nn}
        \]
    \end{block}

    \pause
    Пример. $\trace \begin{pmatrix}
        4 & 6 \\
        9 & 1
    \end{pmatrix} = 4 + 1 = 5$.

\end{frame}





\begin{frame}
    \frametitle{Основное свойство следа}

    \onslide<1->{\begin{block}{Утверждение}
        Если матрицы $A$ и $B$ имеют размер $n\times k$, то
        \[
        \trace A^T B = \sum_{ij} a_{ij} b_{ij} = \trace B^T A
        \]

    \end{block}}

    \begin{overlayarea}{\textwidth}{0.4\textheight}

    \only<2>{Пример. $A = \begin{pmatrix}
        a_1 & a_2 \\
        a_3 & a_4 \\ 
    \end{pmatrix},  \;
    B = \begin{pmatrix}
        b_1 & b_2 \\
        b_3 & b_4 \\ 
    \end{pmatrix}$.

    \[
    \trace A^T B = a_1 b_1 + a_2 b_2 + a_3 b_3 + a_4 b_4    
    \]} %
    \only<3>{\begin{block}{Доказательство}
        \[
        \trace A^T B = \sum_i \langle \row_i A^T, \col_i B \rangle =
        \]
        \[ 
         = \sum_i \langle \col_i A, \col_i B \rangle = \sum_{ij} a_{ij} b_{ij}
        \]
    \end{block}}
    \end{overlayarea}


\end{frame}


% \begin{frame}
%     \frametitle{Основное свойство следа}

% \begin{block}{Утверждение}
%     Если матрицы $A$ и $B$ имеют размер $n\times k$, то
%     \[
%     \trace A^T B = \sum_{ij} a_{ij} b_{ij} = \trace B^T A
%     \]

% \end{block}
% \pause

    
    

% \end{frame}


\begin{frame}
    \frametitle{И ещё немного свойств}

    Если $A$ имеет размер $n\times k$, а $B$ — размер $k\times n$, то:
    \[
    \trace AB = \trace BA    
    \]
    \pause

    След — линейный оператор, превращающий матрицы размера $n\times n$ в числа!
    \pause

    \[
    \trace \lambda A = \lambda \trace A    
    \]


    \[
    \trace (A+B) = \trace A + \trace B
    \]


\end{frame}


\begin{frame}
    \frametitle{Зачем нужен след?}

    \pause
    Элегантно позволяет записывать сложные выражения.
    
    \[
    \sum_{ij} a_{ij}^2 = \trace A^T A    
    \]

    \pause
    
    Упрощает теоретические выкладки. 

\end{frame}


