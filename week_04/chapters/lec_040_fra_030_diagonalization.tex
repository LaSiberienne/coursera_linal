% !TEX root = ../linal_lecture_04.tex

\begin{frame} % название фрагмента

\videotitle{Диагонализация матрицы}

\end{frame}



\begin{frame}{Краткий план:}
  \begin{itemize}[<+->]
    \item Собственные векторы как линейное пространство. 
    \item Геометрическая кратность собственных чисел.
    \item Диагонализация матрицы.
  \end{itemize}

\end{frame}


\begin{frame}
    \frametitle{Множество собственных векторов}

    Оператор $\LL :\R^n \to \R^n$ имеет собственное число $\lambda \in \R$.
    
    Рассмотрим множество $\eig_\lambda \LL$ — множество всех собственных векторов, растягивающихся 
    в $\lambda$ раз, дополненное нулевым вектором $\bzero$:
    \[
        \eig_\lambda \LL = \{ \bv \mid \LL \bv = \lambda \bv \}.
    \]

    \pause

    \begin{block}{Утверждение}
        Множество $\eig_{\lambda} \LL$ является векторным пространством:
        \pause

        Если вектор $\bv$ растягивается в $\lambda$ раз, то и вектор $t \bv$ растягивается в $\lambda$ раз.
        \pause

       Если векторы $\ba$ и $\bb$ растягивается в $\lambda$ раз, 
       то и их сумма $\bc = \ba + \bb$ растягивается в $\lambda$ раз.
    \end{block}

    

\end{frame}


\begin{frame}
    \frametitle{Геометрическая кратность}

    \begin{block}{Определение}
        Размерность пространства $\eig_\lambda \LL$ называется
        \alert{геометрической кратностью} собственного числа $\lambda \in \R$.        
    \end{block}

    \pause
    \begin{block}{Эквивалентное определение}
        Максимальное количество линейно независимых собственных векторов,
        соответствующих собственному числу $\lambda  \in \R$, называют его 
        \alert{геометрической кратностью}.        
    \end{block}

\end{frame}


\begin{frame}
\frametitle{Разные кратности связаны!}

\begin{block}{Утверждение}
    Геометрическая кратность собственного числа $\lambda\in \R$ не превосходит его алгебраической кратности и не меньше единицы.     
\end{block}

\pause
Пример. У матрицы $A$ характеристический многочлен равен $\charp_A(\lambda) = -(\lambda- 7)(\lambda - 9)^2$. \pause

Числу $\lambda =7$ соответствует ровно один линейно независимый собственный вектор.\pause

Числу $\lambda =9$ соответствуют один или два линейно независимых собственных вектора. 

\end{frame}


\begin{frame}
    \frametitle{Независимость собственных векторов}

    \begin{block}{Утверждение}
        Если векторы набора $A = \{\bv_1$, $\bv_2$, \ldots, $\bv_k\}$ относятся к различным
        собственным числам, то набор $A$ линейно независимый.
    \end{block}
    \pause
    \begin{block}{Идея доказательства}
        Пусть вектора $\bv_1$, $\bv_2$ и $\bv_3$ растягиваются в $2$, $3$ и $8$ раз соответственно, 
        и $\bv_3 = 7\bv_1 - 4 \bv_2$.\pause

        Домножим $A$ на обе части равенства, $8\bv_3 = 2 \cdot 7\bv_1 - 3\cdot 4\bv_2$.\pause

        Поделим на большее собственное число, $\bv_3 = \frac{2}{8} \cdot 7\bv_1 - \frac{3}{8}\cdot 4\bv_2$. \pause
        
        Повторим бесконечно много раз, $\bv_3 = \bzero$.
        Противоречие.
        
    \end{block}

    

\end{frame}




\begin{frame}
    \frametitle{Базис из собственных векторов}
    Векторы, отвечающие различным собственным числам, независимы.
    \pause 

    В каждом пространстве $\eig_{\lambda_i}\LL$ найдётся базис из
     $\gamma_i = \dim \eig_{\lambda_i}\LL$ собственных векторов.
     \pause

    \begin{block}{Утверждение}
Если $\sum_i \gamma_i = n$, то в $\R^n$ существует базис 
из $n$ векторов, являющихся собственными векторами оператора $\LL$.            
    \end{block}

\end{frame}


\begin{frame}
    \frametitle{Диагонализация: обозначения}
    Допустим, у оператора $\LL:\R^n\to\R^n$ нашлось $n$ линейно независимых 
    собственных векторов $\{\bv_1, \bv_2, \ldots, \bv_n\}$, которым соответствуют
    собственные числа $\{ \lambda_1, \lambda_2, \ldots, \lambda_n\}$.
    \pause

    Запишем все собственные векторы в матрицу $P$ столбцами друг за другом. 

    А в матрицу $D$ поместим все собственные числа на главную диагональ.



    \[
        P = \begin{pmatrix}
            \vert & \vert &   & \vert \\
            \bv_1 & \bv_2 & \ldots & \bv_n \\
            \vert & \vert &  & \vert \\
        \end{pmatrix}, \;
        D = \begin{pmatrix}
            \lambda_1 & 0 & \ldots & 0 \\
            0 & \lambda_2 & \ldots & 0 \\
            0 & 0 & \cdots &  0 \\
            0 & 0 & \ldots & \lambda_n \\
        \end{pmatrix}
    \]
\end{frame}    



\begin{frame}
\frametitle{Диагонализация: мне повезёт!}
    

    \begin{block}{Утверждение}
        Если у оператора $\LL:\R^n\to\R^n$ нашлось $n$ линейно независимых 
        собственных векторов $\{\bv_1, \bv_2, \ldots, \bv_n\}$,
        то $\LL$ представим в виде
        \[
        \LL = PDP^{-1}.    
        \]
    \end{block}
    \pause

    \begin{block}{Доказательство}
        Заметим, что $P \be_i = \bv_i$, и $\LL P \be_i = \lambda_i P \be_i$.
        \pause

        Домножаем на $P^{-1}$ и получаем $P^{-1}\LL P \be_i = \lambda_i \be_i$.
        \pause 

        Диагональная матрица растягивает базисные вектора, $P^{-1}\LL P \be_i = D \be_i$.
        \pause
        \[
        D = P^{-1} \LL P, \text{ или } \LL = P D P^{-1}    
        \]
    
    \end{block}
    

\end{frame}    


