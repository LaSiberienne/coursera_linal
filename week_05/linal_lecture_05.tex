\documentclass[14 pt,xcolor=dvipsnames]{beamer}

\usepackage{epsdice}

\usepackage[absolute,overlay]{textpos}

\usepackage[orientation=portrait,size=custom,width=25.4,height=19.05]{beamerposter}

%25,4 см 19,05 см размеры слайда в powerpoint

\usetheme{metropolis}
\metroset{
  %progressbar=none,
  numbering=none,
  subsectionpage=progressbar,
  block=fill
}

%\usecolortheme{seahorse}

\usepackage{fontspec}
\usepackage{polyglossia}
\setmainlanguage{russian}


\usepackage{fontawesome5} % removed [fixed]
\setmainfont[Ligatures=TeX]{Myriad Pro}
\setsansfont{Myriad Pro}


\usepackage{amssymb,amsmath,amsxtra,amsthm}


\usepackage{unicode-math}
\usepackage{centernot}

\usepackage{graphicx}
\graphicspath{{img/}}

\usepackage{wrapfig}
\usepackage{animate}
\usepackage{tikz}
%\usetikzlibrary{shapes.geometric,patterns,positioning,matrix,calc,arrows,shapes,fit,decorations,decorations.pathmorphing}
\usepackage{pifont}
\usepackage{comment}
\usepackage[font=small,labelfont=bf]{caption}
\captionsetup[figure]{labelformat=empty}
\includecomment{techno}

\usefonttheme[onlymath]{serif}


%Расположение

\setbeamersize{text margin left=15 mm,text margin right=5mm} 
\setlength{\leftmargini}{38 pt}

%\usepackage{showframe}
%\usepackage{enumitem}
%\setlist{leftmargin=5.5mm}


%Цвета от дирекции

\definecolor{dirblack}{RGB}{58, 58, 58}
\definecolor{dirwhite}{RGB}{245, 245, 245}
\definecolor{dirred}{RGB}{149, 55, 53}
\definecolor{dirblue}{RGB}{0, 90, 171}
\definecolor{dirorange}{RGB}{235, 143, 76}
\definecolor{dirlightblue}{RGB}{75, 172, 198}
\definecolor{dirgreen}{RGB}{155, 187, 89}
\definecolor{dircomment}{RGB}{128, 100, 162}

\setbeamercolor{title separator}{bg=dirlightblue!50, fg=dirblue}

%Цвета блоков

\setbeamercolor{block title}{bg=dirblue!30,fg=dirblack}

\setbeamercolor{block title example}{bg=dirlightblue!50,fg=dirblack}

\setbeamercolor{block body example}{bg=dirlightblue!20,fg=dirblack}

\AtBeginEnvironment{exampleblock}{\setbeamercolor{itemize item}{fg=dirblack}}
%\setbeamertemplate{blocks}[rounded][shadow]

% Набор команд для удобства верстки

\newcommand{\RR}{\mathbb{R}}
\newcommand{\ZZ}{\mathbb{Z}}
\newcommand{\la}{\lambda}

% Набор команд для структуризации

%\newcommand{\quest}{\faQuestionCircleO}
%\faPencilSquareO \faPuzzlePiece \faQuestionCircleO  \faIcon*[regular]{file} {\textcolor{dirblue}
%\newcommand{\quest}{\textcolor{dirblue}{\boxed{\textbf{?}}}
\newcommand{\task}{\faIcon{tasks}}
\newcommand{\exmpl}{\faPuzzlePiece}
\newcommand{\dfn}{\faIcon{pen-square}}
\newcommand{\quest}{\textcolor{dirblue}{\faQuestionCircle[regular]}}
\newcommand{\acc}[1]{\textcolor{dirred}{#1}}
\newcommand{\accm}[1]{\textcolor{dirred}{#1}}
\newcommand{\acct}[1]{\textcolor{dirblue}{#1}}
\newcommand{\acctm}[1]{\textcolor{dirblue}{#1}}
\newcommand{\accex}[1]{\textcolor{dirblack}{\bf #1}}
\newcommand{\accexm}[1]{\textcolor{dirblack}{ \mathbf{#1}}}
\newcommand{\acclp}[1]{\textcolor{dirorange}{\it #1}}


\newcommand{\videotitle}[1]{\begin{center}
    \textcolor{dirblue}{#1}

    \todo{название видеофрагмента}
\end{center}}

\newcommand{\lecturetitle}[1]{\begin{center}
    \textcolor{dirblue}{#1}

    \todo{название лекции}
\end{center}}




\newcommand{\todo}[1]{\textcolor{dircomment}{\bf #1}}

\newcommand{\spcbig}{\vspace{-10 pt}}
\newcommand{\spcsmall}{\vspace{-5 pt}}

%\usepackage{listings}
%\lstset{
%xleftmargin=0 pt,
%  basicstyle=\small, 
%  language=Python,
  %tabsize = 2,
%  backgroundcolor=\color{mc!20!white}
%}



%\newcommand{\mypart}[1]{\begin{frame}[standout]{\huge #1}\end{frame}}

\setbeamercolor{background canvas}{bg=}

% frame title setup
\setbeamercolor{frametitle}{bg=,fg=dirblue}
\setbeamertemplate{frametitle}[default][left]

\addtobeamertemplate{frametitle}{\hspace*{-0.5 cm}}{\vspace*{0.25cm}}


%Шрифты
\setbeamerfont{frametitle}{family=\rmfamily,series=\bfseries,size={\fontsize{33}{30}}}
\setbeamerfont{framesubtitle}{family=\rmfamily,series=\bfseries,size={\fontsize{26}{20}}}





\usepackage{physics}
\newcommand{\R}{\mathbb{R}}
\newcommand{\CC}{\mathbb{C}}

\newcommand{\Rot}{\mathrm{R}}
\newcommand{\HH}{\mathrm{H}}
\newcommand{\Id}{\mathrm{I}}



\usepackage[outline]{contour}


\usepackage{pgfplots}
\pgfplotsset{compat=newest}

\usepackage{tikz}
\usetikzlibrary{calc}
\usetikzlibrary{quotes,angles}
\usetikzlibrary{arrows}
\usetikzlibrary{arrows.meta}
\usetikzlibrary{positioning,intersections,decorations.markings}
\usetikzlibrary{patterns}

\usepackage{tkz-euclide} 

\newcommand{\grid}{\draw[color=gray,step=1.0,dotted] (-2.1,-2.1) grid (9.6,6.1)}

\newcommand{\ba}{\symbf{a}}
\newcommand{\be}{\symbf{e}}
\newcommand{\bb}{\symbf{b}}
\newcommand{\bc}{\symbf{c}}
\newcommand{\bd}{\symbf{d}}
\newcommand{\bx}{\symbf{x}}
\newcommand{\by}{\symbf{y}}
\newcommand{\bhy}{\symbf{\hat{y}}}
\newcommand{\bff}{\symbf{f}} % \bf is already def
\newcommand{\bv}{\symbf{v}}
\newcommand{\bzero}{\symbf{0}}
\newcommand{\red}[1]{\textcolor{red}{#1}}
\newcommand{\green}[1]{\textcolor{green}{#1}}
\newcommand{\blue}[1]{\textcolor{blue}{#1}}


\DeclareMathOperator{\eig}{Eig}

\DeclareMathOperator{\Lin}{Span}
\DeclareMathOperator{\col}{col}
\DeclareMathOperator{\row}{row}

\DeclareMathOperator{\adj}{adj}

\DeclareMathOperator{\sign}{sign}

\DeclareMathOperator{\charp}{char}

\DeclareMathOperator{\Span}{Span}
\DeclareMathOperator{\Image}{Image}


\DeclareMathOperator{\LL}{L}

%\tikzset{>=latex}

\colorlet{veca}{red}
\colorlet{vecb}{blue}
\colorlet{vecc}{olive}


\tikzset{cross/.style={cross out, draw=black, minimum size=2*(#1-\pgflinewidth), inner sep=0pt, outer sep=0pt},
%default radius will be 1pt. 
cross/.default={5pt}}





\begin{document}

% \maketitle


\begin{frame} % название лекции


\lecturetitle{Квадратичные формы}

\end{frame}


% % !TEX root = ../linal_lecture_05.tex

\begin{frame} % название фрагмента

\videotitle{Квадратичная форма}

\end{frame}



\begin{frame}{Краткий план:}
  \begin{itemize}[<+->]
    \item Определение квадратичной формы.
    \item Определённость формы.
  \end{itemize}

\end{frame}



\begin{frame}
    \frametitle{Квадратичная форма}    

    \begin{block}{Определение}
       Многочлен от нескольких переменных $f(x_1, x_2, \ldots, x_n)$, который содержит только слагаемые вида $x_i^2$ и $x_i x_j$ 
       \alert{квадратичной формой}.
    \end{block}

    \pause
    Функция $f(x,y) = x^2 + 6xy - 7y^2$ — квадратичная форма.

    \pause
    Функция $f(x, y, z) = x^2 + 6xz - 8xy + 3z + 9$ — не квадратичная форма. 
\end{frame}


\begin{frame}{Зачем нужны квадратичные формы?}
    
    Многие функции хорошо аппроксимируются суммой вида
    \[
    f(x, y) \approx 6 + 2x + 4y + 7x^2 + 8xy - 9y^2
    \]
    \pause
    Свойства квадратичной формы позволяют понять свойства многих функций!
    
\end{frame}


\begin{frame}{Квадратичная форма и матрицы}
\[
\begin{pmatrix}
    x_1 & x_2 & x_3 
\end{pmatrix} \cdot \begin{pmatrix}
    5 & \red{-1} & \blue{-3} \\
    \red{-1} & 7 & 2 \\
    \blue{-3} & 2 & 11 \\
\end{pmatrix} \cdot 
\begin{pmatrix}
    x_1 \\
     x_2 \\ 
     x_3 \\
\end{pmatrix} = \pause
\]
\[
 = 5x_1^2 + 7x_2^2 + 11x_3^2 \red{- 2}x_1 x_2 \blue{- 6}x_1 x_3 + 4 x_2 x_3 \pause
\]
    
\begin{block}{Утверждение}
    Любая квадратичная форма $f(\bx)$ может быть записана в виде 
    \[
      f(\bx) = \bx^T A \bx,  
    \]
    где $A$ — симметричная матрица, $A^T = A$.
\end{block}


\end{frame}


\begin{frame}
    \frametitle{Квадратичные формы в нуле}

    \begin{block}{Утверждение}
        Любая квадратичная форма $f$ равна $0$ в точке $\bzero$,
        \[
        f(\bzero)  = \bzero^T \cdot A \cdot \bzero = 0.
        \]
    \end{block}
    \pause
    Нас будет интересовать знак формы $f(\bx)$ при $\bx \neq \bzero$.

\end{frame}


\begin{frame}
    \frametitle{Положительно определённая форма}

    \begin{block}{Определение}
        Форма $f$ называется \alert{положительно определённой}, если $f(\bx) > 0$ при $\bx\neq\bzero$.
    \end{block}

    \pause

    \begin{center}
        \begin{tikzpicture}[
        scale=1.8,
        every node/.style={scale=0.4}
        ]
        \begin{axis}[grid=both, title = {$f(x_1,x_2) = x_1^2+x_2^2$},xlabel = $x_1$, ylabel = $x_2$, xtick = {-4,-2,...,4},
        ytick = {-4,-2,...,4}]
        \addplot3[surf,shader=faceted, samples=40] {x^2+y^2};
        \end{axis}
        \end{tikzpicture}
    \end{center}
    
\end{frame}


\begin{frame}
    \frametitle{Отрицательно определённая форма}

    \begin{block}{Определение}
        Форма $f$ называется \alert{отрицательно определённой}, если $f(\bx) < 0$ при $\bx\neq\bzero$.
    \end{block}

    \pause

\begin{center}		
    \begin{tikzpicture}[
    scale=1.8,
    every node/.style={scale=0.4}
    ]
    \begin{axis}[grid=both, title = {$f(x_1,x_2) = -x_1^2-x_2^2$}, xlabel = $x_1$, ylabel = $x_2$, xtick = {-4,-2,...,4},
    ytick = {-4,-2,...,4}]
    \addplot3[surf,shader=faceted, samples=40] {-x^2-y^2};
    \end{axis}
    \end{tikzpicture}
\end{center}
    
\end{frame}



\begin{frame}
    \frametitle{Положительно полуопределённая форма}

    \begin{block}{Определение}
        Форма $f$ называется \alert{положительно полуопределённой} или
        \alert{неотрицательно определённой}, если $f(\bx) \geq 0$.
    \end{block}

    \pause

\begin{center}
    \begin{tikzpicture}[
    scale=1.8,
    every node/.style={scale=0.4}
    ]
    \begin{axis}[grid=both, title = {$f(x_1,x_2) = x_1^2$}, xlabel = $x_1$, ylabel = $x_2$, xtick = {-4,-2,...,4},
    ytick = {-4,-2,...,4}]
    \addplot3[surf,shader=faceted, samples=40] {x^2};
    \end{axis}
    \end{tikzpicture}
\end{center}
    
\end{frame}



\begin{frame}
\frametitle{Отрицательно полуопределённая форма}

\begin{block}{Определение}
    Форма $f$ называется \alert{отрицательно полуопределённой} или
    \alert{неположительно определённой}, если $f(\bx) \leq 0$.
\end{block}

    \pause

\begin{center}
    \begin{tikzpicture}[
    scale=1.8,
    every node/.style={scale=0.4}
    ]
    \begin{axis}[grid=both, title = {$f(x_1,x_2) = -x_1^2$},xlabel = $x_1$, ylabel = $x_2$, xtick = {-4,-2,...,4},
    ytick = {-4,-2,...,4}]
    \addplot3[surf,shader=faceted, samples=40] {-x^2};
    \end{axis}
    \end{tikzpicture}
\end{center}
    
\end{frame}



\begin{frame}
    \frametitle{Неопределённая форма}

    \begin{block}{Определение}
        Форма $f$ называется \alert{неопределённой}, если она принимает и положительные и отрицательные значения.
    \end{block}

    \pause

\begin{center}
    \begin{tikzpicture}[
    scale=1.8,
    every node/.style={scale=0.4}
    ]
    \begin{axis}[grid=both, title = {$f(x_1,x_2) = x_1^2-x_2^2$},xlabel = $x_1$, ylabel = $x_2$, xtick = {-4,-2,...,4},
    ytick = {-4,-2,...,4}]
    \addplot3[surf,shader=faceted, samples=40] {x^2-y^2};
    \end{axis}
    \end{tikzpicture}
\end{center}
    
\end{frame}



\begin{frame}
    \frametitle{Когда форма равна нулю?}

    \begin{block}{Утверждение}
        Если форма $f$ равна $0$ в точке $\bx$, то она равна нулю и
        в любой точке $t\bx$. 
    \end{block}
    \pause
    \begin{block}{Доказательство}
        \[
        f(t\bx) = t\bx^T \cdot A \cdot t\bx = t^2 \bx^T A \bx = 0
        \]
    \end{block}
    \pause
    Квадратичная форма возможно равна нулю на прямых, проходящих через $\bzero$.
    
\end{frame} % долго рендерится из-за 3d графиков!

\begin{frame}
\lecturetitle{Метод полных квадратов}
\todo{Это видеофрагмент с доской, слайдов здесь нет :)}
\end{frame}
    

% !TEX root = ../linal_lecture_05.tex

\begin{frame} % название фрагмента

\videotitle{Диагонализация квадратичной формы}

\end{frame}



\begin{frame}{Краткий план:}
  \begin{itemize}[<+->]
    \item Симметричная матрица и собственные числа.
    \item Диагонализация квадратичной формы.
  \end{itemize}

\end{frame}


\begin{frame}{Всегда диагонализуема!}
    \begin{block}{Утверждение}
        Если $A$ — симметричная матрица, $A^T = A$, то у неё всегда найдётся
        ровно $n$ \alert{действительных} собственных чисел $\lambda_i$ \pause 
        и ровно $n$ линейно независимых \alert{ортогональных} собственных векторов.
    \end{block}
    \pause
    \begin{block}{Следствие}
        У симметричной $A$ можно найти $n$ ортогональных собственных векторов единичной длины.

        Симметричная матрица $A$ всегда диагонализуема!
    \end{block}
\end{frame}    


\begin{frame}
    \frametitle{Чем хороши ортогональные векторы?}

    Векторы $\bv_1$, \ldots, $\bv_n$ — ортогональные и единичной длины.
    \[
    P = \begin{pmatrix}
        \vert &  & \vert \\
        \bv_1 & \ldots & \bv_n \\
        \vert &  & \vert \\
    \end{pmatrix}    \pause \quad
    P^T = \begin{pmatrix}
\text{———} \hspace{-0.2cm} & \bv_1 & \hspace{-0.2cm} \text{———} \\
 & \vdots &  \\
\text{———} \hspace{-0.2cm} & \bv_n & \hspace{-0.2cm} \text{———} \\
        \end{pmatrix}    \pause
    \]
%
    \[
    P^T P = \begin{pmatrix}        
        1 & 0 & \ldots & 0 \\
        0 & 1 & \ldots & 0 \\
        0 & 0 & \ldots & 0 \\
        0 & 0 & \ldots & 1 \\
    \end{pmatrix}  = \Id   \pause
    \]
%
    \[
    P^T = P^{-1}    
    \]
    
\end{frame}

\begin{frame}
    \frametitle{Диагонализация формы}

    \begin{block}{Утвеждение}
        Квадратичная форма $f(\bx) = \bx^T A \bx$ с симметричной $A$ представима в виде
        \[
        f(\bx) = \bx^T PDP^{-1} \bx,    
        \] 
        где $D$ — диагональная матрица из собственных чисел матрицы $A$, 
        а $P$ — матрица из линейно независимых собственных векторов матрицы $A$.
    \end{block}
    \pause
\begin{block}{Утвеждение}
    Всегда можно выбрать ортогональные собственные векторы $A$ единичной длины, 
    при этом представление примет вид:
    \[
    f(\bx) = \bx^T PDP^T \bx = (P^T \bx)^T D (P^T \bx),    
    \] 
\end{block}
    
\end{frame}


\begin{frame}
    \frametitle{Диагонализация формы}
\begin{block}{Утвеждение}
    Всегда можно выбрать ортогональные собственные векторы единичной длины, 
    при этом представление примет вид:
    \[
    f(\bx) = \bx^T PDP^T \bx = (P^T \bx)^T D (P^T \bx),    
    \] 
    где $D$ — диагональная матрица из собственных чисел матрицы $A$, 
    а $P$ — матрица из собственных векторов матрицы $A$.
\end{block}
    \pause
    Это просто удачная замена переменных $\by=P^T \bx$!\pause
    \[
    f(\bx) =  (P^T \bx)^T D (P^T \bx) = \by^T D \by = 
    \]
    \[
    = \lambda_1 y_1^2 + \lambda_2 y_2^2 + \ldots + \lambda_n y_n^2
    \]
\end{frame}

\begin{frame}
    \frametitle{Определённость формы}
    Пример, $f(\bx) = 5 y_1^2 + 6y_2^2 - 9y_3^2$. \pause

    Квадратичная форма $f$ неопределена. 
    \pause
\begin{block}{Утверждение}
    Квадратичная форма 
    \[
      f(\bx) = \lambda_1 y_1^2 + \lambda_2 y_2^2 + \ldots + \lambda_n y_n^2  
    \]
    является\ldots\pause 

    положительно определённой, если все $\lambda_i >0$.\pause

    отрицательно определённой, если все $\lambda_i <0$.\pause

    положительно полуопределённой, если все $\lambda_i \geq 0$.\pause

    отрицательно полуопределённой, если все $\lambda_i \leq 0$.\pause

    неопределённой, если найдётся $\lambda_i >0$ и $\lambda_j < 0$.
\end{block}

\end{frame}

\begin{frame}
    \frametitle{Кусочек доказательства}

    \begin{block}{Утверждение}
        Для симметричной матрицы $A$, $A^T=A$, собственные вектора, 
        соответствующие разным $\lambda$, ортогональны. 
    \end{block}

    \pause

\begin{block}{Доказательство}
К примеру, $A\bx = 5\bx$ и $A\by = 7\by$. \pause
\[
\begin{array}{c} 
    \langle A\bx, \by \rangle = \langle 5\bx, \by \rangle = 5 \langle \bx, \by \rangle \pause \\
    \langle \bx, A\by \rangle = \langle \bx, 7\by \rangle = 7 \langle \bx, \by \rangle \pause \\
    \langle A\bx, \by \rangle =  \langle \bx, A^T\by \rangle =  \langle \bx, A \by \rangle \pause \\
 \end{array}
\]

Равенство возможно, только если $\bx \perp \by$:
\[
    5 \langle \bx, \by \rangle = 7 \langle \bx, \by \rangle
\]

\end{block}


    

\end{frame}

% !TEX root = ../linal_lecture_05.tex

\begin{frame} % название фрагмента

\videotitle{Критерий Сильвестра}

\end{frame}



\begin{frame}{Краткий план:}
  \begin{itemize}[<+->]
    \item Критерий Сильвестра.
    \item Расширенный критерий Сильвестра.
  \end{itemize}

\end{frame}


\begin{frame}
    \frametitle{Обозначение}

    Будем вычёркивать из матрицы $A$ строки и столбцы так, чтобы остались строки и столбцы с одинаковыми номерами.\pause

    Скажем, оставим в матрице $A$ только $2$-ю и $4$-ю строки и 
    $2$-й и $4$-й столбцы. \pause

    Определитель полученной подматрицы обозначим $m_{24}$. \pause

    Пример. 
    \[
    A = \begin{pmatrix}
        5 & 2 & 3 & -1 \\
        2 & \blue{6} & 2 & \blue{1} \\
        3 & 2 & 9 & 5 \\
        -1 & \blue{1} & 5 & \blue{8} \\
    \end{pmatrix}, \; m_{24} = \begin{vmatrix}
        \blue{6} & \blue{1} \\
        \blue{1} & \blue{8} \\
    \end{vmatrix} = 47.
    \]
    
\end{frame}


\begin{frame}
\frametitle{Названия миноров}

\begin{block}{Определения}
    В матрице $A$ вычеркнули несколько строк и столбцов так,
    что остались  строки и столбцы с одинаковыми номерами.

    Определитель полученной подматрицы называется \alert{главным минором}.
\end{block}
\pause

\begin{block}{Определения}
    В матрице $A$ вычеркнули несколько строк и столбцов так,
    что остались  строки и столбцы с номерами $1$, $2$, \ldots, $k$.

    Определитель полученной подматрицы называется \alert{угловым минором}.
\end{block}

\pause
\begin{block}{Определение}
    \alert{Порядком} минора называется число строк (или столбцов) в соответствующей подматрице.    
\end{block}


\end{frame}




\begin{frame}
    \frametitle{Критерий Сильвестра}

    \begin{block}{Утверждение}
        Симметричная матрица $A$ является положительно определённой, если и только если

        $m_1 > 0$, $m_{12} > 0$, $m_{123} > 0$, $m_{1234}>0$, \ldots   \pause     
    \end{block}

    Пример. 
\[
A = \begin{pmatrix}
    \blue{5} & 2 & \blue{3}  \\
    2 & 6 & \blue{2} \\
    \blue{3} & \blue{2} & \blue{9} \\
\end{pmatrix}
\]
\[
    m_1 = 5, \; m_{12} = \begin{vmatrix}
        5 & 2 \\
        2 & 6
    \end{vmatrix} = 26, \; 
    m_{123} = \begin{vmatrix}
        5 & 2 & 3 \\
        2 & 6 & 2 \\
        3 & 2 & 9
    \end{vmatrix}=  184
\]
    
\end{frame}


\begin{frame}
    \frametitle{Наблюдение}

    \begin{block}{Утверждение}
    Если помножить на $(-1)$ все элементы матрицы $A$ размера $n\times n$, то определитель матрицы $A$\ldots \pause

    поменяет знак, если $n$ — нечётное; \pause

    сохранит знак, если $n$ — чётное. 
    \end{block}

    \pause
    Легко получим критерий отрицательной определённости!


\end{frame}

    

\begin{frame}
\frametitle{Критерий Сильвестра}

\begin{block}{Утверждение}
    Симметричная матрица $A$ является отрицательно определённой, если и только если

    $\blue{m_1 < 0}$, $\red{m_{12} > 0}$, $\blue{m_{123} < 0}$, $\red{m_{1234}>0}$, \ldots   \pause   
\end{block}

Пример. 
\[
B = \begin{pmatrix}
    \blue{-5} & -2 & \blue{-3}  \\
    -2 & -6 & \blue{-2} \\
    \blue{-3} & \blue{-2} & \blue{-9} \\
\end{pmatrix}
\]
\[
    m_1 = -5, \; m_{12} = \begin{vmatrix}
        -5 & -2 \\
        -2 & -6
    \end{vmatrix} = 26, \; 
    m_{123} = \begin{vmatrix}
        -5 & -2 & -3 \\
        -2 & -6 & -2 \\
        -3 & -2 & -9
    \end{vmatrix}=  -184
\]



\end{frame}







\begin{frame}
    \frametitle{Расширенный критерий}

    \begin{block}{Утверждение}
        Симметричная матрица $A$ является положительно полуопределённой, если и только если (для всех $i$, $j$, $k$, \ldots)

        $m_i \geq 0$,  $m_{ij} \geq 0$, $m_{ijk} \geq 0$, $m_{ijkl} \geq 0$, \ldots \pause
    \end{block}
\[
A = \begin{pmatrix}
    4 & 6 \\
    6 & 9 \\ 
\end{pmatrix}
\]
\[
    m_1 = 4, \; m_2 = 9, \; m_{12} = \begin{vmatrix}
        4 & 6 \\
        6 & 9
    \end{vmatrix} = 0
\]
    
\end{frame}




\begin{frame}
    \frametitle{Расширенный критерий}

    \begin{block}{Утверждение}
        Симметричная матрица $A$ является отрицательно полуопределённой, если и только если (для всех $i$, $j$, $k$, \ldots)

        $\blue{m_i \leq 0}$,  $\red{m_{ij} \geq 0}$, $\blue{m_{ijk} \leq 0}$, $\red{m_{ijkl} \geq 0}$, \ldots \pause
    \end{block}
\[
A = \begin{pmatrix}
    -4 & 6 \\
    6 & -9 \\ 
\end{pmatrix}
\]
\[
    m_1 = -4, \; m_2 = -9, \; m_{12} = \begin{vmatrix}
        -4 & 6 \\
        6 & -9
    \end{vmatrix} = 0
\]
    
\end{frame}


\begin{frame}
    \frametitle{Резюме для положительной формы}

    \begin{block}{Утверждение}
        Квадратичная форма $f(\bx) = \bx^T A\bx$ является положительно определённой, если \pause

        \begin{enumerate}
            \item В любой точке $\bx\neq \bzero$ она положительна, $f(\bx) > 0$. \pause
            \item Все собственные числа матрицы $A$ положительны, $\lambda_i > 0$. \pause
            \item Все угловые миноры матрицы $A$ положительны, $m_{12\ldots k} > 0$.
        \end{enumerate}
            
    \end{block}


\end{frame}


\begin{frame}
    \frametitle{Резюме для отрицательной формы}

    \begin{block}{Утверждение}
        Квадратичная форма $f(\bx) = \bx^T A\bx$ является отрицательно определённой, если \pause

        \begin{enumerate}
            \item В любой точке $\bx\neq \bzero$ она отрицательна, $f(\bx) < 0$. \pause
            \item Все собственные числа матрицы $A$ отрицательны, $\lambda_i < 0$. \pause
            \item Нечётные угловые миноры матрицы $A$ отрицательны, а чётные — положительны.
        \end{enumerate}
            
    \end{block}


\end{frame}




\begin{frame}
    \frametitle{Резюме для полуопределённости}

    \begin{block}{Утверждение}
        Квадратичная форма $f(\bx) = \bx^T A\bx$ является положительно полуопределённой (неотрицательно определённой), если \pause

        \begin{enumerate}
            \item В любой точке $\bx$ она неотрицательна, $f(\bx) \geq 0$. \pause
            \item Все собственные числа матрицы $A$ неотрицательны, $\lambda_i \geq 0$. \pause
            \item Все главные миноры матрицы $A$ неотрицательны.
        \end{enumerate}
            
    \end{block}


\end{frame}



\begin{frame}
    \frametitle{Резюме для полуопределённости}

    \begin{block}{Утверждение}
        Квадратичная форма $f(\bx) = \bx^T A\bx$ является отрицательно полуопределённой (неположительно определённой), если \pause

        \begin{enumerate}
            \item В любой точке $\bx$ она неположительна, $f(\bx) \leq 0$. \pause
            \item Все собственные числа матрицы $A$ неположительны, $\lambda_i \leq 0$. \pause
            \item Нечётные главные миноры матрицы $A$ неположительны, а чётные — неотрицательны.
        \end{enumerate}
            
    \end{block}


\end{frame}



\begin{frame}
\lecturetitle{Расширенный критерий Сильвестра: пример}
\todo{Это видеофрагмент с доской, слайдов здесь нет :)}
\end{frame}

% !TEX root = ../linal_lecture_05.tex

\begin{frame} % название фрагмента

\videotitle{Матрица Грама}

\end{frame}



\begin{frame}{Краткий план:}
  \begin{itemize}[<+->]
    \item Матрица Грама.
    \item Матрица Грама и проекция.
    \item Ортогональный базис.
  \end{itemize}

\end{frame}

\begin{frame}
    \frametitle{Матрица Грама}

    \begin{block}{Определение}
        Возьмём векторы $\bx_1$, $\bx_2$, \ldots, $\bx_k$ из $\R^n$. 
        Матрица их попарных скалярных произведений называется \alert{матрицей Грама},
        \[
            M =  \begin{pmatrix}
                \langle \bx_1, \bx_1 \rangle &  \langle \bx_1, \bx_2 \rangle & \ldots & \langle \bx_1, \bx_k \rangle \\
                \langle \bx_2, \bx_1 \rangle &  \langle \bx_2, \bx_2 \rangle & \ldots & \langle \bx_2, \bx_k \rangle \\
                \ldots & \ldots & \ldots & \ldots \\
                \langle \bx_k, \bx_1 \rangle &  \langle \bx_k, \bx_2 \rangle & \ldots & \langle \bx_k, \bx_k \rangle \\ 
            \end{pmatrix} = X^T X\pause
        \]      

        А определитель этой матрицы называется \alert{определителем Грама}, $G=\det M$.
    \end{block}


\end{frame}


\begin{frame}
    \frametitle{Свойства матрицы Грама}

    \begin{block}{Утверждение}
        Векторы $\bx_1$, $\bx_2$, \ldots, $\bx_k$ линейно независимы, если и только если определитель Грама отличен от нуля, $G\neq 0$. \pause
    \end{block}

    \begin{block}{Утверждение}
        Матрица Грама положительно полуопределена. \pause
    \end{block}


    \begin{block}{Утверждение}
        Если $\bx_1$, $\bx_2$, \ldots, $\bx_n$ лежат в $\R^n$, то определитель Грама $G$ равен квадрату объёма параллелепипеда, 
        образованного векторами $\bx_1$,  $\bx_2$, \ldots, $\bx_n$.
    \end{block}

\end{frame}

\begin{frame}{Положительная полуопределённость}

    \begin{block}{Утверждение}
        Матрица Грама положительно полуопределена. \pause
    \end{block} 

    \begin{block}{Доказательство}
        \[
        \bv^T M \bv  = \sum_{ij} v_i v_j \langle \bx_i, \bx_j \rangle = \sum_{ij} \langle v_i \bx_i, v_j \bx_j \rangle =   \pause
        \]
        \[
        =  \langle \sum_i v_i  \bx_i, \sum_j v_j \bx_j \rangle = \langle \ba, \ba \rangle \geq 0
        \]
    \end{block}
    
\end{frame}



\begin{frame}
    \frametitle{Поиск проекции}
    Хотим найти проекцию $\bhy$ вектора  $\by$ на $\Span\{ \bx_1, \bx_2, \ldots, \bx_k \}$. \pause

    Проекция $\bhy$ — линейная комбинация $\bx_1, \bx_2, \ldots, \bx_k$,
\[
\bhy = v_1 \bx_1 + \ldots + v_k \bx_k = X \bv    \pause
\]
Условия первого порядка:
\[
X^T X  \bv = X^T y \pause \; \text{ или } \; M \bv = X^Ty    \pause 
\]    
\[
    \bv = M^{-1} X^Ty.    
\]
\end{frame}


\begin{frame}
    \frametitle{Ортогональные вектора}

    \begin{block}{Утверждение}
        Если векторы $\bx_1$, $\bx_2$, \ldots, $\bx_k$ ортогональны, то их матрица Грама —
        диагональная.
        \[
            M  = \begin{pmatrix}
                \langle \bx_1, \bx_1 \rangle & 0 & \ldots & 0 \\
                0 & \langle \bx_2, \bx_2 \rangle &  \ldots & 0 \\
                \ldots & \ldots & \ldots & \ldots \\
                0 & 0 & \ldots & \langle \bx_k, \bx_k \rangle \\
            \end{pmatrix}
        \]      

    \end{block}
    

\end{frame}


\begin{frame}
\lecturetitle{Ортогонализация Грамма-Шмидта: пример}
\todo{Это видеофрагмент с доской, слайдов здесь нет :)}
\end{frame}


% % !TEX root = ../linal_lecture_05.tex

\begin{frame} % название фрагмента

\videotitle{Ортогонализация Грама-Шмидта}

\end{frame}



\begin{frame}{Краткий план:}
  \begin{itemize}[<+->]
    \item Ортогонализация.
    \item $QR$-разложение.
  \end{itemize}

\end{frame}

\begin{frame}
    \frametitle{Ортогонализация Грама-Шмидта}

    \begin{block}{Постановка задачи}
        Исходя из стартового независимого набора векторов 
        $\{\bv_1, \bv_2, \ldots, \bv_k\}$

        получить новый набор векторов $\{\bff_1, \bff_2, \ldots, \bff_k\}$ 
        
        со свойствами:\pause
        \[
            \begin{array}{l}
                \bff_1, \bff_2, \ldots, \bff_k — \text{ ортогональны;} \\
            \Span \bv_1 = \Span \bff_1; \\ \pause 
            \Span \{ \bv_1, \bv_2 \} = \Span \{ \bff_1, \bff_2\}; \\ \pause 
\Span \{ \bv_1, \bv_2, \bv_3 \} = \Span \{ \bff_1, \bff_2, \bff_3 \}; \\
                \ldots \\
            \end{array}
        \]

    \end{block}


\end{frame}


\begin{frame}
    \frametitle{Ортогонализация Грама-Шмидта}
    
    \begin{block}{Обозначение}
        С помощью $H_p(\bv)$ обозначим проекцию $\bv$ на $\Span \{ \bv_1, \bv_2, \ldots, \bv_p \} = \Span \{ \bff_1, \bff_2, \ldots, \bff_p \}$. \pause
    \end{block}

    \begin{block}{Алгоритм}
        \begin{enumerate}
            \item $\bff_1 = \bv_1$; \pause
            \item $\bff_2 = \bv_2 - H_1(\bv_2)$; \pause
            \item $\bff_3 = \bv_3 - H_2(\bv_3)$; \pause
            \item \ldots
        \end{enumerate}        \pause
    \end{block}

    Если нужно получить ортогональные вектора $\bq_i$ единичной длины, то дополнительно масштабируют
    $\bq_i = \bff_i / \norm{\bff_i}$.
    
\end{frame}


\begin{frame}
    \frametitle{Проецировать бывает легко!}
    Хотим найти проекцию $H_p(\bv_{p+1})$ вектора  $\bv_{p+1}$ на $\Span\{ \bff_1, \bff_2, \ldots, \bff_p \}$. 

    Проекция $H_p(\bv_{p+1})$ — линейная комбинация $\bff_1, \bff_2, \ldots, \bff_p$,
    \[
        H_k(\bv_{p+1}) = \alpha_1 \bff_1 + \ldots + \alpha_k \bff_p = F \alpha    \pause
    \]
    \[
    \alpha = (F^TF)^{-1} F^T \bv_{p+1} \pause    
    \]
    Столбцы матрицы $F$ ортогональны, поэтому проецировать очень легко!
    \[
    \alpha = \begin{pmatrix}
        \langle \bv_{p+1}, \bff_1 \rangle / \langle \bff_1, \bff_1 \rangle \\
        \langle \bv_{p+1}, \bff_2 \rangle / \langle \bff_2, \bff_2 \rangle \\
        \ldots \\
\langle \bv_{p+1}, \bff_p \rangle / \langle \bff_p, \bff_p \rangle \\
    \end{pmatrix}    
    \]

\end{frame}


\begin{frame}
    \frametitle{Ортогонализация Грама-Шмидта}
    
    \begin{block}{Алгоритм}
        \begin{enumerate}
            \item $\bff_1 = \bv_1$; \pause
            \item $\bff_2 = \bv_2 - \frac{ \langle \bv_{2}, \bff_1 \rangle}{\langle \bff_1, \bff_1 \rangle } \bff_1$; \pause
            \item $\bff_3 = \bv_3 - \frac{ \langle \bv_{3}, \bff_1 \rangle}{\langle \bff_1, \bff_1 \rangle } \bff_1  - \frac{ \langle \bv_{3}, \bff_2 \rangle}{\langle \bff_2, \bff_2 \rangle } \bff_2$; \pause
            \item \ldots
        \end{enumerate}        \pause
    \end{block}

    Если нужно получить ортогональные вектора $\bq_i$ единичной длины, то дополнительно масштабируют
    $\bq_i = \bff_i / \norm{\bff_i}$.
    
\end{frame}



\begin{frame}
    \frametitle{Подметим особенность!}

    \begin{block}{Утверждение}
        В процедуре Грама-Шмидта: \\

        Вектор $\bff_i$ является линейной комбинацией $\bv_1$, $\bv_2$, \ldots, $\bv_i$. \pause

        Вектор $\bv_i$ является линейной комбинацией $\bff_1$, $\bff_2$, \ldots, $\bff_i$. \pause    
    \end{block} \pause

    На лаконичном языке матриц:
    \[
        F = VU =  \begin{pmatrix}
            \vert &  & \vert \\
            \bv_1 & \ldots & \bv_k \\
            \vert &  & \vert \\
        \end{pmatrix}  \cdot 
            \begin{pmatrix}
                u_{11} & u_{12} & \ldots & u_{1k} \\
                0 & u_{22} & \ldots & u_{2k} \\
\ldots & \ldots & \ldots & \ldots \\
0 & 0 & \ldots & u_{kk} \\        
            \end{pmatrix}, \pause
    \]
    Матрица $U$ — верхнетреугольная обратимая.

    
\end{frame}


\begin{frame}
    \frametitle{Осталось поделить на длину}

    Деление столбцов матрицы $F$ на длину можно реализовать с помощью диагональной матрицы $D$: \pause
    \[
    Q = F D  = \begin{pmatrix}
        \vert &  & \vert \\
        \bff_1 & \ldots & \bff_k \\
        \vert &  & \vert \\
    \end{pmatrix}  \cdot \begin{pmatrix}
        1/ \norm{\bff_1} & 0 & \ldots & 0 \\
0 & 1/ \norm{\bff_2} & \ldots & 0 \\
\ldots & \ldots & \ldots & \ldots \\
0 & 0 & \ldots & 1/ \norm{\bff_k} \\
    \end{pmatrix} \pause
    \]

    \[
    \begin{array}{l}
    Q = FD = VUD \\
    V = Q(UD)^{-1} = QR
    \end{array}
    \]
    
    Матрица $R$ — верхнетреугольная обратимая.


\end{frame}


\begin{frame}
    \frametitle{$QR$-разложение}

    \begin{block}{Утверждение}
        Любая квадратная матрица $V$ может быть представлена в виде
        \[
        V = QR, 
        \]
        где матрица $Q$ ортогональная, $Q^T Q= \Id$, а матрица $R$ — верхнетреугольная. \pause
    \end{block}
    
    Утверждение верно, даже если $V$ — необратимая матрица.

    В этом случае матрица $R$ также будет необратимой.

\end{frame}


\begin{frame}
    \frametitle{Резюме}

    \begin{itemize}[<+->]
    \item Квадратичная форма.
    \item Определённость квадратичной формы
    \item Метод полных квадратов, собственные числа и критерий Сильвестра.
    \item Матрица Грама.
    \item Ортогонализация Грама-Шмидта.
    \item $QR$-разложение.
    \item Бонусное видео: задача о переливании красок.
    \end{itemize}
    \pause
    \alert{Следующая лекция:} сингулярное разложение.
        


\end{frame}


% \input{chapters/lec_040_fra_090_qr.tex}


\begin{frame}
\lecturetitle{Бонус: задача про переливание красок}
\todo{Это видеофрагмент с доской, слайдов здесь нет :)}
\end{frame}
    


\end{document}
