% !TEX root = ../linal_lecture_05.tex

\begin{frame} % название фрагмента

\videotitle{Ортогонализация Грама-Шмидта}

\end{frame}



\begin{frame}{Краткий план:}
  \begin{itemize}[<+->]
    \item Ортогонализация.
    \item $QR$-разложение.
  \end{itemize}

\end{frame}

\begin{frame}
    \frametitle{Ортогонализация Грама-Шмидта}

    \begin{block}{Постановка задачи}
        Исходя из стартового независимого набора векторов 
        $\{\bv_1, \bv_2, \ldots, \bv_k\}$

        получить новый набор векторов $\{\bff_1, \bff_2, \ldots, \bff_k\}$ 
        
        со свойствами:\pause
        \[
            \begin{array}{l}
                \bff_1, \bff_2, \ldots, \bff_k — \text{ ортогональны;} \\
            \Span \bv_1 = \Span \bff_1; \\ \pause 
            \Span \{ \bv_1, \bv_2 \} = \Span \{ \bff_1, \bff_2\}; \\ \pause 
\Span \{ \bv_1, \bv_2, \bv_3 \} = \Span \{ \bff_1, \bff_2, \bff_3 \}; \\
                \ldots \\
            \end{array}
        \]

    \end{block}


\end{frame}


\begin{frame}
    \frametitle{Ортогонализация Грама-Шмидта}
    
    \begin{block}{Обозначение}
        С помощью $H_p(\bv)$ обозначим проекцию $\bv$ на $\Span \{ \bv_1, \bv_2, \ldots, \bv_p \} = \Span \{ \bff_1, \bff_2, \ldots, \bff_p \}$. \pause
    \end{block}

    \begin{block}{Алгоритм}
        \begin{enumerate}
            \item $\bff_1 = \bv_1$; \pause
            \item $\bff_2 = \bv_2 - H_1(\bv_2)$; \pause
            \item $\bff_3 = \bv_3 - H_2(\bv_3)$; \pause
            \item \ldots
        \end{enumerate}        \pause
    \end{block}

    Если нужно получить ортогональные вектора $\bq_i$ единичной длины, то дополнительно масштабируют
    $\bq_i = \bff_i / \norm{\bff_i}$.
    
\end{frame}


\begin{frame}
    \frametitle{Проецировать бывает легко!}
    Хотим найти проекцию $H_k(\bv_{k+1})$ вектора  $\bv_{k+1}$ на $\Span\{ \bff_1, \bff_2, \ldots, \bff_k \}$. 

    Проекция $H_k(\bv_{k+1})$ — линейная комбинация $\bff_1, \bff_2, \ldots, \bff_k$,
    \[
        H_k(\bv_{k+1}) = \alpha_1 \bff_1 + \ldots + \alpha_k \bff_k = F \alpha    \pause
    \]
    \[
    \alpha = (F^TF)^{-1} F^T \bv_{k+1} \pause    
    \]
    Столбцы матрицы $F$ ортогональны, поэтому проецировать очень легко!
    \[
    \alpha = \begin{pmatrix}
        \langle \bv_{k+1}, \bff_1 \rangle / \langle \bff_1, \bff_1 \rangle \\
        \langle \bv_{k+1}, \bff_2 \rangle / \langle \bff_2, \bff_2 \rangle \\
        \ldots \\
\langle \bv_{k+1}, \bff_k \rangle / \langle \bff_k, \bff_k \rangle \\
    \end{pmatrix}    
    \]

\end{frame}


\begin{frame}
    \frametitle{Ортогонализация Грама-Шмидта}
    
    \begin{block}{Алгоритм}
        \begin{enumerate}
            \item $\bff_1 = \bv_1$; \pause
            \item $\bff_2 = \bv_2 - \frac{ \langle \bv_{2}, \bff_1 \rangle}{\langle \bff_1, \bff_1 \rangle } \bff_1$; \pause
            \item $\bff_3 = \bv_3 - \frac{ \langle \bv_{3}, \bff_1 \rangle}{\langle \bff_1, \bff_1 \rangle } \bff_1  - \frac{ \langle \bv_{3}, \bff_2 \rangle}{\langle \bff_2, \bff_2 \rangle } \bff_2$; \pause
            \item \ldots
        \end{enumerate}        \pause
    \end{block}

    Если нужно получить ортогональные вектора $\bq_i$ единичной длины, то дополнительно масштабируют
    $\bq_i = \bff_i / \norm{\bff_i}$.
    
\end{frame}



\begin{frame}
    \frametitle{Подметим особенность!}

    \begin{block}{Утверждение}
        В процедуре Грама-Шмидта: \\

        Вектор $\bff_i$ является линейной комбинацией $\bv_1$, $\bv_2$, \ldots, $\bv_i$. \pause

        Вектор $\bv_i$ является линейной комбинацией $\bff_1$, $\bff_2$, \ldots, $\bff_i$. \pause    
    \end{block} \pause

    На лаконичном языке матриц:
    \[
        F = VU =  \begin{pmatrix}
            \vert &  & \vert \\
            \bv_1 & \ldots & \bv_k \\
            \vert &  & \vert \\
        \end{pmatrix}  \cdot 
            \begin{pmatrix}
                u_{11} & u_{12} & \ldots & u_{1k} \\
                0 & u_{22} & \ldots & u_{2k} \\
\ldots & \ldots & \ldots & \ldots \\
0 & 0 & \ldots & u_{kk} \\        
            \end{pmatrix}, \pause
    \]
    Матрица $U$ — верхнетреугольная обратимая.

    
\end{frame}


\begin{frame}
    \frametitle{Осталось поделить на длину}

    Деление столбцов матрицы $F$ на длину можно реализовать с помощью диагональной матрицы $D$: \pause
    \[
    Q = F D  = \begin{pmatrix}
        \vert &  & \vert \\
        \bff_1 & \ldots & \bff_k \\
        \vert &  & \vert \\
    \end{pmatrix}  \cdot \begin{pmatrix}
        1/ \norm{\bff_1} & 0 & \ldots & 0 \\
0 & 1/ \norm{\bff_2} & \ldots & 0 \\
\ldots & \ldots & \ldots & \ldots \\
0 & 0 & \ldots & 1/ \norm{\bff_k} \\
    \end{pmatrix} \pause
    \]

    \[
    \begin{array}{l}
    Q = FD = VUD \\
    V = Q(UD)^{-1} = QR
    \end{array}
    \]
    
    Матрица $R$ — верхнетреугольная обратимая.


\end{frame}


\begin{frame}
    \frametitle{$QR$-разложение}

    \begin{block}{Утверждение}
        Любая квадратная матрица $V$ может быть представлена в виде
        \[
        V = QR, 
        \]
        где матрица $Q$ ортогональная, $Q^T Q= \Id$, а матрица $R$ — верхнетреугольная. \pause
    \end{block}
    
    Утверждение верно, даже если $V$ — необратимая матрица.

    В этом случае матрица $R$ также будет необратимой.

\end{frame}