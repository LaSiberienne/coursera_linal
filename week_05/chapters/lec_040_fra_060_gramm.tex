% !TEX root = ../linal_lecture_05.tex

\begin{frame} % название фрагмента

\videotitle{Матрица Грамма}

\end{frame}



\begin{frame}{Краткий план:}
  \begin{itemize}[<+->]
    \item Матрица Грама.
    \item Матрица Грама и проекция.
    \item Ортогональный базис.
  \end{itemize}

\end{frame}

\begin{frame}
    \frametitle{Матрица Грама}

    \begin{block}{Определение}
        Возьмём векторы $\bx_1$, $\bx_2$, \ldots, $\bx_k$ из $\R^n$. 
        Матрица их попарных скалярных произведений называется \alert{матрицей Грама},
        \[
            M =  \begin{pmatrix}
                \langle \bx_1, \bx_1 \rangle &  \langle \bx_1, \bx_2 \rangle & \ldots & \langle \bx_1, \bx_k \rangle \\
                \langle \bx_2, \bx_1 \rangle &  \langle \bx_2, \bx_2 \rangle & \ldots & \langle \bx_2, \bx_k \rangle \\
                \ldots & \ldots & \ldots & \ldots \\
                \langle \bx_k, \bx_1 \rangle &  \langle \bx_k, \bx_2 \rangle & \ldots & \langle \bx_k, \bx_k \rangle \\ 
            \end{pmatrix} = X^T X\pause
        \]      

        А определитель этой матрицы называется \alert{определителем Грама}, $G=\det M$.
    \end{block}


\end{frame}


\begin{frame}
    \frametitle{Свойства матрицы Грама}

    \begin{block}{Утверждение}
        Векторы $\bx_1$, $\bx_2$, \ldots, $\bx_k$ линейно независимы если и только если определитель Грама отличен от нуля, $G\neq 0$. \pause
    \end{block}

    \begin{block}{Утверждение}
        Матрица Грама положительно полуопределена. \pause
    \end{block}


    \begin{block}{Утверждение}
        Если $\bx_1$, $\bx_2$, \ldots, $\bx_n$ лежат в $\R^n$, то определитель Грама $G$ равен квадрату объёма параллелепипеда, 
        образованного векторами $\bx_1$,  $\bx_2$, \ldots, $\bx_n$.
    \end{block}

\end{frame}

\begin{frame}{Положительная полуопределённость}

    \begin{block}{Утверждение}
        Матрица Грама положительно полуопределена. \pause
    \end{block} 

    \begin{block}{Доказательство}
        \[
        \bv^T M \bv  = \sum_{ij} v_i v_j \langle \bx_i, \bx_j \rangle = \sum_{ij} \langle v_i \bx_i, v_j \bx_j \rangle =   \pause
        \]
        \[
        =  \langle \sum_i v_i  \bx_i, \sum_j v_j \bx_j \rangle = \langle \ba, \ba \rangle \geq 0
        \]
    \end{block}
    
\end{frame}



\begin{frame}
    \frametitle{Поиск проекции}
    Хотим найти проекцию $\bhy$ вектора  $\by$ на $\Span\{ \bx_1, \bx_2, \ldots, \bx_k \}$. \pause

    Проекция $\bhy$ — линейная комбинация $\bx_1, \bx_2, \ldots, \bx_k$,
\[
\bhy = v_1 \bx_1 + \ldots + v_k \bx_k = X \bv    \pause
\]
Условия первого порядка:
\[
X^T X  \bv = X^T y \pause \; \text{ или } \; M \bv = X^Ty    \pause 
\]    
\[
    \bv = M^{-1} X^Ty.    
\]
\end{frame}


\begin{frame}
    \frametitle{Ортогональные вектора}

    \begin{block}{Утверждение}
        Если векторы $\bx_1$, $\bx_2$, \ldots, $\bx_k$ ортогональны, то их матрица Грама —
        диагональная.
        \[
            M  = \begin{pmatrix}
                \langle \bx_1, \bx_1 \rangle & 0 & \ldots & 0 \\
                0 & \langle \bx_2, \bx_2 \rangle &  \ldots & 0 \\
                \ldots & \ldots & \ldots & \ldots \\
                0 & 0 & \ldots & \langle \bx_k, \bx_k \rangle \\
            \end{pmatrix}
        \]      

    \end{block}
    

\end{frame}