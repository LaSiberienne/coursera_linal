% !TEX root = ../linal_lecture_05.tex

\begin{frame} % название фрагмента

\videotitle{Квадратичная форма}

\end{frame}



\begin{frame}{Краткий план:}
  \begin{itemize}[<+->]
    \item Определение квадратичной формы.
    \item Определённость формы.
  \end{itemize}

\end{frame}



\begin{frame}
    \frametitle{Квадратичная форма}    

    \begin{block}{Определение}
       Многочлен от нескольких переменных $f(x_1, x_2, \ldots, x_n)$, который содержит только слагаемые вида $x_i^2$ и $x_i x_j$ 
       \alert{квадратичной формой}.
    \end{block}

    \pause
    Функция $f(x,y) = x^2 + 6xy - 7y^2$ — квадратичная форма.

    \pause
    Функция $f(x, y, z) = x^2 + 6xz - 8xy + 3z + 9$ — не квадратичная форма. 
\end{frame}


\begin{frame}{Зачем нужны квадратичные формы?}
    
    Многие функции хорошо аппроксимируются суммой вида
    \[
    f(x, y) \approx 6 + 2x + 4y + 7x^2 + 8xy - 9y^2
    \]
    \pause
    Свойства квадратичной формы позволяют понять свойства многих функций!
    
\end{frame}


\begin{frame}{Квадратичная форма и матрицы}
\[
\begin{pmatrix}
    x_1 & x_2 & x_3 
\end{pmatrix} \cdot \begin{pmatrix}
    5 & \red{-1} & \blue{-3} \\
    \red{-1} & 7 & 2 \\
    \blue{-3} & 2 & 11 \\
\end{pmatrix} \cdot 
\begin{pmatrix}
    x_1 \\
     x_2 \\ 
     x_3 \\
\end{pmatrix} = \pause
\]
\[
 = 5x_1^2 + 7x_2^2 + 11x_3^2 \red{- 2}x_1 x_2 \blue{- 6}x_1 x_3 + 4 x_2 x_3 \pause
\]
    
\begin{block}{Утверждение}
    Любая квадратичная форма $f(\bx)$ может быть записана в виде 
    \[
      f(\bx) = \bx^T A \bx,  
    \]
    где $A$ — симметричная матрица, $A^T = A$.
\end{block}


\end{frame}


\begin{frame}
    \frametitle{Квадратичные формы в нуле}

    \begin{block}{Утверждение}
        Любая квадратичная форма $f$ равна $0$ в точке $\bzero$,
        \[
        f(\bzero)  = \bzero^T \cdot A \cdot \bzero = 0.
        \]
    \end{block}
    \pause
    Нас будет интересовать знак квадратичной формы $f(\bx)$ при $\bx \neq \bzero$.

\end{frame}


\begin{frame}
    \frametitle{Положительно определённая форма}

    \begin{block}{Определение}
        Форма $f$ называется \alert{положительно определённой}, если $f(\bx) > 0$ при $\bx\neq\bzero$.
    \end{block}

    

\end{frame}