% !TEX root = ../linal_lecture_05.tex

\begin{frame} % название фрагмента

\videotitle{Критерий Сильвестра}

\end{frame}



\begin{frame}{Краткий план:}
  \begin{itemize}[<+->]
    \item Критерий Сильвестра.
    \item Расширенный критерий Сильвестра.
  \end{itemize}

\end{frame}


\begin{frame}
    \frametitle{Обозначение}

    Будем вычёркивать из матрицы $A$ строки и столбцы так, чтобы остались строки и столбцы с одинаковыми номерами.\pause

    Скажем, оставим в матрице $A$ только $2$-ю и $4$-ю строки и 
    $2$-й и $4$-й столбцы. \pause

    Определитель полученной подматрицы обозначим $m_{24}$. \pause

    Пример. 
    \[
    A = \begin{pmatrix}
        5 & 2 & 3 & -1 \\
        2 & \blue{6} & 2 & \blue{1} \\
        3 & 2 & 9 & 5 \\
        -1 & \blue{1} & 5 & \blue{8} \\
    \end{pmatrix}, \; m_{24} = \begin{vmatrix}
        \blue{6} & \blue{1} \\
        \blue{1} & \blue{8} \\
    \end{vmatrix} = 47.
    \]
    
\end{frame}


\begin{frame}
\frametitle{Названия миноров}

\begin{block}{Определения}
    В матрице $A$ вычеркнули несколько строк и столбцов так,
    что остались  строки и столбцы с одинаковыми номерами.

    Определитель полученной подматрицы называется \alert{главным минором}.
\end{block}
\pause

\begin{block}{Определения}
    В матрице $A$ вычеркнули несколько строк и столбцов так,
    что остались  строки и столбцы с номерами $1$, $2$, \ldots, $k$.

    Определитель полученной подматрицы называется \alert{угловым минором}.
\end{block}

\pause
\begin{block}{Определение}
    \alert{Порядком} минора называется число строк (или столбцов) в соответствующей подматрице.    
\end{block}


\end{frame}




\begin{frame}
    \frametitle{Критерий Сильвестра}

    \begin{block}{Утверждение}
        Симметричная матрица $A$ является положительно определённой, если и только если

        $m_1 > 0$, $m_{12} > 0$, $m_{123} > 0$, $m_{1234}>0$, \ldots   \pause     
    \end{block}

    Пример. 
\[
A = \begin{pmatrix}
    \blue{5} & 2 & \blue{3}  \\
    2 & 6 & \blue{2} \\
    \blue{3} & \blue{2} & \blue{9} \\
\end{pmatrix}
\]
\[
    m_1 = 5, \; m_{12} = \begin{vmatrix}
        5 & 2 \\
        2 & 6
    \end{vmatrix} = 26, \; 
    m_{123} = \begin{vmatrix}
        5 & 2 & 3 \\
        2 & 6 & 2 \\
        3 & 2 & 9
    \end{vmatrix}=  184
\]
    
\end{frame}


\begin{frame}
    \frametitle{Наблюдение}

    \begin{block}{Утверждение}
    Если помножить на $(-1)$ все элементы матрицы $A$ размера $n\times n$, то определитель матрицы $A$\ldots \pause

    поменяет знак, если $n$ — нечётное; \pause

    сохранит знак, если $n$ — чётное. 
    \end{block}

    \pause
    Легко получим критерий отрицательной определённости!


\end{frame}

    

\begin{frame}
\frametitle{Критерий Сильвестра}

\begin{block}{Утверждение}
    Симметричная матрица $A$ является отрицательно определённой, если и только если

    $\blue{m_1 < 0}$, $\red{m_{12} > 0}$, $\blue{m_{123} < 0}$, $\red{m_{1234}>0}$, \ldots   \pause   
\end{block}

Пример. 
\[
B = \begin{pmatrix}
    \blue{-5} & -2 & \blue{-3}  \\
    -2 & -6 & \blue{-2} \\
    \blue{-3} & \blue{-2} & \blue{-9} \\
\end{pmatrix}
\]
\[
    m_1 = -5, \; m_{12} = \begin{vmatrix}
        -5 & -2 \\
        -2 & -6
    \end{vmatrix} = 26, \; 
    m_{123} = \begin{vmatrix}
        -5 & -2 & -3 \\
        -2 & -6 & -2 \\
        -3 & -2 & -9
    \end{vmatrix}=  -184
\]



\end{frame}







\begin{frame}
    \frametitle{Расширенный критерий}

    \begin{block}{Утверждение}
        Симметричная матрица $A$ является положительно полуопределённой, если и только если (для всех $i$, $j$, $k$, \ldots)

        $m_i \geq 0$,  $m_{ij} \geq 0$, $m_{ijk} \geq 0$, $m_{ijkl} \geq 0$, \ldots \pause
    \end{block}
\[
A = \begin{pmatrix}
    4 & 6 \\
    6 & 9 \\ 
\end{pmatrix}
\]
\[
    m_1 = 4, \; m_2 = 9, \; m_{12} = \begin{vmatrix}
        4 & 6 \\
        6 & 9
    \end{vmatrix} = 0
\]
    
\end{frame}




\begin{frame}
    \frametitle{Расширенный критерий}

    \begin{block}{Утверждение}
        Симметричная матрица $A$ является отрицательно полуопределённой, если и только если (для всех $i$, $j$, $k$, \ldots)

        $\blue{m_i \leq 0}$,  $\red{m_{ij} \geq 0}$, $\blue{m_{ijk} \leq 0}$, $\red{m_{ijkl} \geq 0}$, \ldots \pause
    \end{block}
\[
A = \begin{pmatrix}
    -4 & 6 \\
    6 & -9 \\ 
\end{pmatrix}
\]
\[
    m_1 = -4, \; m_2 = -9, \; m_{12} = \begin{vmatrix}
        -4 & 6 \\
        6 & -9
    \end{vmatrix} = 0
\]
    
\end{frame}


\begin{frame}
    \frametitle{Резюме для положительной формы}

    \begin{block}{Утверждение}
        Квадратичная форма $f(\bx) = \bx^T A\bx$ является положительно определённой, если \pause

        \begin{enumerate}
            \item В любой точке $\bx\neq \bzero$ она положительна, $f(\bx) > 0$. \pause
            \item Все собственные числа матрицы $A$ положительны, $\lambda_i > 0$. \pause
            \item Все угловые миноры матрицы $A$ положительны, $m_{12\ldots k} > 0$.
        \end{enumerate}
            
    \end{block}


\end{frame}


\begin{frame}
    \frametitle{Резюме для отрицательной формы}

    \begin{block}{Утверждение}
        Квадратичная форма $f(\bx) = \bx^T A\bx$ является отрицательно определённой, если \pause

        \begin{enumerate}
            \item В любой точке $\bx\neq \bzero$ она отрицательна, $f(\bx) < 0$. \pause
            \item Все собственные числа матрицы $A$ отрицательны, $\lambda_i < 0$. \pause
            \item Нечётные угловые миноры матрицы $A$ отрицательны, а чётные — положительны.
        \end{enumerate}
            
    \end{block}


\end{frame}




\begin{frame}
    \frametitle{Резюме для полуопределённости}

    \begin{block}{Утверждение}
        Квадратичная форма $f(\bx) = \bx^T A\bx$ является положительно полуопределённой (неотрицательно определённой), если \pause

        \begin{enumerate}
            \item В любой точке $\bx$ она неотрицательна, $f(\bx) \geq 0$. \pause
            \item Все собственные числа матрицы $A$ неотрицательны, $\lambda_i \geq 0$. \pause
            \item Все главные миноры матрицы $A$ неотрицательны.
        \end{enumerate}
            
    \end{block}


\end{frame}



\begin{frame}
    \frametitle{Резюме для полуопределённости}

    \begin{block}{Утверждение}
        Квадратичная форма $f(\bx) = \bx^T A\bx$ является отрицательно полуопределённой (неположительно определённой), если \pause

        \begin{enumerate}
            \item В любой точке $\bx$ она неположительна, $f(\bx) \leq 0$. \pause
            \item Все собственные числа матрицы $A$ неположительны, $\lambda_i \leq 0$. \pause
            \item Нечётные главные миноры матрицы $A$ неположительны, а чётные — неотрицательны.
        \end{enumerate}
            
    \end{block}


\end{frame}
